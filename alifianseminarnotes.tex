\documentclass[a4paper,12pt]{article}
\usepackage[a4paper, hmargin={3cm,2cm}, vmargin={2cm,2cm}]{geometry}
\usepackage{amsmath}
\usepackage{amsthm}
\usepackage{amsfonts}
\usepackage{color}
\usepackage[final]{graphicx}
\usepackage{subcaption}
\usepackage{wrapfig}
\newtheorem{remark}{Remark}[]
\newtheorem{prop}{Proposition}
\usepackage{amssymb}

\newcommand{\R}{\mathbb{R}}
\newcommand{\C}{\mathbb{C}}


% Title Page
\title{Seminar Notes Alifian}
\author{Alifian Mahardhika Maulana}


\begin{document}
\maketitle
\section{3D Linear Elasticity}
\begin{equation}\nonumber
\begin{aligned}
\Omega \subset \R^d & (d=2,3) \\
u = \Omega & \rightarrow \R^2 \text{(small displacement)}\\
x & \mapsto u(x)
\end{aligned}
\end{equation}
\subsection{Strain Tensor}
\begin{equation}\nonumber
\begin{aligned}
e[u] & = (e_{ij}[u]) \in \R^{d x d}_{sym}\\
e[u] & := \frac{1}{2} (\bigtriangledown^Tu + (\bigtriangledown^Tu)^T)
\end{aligned}
\end{equation}
\subsection{Stress Tensor}
\begin{equation}\nonumber
\begin{aligned}
\sigma[u] &= (\sigma{ij}[u]) \in \R^{d x d}_{sym}\\
\end{aligned}
\end{equation}
Based on Hook's Law, stress tensor must have equality with strain so that
\begin{equation}\nonumber
\begin{aligned}
\sigma &= \C e\\
\text{with } \C &= \C_{ijkl} \text{(is a 4th order elasticity tensor)}\\
\sigma{ij} &= \C_{ijkl} e_{kl}\\
\C_{ijkl} &= \C_{ijlk} = \C_{klij} \text{(symmetry)}\\
\C_{ijkl} \xi_{ij} \xi_{kl} & \geq \C_* |\xi|^2
\end{aligned}
\end{equation}
\subsection{Boundary Value Problem}
\begin{equation}
\begin{cases}
-\partial_i \sigma_{ij}[u] &= f_j(x), x \in \Omega\\
u &= g(x), x \in \Gamma_D\\
\sigma[u]_\nu &= q(x), x \in \Gamma_N
\end{cases}
\end{equation}
\newpage
\subsection{Equilibrium Equations of Force in $\Omega$ and on $\Gamma_N$}
\subsubsection{Strain Energy Density}
\begin{equation}
\label{energy_density}
\omega[u](x) := \frac{1}{2} \sigma[u] : e[u]
\end{equation}
Solving using Sobolev Space in Isotropic Case, equation \ref{energy_density} becomes
\begin{equation}\nonumber
c_{ijkl} = \lambda \delta_{ij} \delta_{kl} + \mu(\delta_{ik} \delta_{jl} + \delta_{il} \delta_{jk})
\end{equation}
with $\lambda, \mu$ called Lame Constant
\begin{equation}\nonumber
\delta_{ij} = \begin{cases}
1, i = j\\
0, i \neq j
\end{cases}
\end{equation}
\begin{equation}\nonumber
\begin{aligned}
\sigma[u] &= (\sigma_{ij}[u])\\
\sigma_{ij}[u] &= c_{ijkl} e_{kl}[u]\\
&= \lambda(\delta_k u_k)\delta_{ij} + \mu(\delta_i u_j + \delta_j u_i)\\
&= \lambda(\text{div } u) I + 2\mu e[u]
\end{aligned}
\end{equation}
\begin{equation}\nonumber
\begin{aligned}
\omega[u] = \frac{1}{2} (\lambda(\text{div } u) I + 2\mu e[u]) : e[u]\\
\omega[u] = \frac{1}{2} (\lambda(\text{div } u)^2 + \mu |e[u]|^2\\
\end{aligned}
\end{equation}
\begin{remark}
Positivity of $\C$
\begin{equation}\nonumber
\begin{aligned}
(\C \xi) : \xi \geq C_* |\xi|^2 (\forall \xi \in \R^{d x d}_{sym})\\
(\C \xi) : \xi = \lambda |\text{tr }\xi|^2 + 2\mu|\xi|^2\\
\end{aligned}
\end{equation}
If $\lambda \geq 0, \mu > 0$, then $C_* = 2 \mu$
\begin{equation}\nonumber
\xi = (\xi_{ij}), |\xi|^2 = \xi_{ij} \xi_{ij} = \sum_{i = 1 \dots d}^{d} \sum_{j = 1 \dots d}^{d} |\xi_{ij}|^2
\end{equation}
\end{remark}
\subsection{Elasticity Problem}
\begin{equation}\nonumber
\begin{cases}
-\text{div } \sigma[u] &= f(x) \text{ in } \Omega \subset \R^d\\
u &= g(x) \text{ on } \Gamma_D\\
\sigma[u]v &= q(x) \text{ on } \Gamma_N
\end{cases}
\end{equation}
\subsection{Crack Problem}
\begin{equation}\nonumber
\begin{cases}
-\text{div } \sigma[u] &= f(x) \text{ in } \Omega \setminus \Sigma \subset \R^d\\
u &= g(x) \text{ on } \Gamma_D\\
\sigma[u]v &= q(x) \text{ on } \Gamma_N\\
\sigma[u]v &= 0 \text{ on } \Sigma^+ \cup \Sigma^-
\end{cases}
\end{equation}
\end{document}          
