\documentclass[a4paper,12pt]{article}
\usepackage[a4paper, hmargin={3cm,2cm}, vmargin={2cm,2cm}]{geometry}
\usepackage{amsmath}
\usepackage{amsthm}
\usepackage{amsfonts}
\usepackage{color}
\usepackage[final]{graphicx}
\usepackage{subcaption}
\usepackage{wrapfig}
\newtheorem{remark}{Remark}[]
\newtheorem{prop}{Properties}
\newtheorem{theorem}{Theorem}
\usepackage{amssymb}
\usepackage{enumitem}

\newcommand{\R}{\mathbb{R}}
\newcommand{\C}{\mathbb{C}}
\newcommand{\N}{\mathbb{N}}
\newcommand{\Q}{\mathbb{Q}}
\newcommand{\W}{\mathbb{W}}
\newcommand{\Vspace}{\mathbb{V}}
\newcommand{\Hspace}{\mathbb{H}}
\newcommand{\Lspace}{\mathbb{L}}
\newcommand{\Lagr}{\mathcal{L}}
\newcommand{\Cmod}{\mathcal{C}}

\DeclareMathOperator*{\argmin}{argmin}

% Title Page
\title{Seminar Notes Alifian}
\author{Alifian Mahardhika Maulana}


\begin{document}
\maketitle
\section{3D Linear Elasticity}
\begin{equation}\nonumber
\begin{aligned}
\Omega \subset \R^d & (d=2,3) \\
u = \Omega & \rightarrow \R^2 \text{(small displacement)}\\
x & \mapsto u(x)
\end{aligned}
\end{equation}
\subsection{Strain Tensor}
\begin{equation}
\begin{aligned}
e[u] & = (e_{ij}[u]) \in \R^{d x d}_{sym}\\
e[u] & := \frac{1}{2} (\bigtriangledown^Tu + (\bigtriangledown^Tu)^T)
\end{aligned}
\end{equation}
\subsection{Stress Tensor}
\begin{equation}
\begin{aligned}
\sigma[u] &= (\sigma{ij}[u]) \in \R^{d x d}_{sym}\\
\end{aligned}
\end{equation}
Based on Hook's Law, stress tensor must have equality with strain so that
\begin{equation}\nonumber
\begin{aligned}
\sigma &= \textbf{C} e\\
\text{with } \textbf{C} &= \textbf{C}_{ijkl} \text{(is a 4th order elasticity tensor)}\\
\sigma{ij} &= \textbf{C}{ijkl} e_{kl}\\
\textbf{C}_{ijkl} &= \textbf{C}_{ijlk} = \textbf{C}_{klij} \text{(symmetry)}\\
\textbf{C}_{ijkl} \xi_{ij} \xi_{kl} & \geq \textbf{C}_* |\xi|^2
\end{aligned}
\end{equation}
\subsection{Boundary Value Problem}
\begin{equation}
\begin{cases}
-\partial_i \sigma_{ij}[u] &= f_j(x), x \in \Omega\\
u &= g(x), x \in \Gamma_D\\
\sigma[u]_\nu &= q(x), x \in \Gamma_N
\end{cases}
\end{equation}
\newpage
\subsection{Equilibrium Equations of Force in $\Omega$ and on $\Gamma_N$}
\subsubsection{Strain Energy Density}
\begin{equation}
\label{energy_density}
\omega[u](x) := \frac{1}{2} \sigma[u] : e[u]
\end{equation}
Solving using Sobolev Space in Isotropic Case, equation \ref{energy_density} becomes
\begin{equation}\nonumber
c_{ijkl} = \lambda \delta_{ij} \delta_{kl} + \mu(\delta_{ik} \delta_{jl} + \delta_{il} \delta_{jk})
\end{equation}
with $\lambda, \mu$ called Lame Constant
\begin{equation}\nonumber
\delta_{ij} = \begin{cases}
1, i = j\\
0, i \neq j
\end{cases}
\end{equation}
\begin{equation}\nonumber
\begin{aligned}
\sigma[u] &= (\sigma_{ij}[u])\\
\sigma_{ij}[u] &= c_{ijkl} e_{kl}[u]\\
&= \lambda(\delta_k u_k)\delta_{ij} + \mu(\delta_i u_j + \delta_j u_i)\\
&= \lambda(\text{div } u) I + 2\mu e[u]
\end{aligned}
\end{equation}
\begin{equation}\nonumber
\begin{aligned}
\omega[u] = \frac{1}{2} (\lambda(\text{div } u) I + 2\mu e[u]) : e[u]\\
\omega[u] = \frac{1}{2} (\lambda(\text{div } u)^2 + \mu |e[u]|^2\\
\end{aligned}
\end{equation}
\begin{remark}
Positivity of \textbf{C}
\begin{equation}\nonumber
\begin{aligned}
(\textbf{C} \xi) : \xi \geq \textbf{C}_* |\xi|^2 (\forall \xi \in \R^{d x d}_{sym})\\
(\textbf{C} \xi) : \xi = \lambda |\text{tr }\xi|^2 + 2\mu|\xi|^2\\
\end{aligned}
\end{equation}
If $\lambda \geq 0, \mu > 0$, then $C_* = 2 \mu$
\begin{equation}\nonumber
\xi = (\xi_{ij}), |\xi|^2 = \xi_{ij} \xi_{ij} = \sum_{i = 1 \dots d}^{d} \sum_{j = 1 \dots d}^{d} |\xi_{ij}|^2
\end{equation}
\end{remark}
\subsection{Elasticity Problem}
\begin{equation}
\begin{cases}
-\text{div } \sigma[u] &= f(x) \text{ in } \Omega \subset \R^d\\
u &= g(x) \text{ on } \Gamma_D\\
\sigma[u]v &= q(x) \text{ on } \Gamma_N
\end{cases}
\end{equation}
\subsection{Crack Problem}
\begin{equation}
\begin{cases}
-\text{div } \sigma[u] &= f(x) \text{ in } \Omega \setminus \Sigma \subset \R^d\\
u &= g(x) \text{ on } \Gamma_D\\
\sigma[u]v &= q(x) \text{ on } \Gamma_N\\
\sigma[u]v &= 0 \text{ on } \Sigma^+ \cup \Sigma^-
\end{cases}
\end{equation}
\subsection{Lebesque Measurable Theory}
\begin{equation}
L^p(\Omega) := \Bigg\{v:\Omega \rightarrow \R|\begin{cases}
v = \text{Lebesque measurable}\\
\int_\Omega|v(x)|^p dx < \infty
\end{cases}\Bigg\}
\end{equation}
\begin{remark}
	for $u,v \in \Lspace^p(\Omega), \text{ if } \exists N \subset \Omega$
	such that $\begin{cases}
		u(x) = v(x) (x \in \Omega \setminus \text{N})\\
		\Lagr^d(N) = 0,
	\end{cases}$
	then we identify $u$ and $v$,
	$\Lagr^d(N) = 0 \Leftrightarrow \text{volume of N } = 0$
	for simplicity, we also can say that
	$u(x) = v(x) \text{ for a.e. } x \in \Omega$
\end{remark}
for example
\begin{equation}
\begin{aligned}
&v:\R \rightarrow \R\\
&v(x) = \begin{cases}
1, x \in \Q\\
0, x \in \R \setminus \Q\\
\end{cases}\\
&\int_\R v dx = 0, \Lagr^1(\Q) = 0\\
&v(x) = 0 \text{ on } \R \setminus \Q, \text{ or we can say } v = 0 \text{ a.e. in } \R
\end{aligned}
\end{equation}
\subsection{Sobolev Space}
\begin{equation}
\W^{1,p}(\Omega) := \bigg\{v\in\Lspace^p(\Omega)\frac{\partial v}{\partial x_j}_{(j=1\dots d)}\in\Lspace^p(\Omega)\bigg\}
\end{equation}
such $\frac{\partial v}{\partial x_j}$ we called it distribution sence.\\
example of Sobolev Space is as follow:
\begin{equation}\nonumber
v \in \Lspace^p(\Omega) \text{ if } \exists \omega_j \in \Lspace(\Omega)
\end{equation}
such that
\begin{equation}\nonumber
\begin{aligned}
\int_\Omega v \frac{\partial \varphi}{\partial x_j} dx = -\int_\Omega \omega_j \varphi dx (\forall \varphi \in \C_0^\infty(\Omega))\\
\Rightarrow \frac{\partial \varphi}{\partial x_j} = \omega_j \text{ in distribution sence}
\end{aligned}
\end{equation}
for
\begin{equation}\nonumber
\begin{aligned}
v \in \C^1(\Omega), &\frac{\partial v}{\partial x_j} (x) = \omega_j(x)\\
&\Updownarrow\\
\int_\Omega \omega_j \varphi dx &= -\int_\Omega v \frac{\partial \varphi}{\partial x_j}dx (\forall \varphi \in \C_0^\infty(\Omega))
\end{aligned}
\end{equation}
In particular,
\begin{equation}\nonumber
\Hspace^1(\Omega) := \W^{1,2}(\Omega), \bigtriangledown u = \left(\begin{array}{c}
\frac{\partial u}{\partial x_1}\\
\vdots\\
\frac{\partial u}{\partial x_d}
\end{array}\right)
\end{equation}
inner product
\begin{equation}\nonumber
(u,v)_{\Hspace^1(\Omega)} := \int_\Omega uv\ dx + \int_\Omega \bigtriangledown u \cdot \bigtriangledown v\ dx
\end{equation}
norm
\begin{equation}\nonumber
||u||_{\Hspace^1(\Omega)} := \sqrt{(u,v)_{\Hspace^1(\Omega)}} = \sqrt{\int_\Omega|u|^2dx + \int_\Omega|\bigtriangledown u|^2dx}
\end{equation}
$\Hspace^1(\Omega)$ is complete ($\Hspace^1(\Omega)$ is a Hilbert Space)
\begin{equation}\nonumber
(u,v)_{\Lspace^2(\Omega)} = \int_\Omega uv dx
\end{equation}
\subsection{Incomplete Hilbert Space}
$\Vspace : \text{ a vector space in } \R$
\begin{equation}\nonumber
\begin{aligned}
\begin{cases}
u,v\ \in \Vspace \Rightarrow \alpha u + \beta v \in \Vspace\\
\alpha,\beta\ \in \R
\end{cases}
\end{aligned}
\end{equation}
$\text{If } (\cdot,\cdot) : \Vspace \times \Vspace \rightarrow \R \text{ satisfies}$
\begin{equation}\nonumber
\begin{aligned}
\begin{cases}
(u\cdot v) \geq 0 \text{ and } (u,u) = 0 \Leftrightarrow u = 0_v \in \Vspace\\
(u,v) = (v,u)\\
(\alpha u + \beta v, \omega) = \alpha(u,\omega) + \beta (v,\omega)
\end{cases}
\end{aligned}
\end{equation}
then we call $[\Vspace \times \Vspace]$ pre Hilbert space or incomplete Hilbert Space.
\subsection{Property of $\Lspace^2(\Omega)$}
For $v \in \C^1(\Omega),$
\begin{equation}\nonumber
\begin{aligned}
\frac{\partial v}{\partial x_j}(x)&=w_j(x)\\
&\Updownarrow\\
\int_\Omega w_j\varphi dx &= -\int\Omega v \frac{\partial \varphi}{\partial x_j} dx\ \big(\forall\varphi\in\C_0^\infty(\Omega)\big)\\
\end{aligned}
\end{equation}
$(u,v)_{\Lspace^2(\Omega)} = \int_\Omega uv dx\\$
\begin{equation}\nonumber
\Rightarrow \bigg|\int_{\Omega}uvdx\bigg| \leq \int_{\Omega}|u||v|dx \leq ||u||_{\Lspace^2(\Omega)} ||v||_{\Lspace^2(\Omega)}
\end{equation}
$u,v \in \Hspace^1(\Omega)$
\begin{equation}\nonumber
\begin{aligned}
\Rightarrow \frac{\partial u}{\partial x_j}, \frac{\partial v}{\partial x_j} & \in {\Lspace^2(\Omega)}\\
\bigg| \int_\Omega \frac{\partial u}{\partial x_j} \frac{\partial v}{\partial x_j}dx\bigg| \leq \int_\Omega \bigg| \frac{\partial u}{\partial x_j} \bigg| \bigg| \frac{\partial v}{\partial x_j} \bigg| dx & \leq \bigg|\bigg| \frac{\partial u}{\partial x_j}\bigg|\bigg|_{\Lspace^2(\Omega)} \bigg|\bigg| \frac{\partial v}{\partial x_j} \bigg|\bigg|_{\Lspace^2(\Omega)}\\
\triangledown u \cdot \triangledown v &= \sum_{j=1}^{d}\frac{\partial u}{\partial x_j} \frac{\partial v}{\partial x_j}\\
\bigg| \int_{\Omega} \triangledown u \cdot \triangledown v dx \bigg| &= \bigg| \int \sum_{j=1}^{d}\frac{\partial u}{\partial x_j} \frac{\partial v}{\partial x_j} dx \bigg| \leq  \sum_{j=1}^{d} \int \bigg| \frac{\partial u}{\partial x_j} \bigg| \bigg|  \frac{\partial v}{\partial x_j} \bigg| dx\\
&\leq \sum_{j=1}^{d} \bigg|\bigg| \frac{\partial u}{\partial x_j} \bigg|\bigg|_{\Lspace^2(\Omega)} \bigg|\bigg| \frac{\partial v}{\partial x_j} \bigg|\bigg|_{\Lspace^2(\Omega)}\\
&\leq \sqrt{\sum_{j=1}^{d} \bigg|\bigg| \frac{\partial u}{\partial x_j} \bigg|\bigg|_{\Lspace^2(\Omega)}} \sqrt{\sum_{j=1}^{d} \bigg|\bigg| \frac{\partial v}{\partial x_j} \bigg|\bigg|_{\Lspace^2(\Omega)}}\\
&= \sqrt{\sum_{j=1}^{d}\int_\Omega\bigg|\frac{\partial u}{\partial x_j}\bigg|^2 dx} \sqrt{\sum_{j=1}^{d}\int_\Omega\bigg|\frac{\partial v}{\partial x_j}\bigg|^2 dx}\\
&= \sqrt{\int_\Omega\bigg(\sum_{j=1}^{d}\bigg|\frac{\partial u}{\partial x_j}\bigg|^2\bigg) dx} \sqrt{\int_\Omega\bigg(\sum_{j=1}^{d}\bigg|\frac{\partial v}{\partial x_j}\bigg|^2\bigg) dx}\\
&= \sqrt{\int_\Omega\big| \triangledown u \big|^2 dx} \sqrt{\int_\Omega\big| \triangledown v \big|^2 dx}\\
\end{aligned}
\end{equation}
\begin{equation}
\therefore \bigg| \int_{\Omega} \triangledown u \cdot \triangledown v dx \bigg| \leq \sqrt{\int_\Omega\big| \triangledown u \big|^2 dx} \sqrt{\int_\Omega\big| \triangledown v \big|^2 dx}
\end{equation}
\subsection{Energy (Revisited)}
\begin{equation}
E(u) := \frac{1}{2}\int_\Omega \sigma[u]:e[u]dx -\int_\Omega f\cdot u dx -\int_{\Gamma_N} q\cdot u ds
\end{equation}
with $u$ is a vector of the elasticity problem define by:
\begin{equation}\nonumber
\begin{aligned}
u\in\Hspace^1(\Omega:\R^d) &:= \{ u:\Omega \rightarrow\R^d |u=(u_i, \cdots, u_d), u_i\in\Hspace^1(\Omega) \}\\
&\Rightarrow E(u) < \infty
\end{aligned}
\end{equation}
$u:$ become solution $\Leftrightarrow u = argmin_{v\in\Hspace^1(\Omega:\R^d)}E(v)$
such a technique we call it variational principle.
\subsection{Variational Principle}
Let's consider a Poisson Equation Problem:
\begin{equation}\label{poissoneq}
\Omega \subset \R^d \begin{cases}
-\triangle u &= f(x) \in\Omega\\
u &= g(x) \text{ on } \Gamma_D\\
\frac{\partial u}{\partial v} &= q(x) \text{ on } \Gamma_N
\end{cases}
f\in L^2(\Omega), g\in H^1(\Omega), q\in L^2(\Gamma_N)
\end{equation}
\begin{remark}
	\begin{equation}\nonumber
	v \in H^1(\Omega) \Rightarrow \exists v|_\Gamma \in L^2(\Gamma)
	\end{equation}
	we choose $v \text{ on } L^2 \text{ because it will has value on the boundary }$
\end{remark}
\subsubsection{Definition of Weak Solution}
\begin{equation}\nonumber
u \in H^1(\Omega)\text{ s.t. } \begin{cases}
\int_\Omega \triangle u \cdot \triangle v dx = \int_\Omega fv dx + \int_{\Gamma_N} qv ds\\
\bigg(\forall v \in V := {v \in H^1(\Omega)\big| v|_{\Gamma_D}}\bigg)\\
v|_{\Gamma_D} = g|_{\Gamma_D} (v-g\in V)
\end{cases}
\end{equation}
$(v-g\in V)$ mean $(v\in V+g:={v+g|v\in V})$ with $V$ is an affine space.
\subsubsection{Definition of Strong Solution}
$u \in H^2(\Omega)$ and $u$ satisfies \eqref{poissoneq}
\begin{remark}
	\begin{equation}\nonumber
	\begin{aligned}
	H^2(\Omega):= \{ u\in L^2(\Omega) \frac{\partial u}{\partial x_j},\frac{\partial^2 u}{\partial x_i x_j} \in L^2(\Omega)\}\\
	u \in H^2(\Omega) \Rightarrow \frac{\partial u}{\partial x_j} \in H^1(\Omega)\\
	\frac{\partial u}{\partial v} = \sum v_i \frac{\partial v}{\partial x_i}
	\end{aligned}
	\end{equation}
\end{remark}
\begin{prop}
	\begin{equation}\nonumber
	u: \text{ strong solution } \Leftrightarrow \begin{cases}
	u: \text{ weak solution}\\
	u \in H^2(\Omega)
	\end{cases}
	\end{equation}
\end{prop}
\begin{proof}
	Let's consider one dimension for simplicity, then:
	\begin{equation*}
	\begin{aligned}
	(\Rightarrow)
	-\triangle u = f(x)\\
	-\frac{\partial^2 u}{\partial x^2} = f(x)
	\end{aligned}
	\end{equation*}
	Suppose we take $v,\forall v \in V:={v \in H^1(\Omega)|v|_{\Gamma_D} = 0}$, if $v$ is smooth enough, then we take integral for both side, thus the equation becomes:
	\begin{equation*}
	\begin{aligned}
	\int_\Omega -\frac{\partial^2 u}{\partial x^2} v dx = \int_\Omega f(x) v dx\\
	\int_\Omega -\frac{\partial}{\partial x}\frac{\partial u}{\partial x} v dx = \int_\Omega f(x) v dx\\
	\text{using integration by parts, the left handside equation becomes:}\\
	\int_\Omega -\frac{\partial}{\partial x}\frac{\partial u}{\partial x} v dx = \bigg[-v \cdot \frac{du}{dx}\bigg]_{\partial\Omega} + \int_\Omega \frac{du}{dx}\frac{dv}{dx} dx\\
	\text{consider dirichlet boundary condition, thus the equation becomes:}\\
	-\int_{\Gamma_N} qv ds + \int_\Omega \frac{du}{dx}\frac{dv}{dx} dx = \int_\Omega fv dx\\
	\text{in general}\\
	\int_\Omega \triangledown u \cdot \triangledown v dx = \int_\Omega fv dx + \int_{\Gamma_N} qv ds
	\end{aligned}
	\end{equation*}
\end{proof}
\begin{equation}\nonumber
\text{Energy } E(v) := \frac{1}{2} \int_\Omega |\triangledown v|^2 dx - \int_\Omega fv dx - \int_{\Gamma_N} qv ds
\end{equation}
\begin{theorem}\label{theorem1}
	\begin{equation}\nonumber
	u: \text{ weak solution } \Leftrightarrow u= \text{ argmin}_{v\in V+g} E(v)
	\end{equation}
\end{theorem}
\begin{theorem}\label{theorem2}
	\begin{equation}\nonumber
	\exists! u = \text{ argmin}_{v\in V+g} E(v)
	\end{equation}
\end{theorem}
\begin{proof}of Theorem \ref{theorem1}
	\begin{equation}\nonumber
	\begin{aligned}
		(\Leftarrow) \text{ If } u = \text{ argmin }_{w\in V+g} E(w)\\
		\text{since } u+fv \in V+g\ (\forall v \in V, \forall t\in \R)\\
		\frac{d}{dt} E(u+tv)|_{t=0} = 0 \text{ (First Variation)}\\
	\end{aligned}
	\end{equation}
	\begin{equation}\nonumber
	\begin{aligned}
	0 &= \frac{d}{dt} E(u+tv)|_{t=0}\\
	&= \frac{d}{dt}\bigg[ \frac{1}{2} \int_\Omega |\triangledown u|^2 + 2t \triangledown u \cdot \triangledown v + t^2 |\triangledown v|^2 dx - \int_\Omega fu dx -\int_{\Gamma_N} qu ds - t(\int_\Omega fv dx + \int_{\Gamma_N} qv ds) \bigg]_{t=0}\\
	&= \int_\Omega \triangledown u \cdot \triangledown v dx - \int_\Omega fv dx - \int_{\Gamma_N} qv ds\\
	&\therefore u=\text{ weak solution }
	\end{aligned}
	\end{equation}
\end{proof}
\begin{proof} of Theorem \ref{theorem1} $\Leftrightarrow$ Theorem \ref{theorem2}\\
	$(\Rightarrow)$ If u is a weak solution\\
	for any $w\in V+g,(v:= w-u \in V)$\\
	\begin{equation}\nonumber
	\begin{aligned}
	E(w) - E(u) &= E(u+v) - E(u)\\
	&= \int_\Omega(\triangledown u \cdot \triangledown v + \frac{1}{2} |\triangledown v|^2) dx - \int_\Omega fv dx -\int_{\Gamma_N} qv ds\\
	&= \frac{1}{2} \int_\Omega |\triangledown v|^2 dx \geq 0\\
	&\therefore E(w) \geq E(u) (\forall w \in V+g)\\
	&\therefore u = \text{ argmin}_{w\in V+g} E(w)
	\end{aligned}
	\end{equation}
\end{proof}
\section{Abstract Theory}
X: a real Hilbert Space (ex: $H^1(\Omega)$)\\
V: a closed subspace of X (ex: $V \subset H^1(\Omega)$) in case of Poisson Equation (Linear)
\subsection{Definition}
\begin{enumerate}
	\item a : X $\times$ X $\rightarrow \R$ is a bilinear form, if\\
	$\begin{cases}
	u \mapsto a(u,v)\ \text{is linear for all}\ v \in X\\
	v \mapsto a(u,v)\ \text{is linear for all}\ u \in X\\
	\end{cases}$
	\item a bilinear form $a(u,v)$ is \underline{bounded}, if $\exists a_0>0$ s.t. $|a(u,v)| \leq a_0 ||u||_x ||v||_x (\forall u,v\in X)$
	\item a bilinear form $a(\cdot,\cdot)$ is \underline{symmetric}, if $a(u,v) = a(v,u)(\forall u,v \in X)$
	\item a bilinear form $a(\cdot,\cdot)$ is \underline{coercive}, if $\exists \alpha > 0$ s.t. $a(u,u)\geq \alpha ||u||^2_x (\forall u \in X)$
\end{enumerate}
\begin{remark}
	A bilinear form $a(\cdot,\cdot)$ is bounded iff $a: X \times X \rightarrow \R$ is continuous
\end{remark}
\subsection{Definition}
\begin{enumerate}
	\item $l: x \rightarrow \R$ is a linear form, if $l: x \rightarrow \R\ (\in u \mapsto l(u))$ is linear
	\item A linear form $l$ is bounded, if $|l(u)| \leq \exists c||u||_x\ (\forall u \in x)$
\end{enumerate}
\begin{remark}
	A linear form $l$ is bounded iff $l: x \rightarrow \R$ is continuous.
\end{remark}
\subsection{Theorem (Lax - Milgram)}
We suppose $a(\cdot,\cdot)$ is a bounded bilinear form on $X\times X$, and $l$ is a bounded linear form on $X$.\\
If $a(\cdot,\cdot)$ is coercive on $V\times V$, then for any $g\in X,\\ \exists! u\in V+g\ \text{s.t.}\ a(u,v)=l(v)(\forall v \in V)$
\subsubsection{Example}
\begin{equation}\nonumber
\begin{aligned}
X=H^1(\Omega), V={v \in H^1(\Omega)|v|_{\Gamma_D}=0}\\
a(u,v) := \int_\Omega \triangledown u \cdot \triangledown v dx\\
l(v) := \int_\Omega fv dx + \int_{\Gamma_N} qv ds
\end{aligned}
\end{equation}
\underline{\textbf{Coercivity}}\\
\begin{equation}\nonumber
\begin{aligned}
\exists \alpha_0 > 0: \int_\Omega |\triangledown v|^2 dx &\geq \alpha_0 \bigg( \int_\Omega |\triangledown v|^2 dx + \int_\Omega |v|^2 dx \bigg)\\
&\Updownarrow\ \text{If we choose $\alpha_0$ so small}\\
\exists \alpha_1 > 0 : \int_\Omega |\triangledown v|^2 dx &\geq \alpha_1 \int_\Omega |v|^2 dx (\forall v \in V)
\end{aligned}
\end{equation}
\underline{\textbf{Boundedness}}\\
\begin{equation}\nonumber
|a(u,v)| \leq \sqrt{\int_\Omega |\triangledown u|^2 dx} \sqrt{\int_\Omega |\triangledown v|^2 dx} \leq ||u||_x ||v||_x
\end{equation}
\begin{equation}\nonumber
|l(v)| \leq ||f||_{L^2(\Omega)} ||v||_{L^2(\Omega)} + ||q||_{L^2(\Gamma_N)} ||v||_{L^2(\Gamma_N)} \leq \bigg( ||f||_{L^2(\Omega)} + C||q||_{L^2(\Gamma_N)}\bigg)||v||_x
\end{equation}
Another Example:\\
Energy $E(u) := \frac{1}{2} a(u,u) - l(u)$
\begin{theorem}(Variational Principle) for $g \in X,$
\begin{equation*}
\begin{cases}
u \in V+g\\
a(u,v) = l(v) (\forall v \in V) \Leftrightarrow\ \text{argmin}_{w \in V+g} E(w)
\end{cases}
\end{equation*}
\begin{proof}
	\begin{equation*}
	\begin{aligned}
	(\Leftarrow) \forall t \in \R,& \forall v \in V, u+tv \in V+g\\
	\frac{d}{dt} E(u+tv)|_{t=0} &= 0\\
	E(u+tv) &= \frac{1}{2} a(u+tv, u+tv) -l(u+tv)\\
	&= \frac{1}{2}\bigg(a(u,u)+2ta(u,v)+t^2a(v,v)-l(u)-tl(u)\bigg)
	\end{aligned}
	\end{equation*}
	\begin{equation*}
	\begin{aligned}
	a(u,v) &= l(v) (\forall v \in V)\\
	\Rightarrow\ \text{For}\ w \in V+g&\ (v:=w-u\in V)\\
	E(w) - E(v) &= \frac{1}{2} a(v,v) \geq \frac{1}{2} \alpha ||v||_x^2 \geq 0  
	\end{aligned}
	\end{equation*}
\end{proof}
\end{theorem}
\begin{remark} Uniqueness in Lax-Milgram
	If $u$ and $w$ are both the minimizer of $E$ among $V+g$, then
	\begin{equation*}
	0=E(w) - E(u) = \frac{1}{2} a(v,v) \geq \frac{\alpha}{2} ||v||^2_x \geqq 0
	\end{equation*}
	$\therefore v=0,\ \therefore w = u$
\end{remark}
\section{Linear Elasticity}
We define:
\begin{equation*}
\begin{aligned}
&\Omega \subset \R^d\ (d=2,3)\\
&u:\Omega \rightarrow \R^d\ \text{(displacement)}\\
&e[u]:= \frac{1}{2}(\triangledown^T u + \triangledown u^T)\ \text{(strain)}\\
&\sigma[u]:=\Cmod e[u]\\
&\Cmod=(C_{ijkl})\begin{cases}
C_{ijkl}=C_{klij}=C_{jikl}\\
(C_\xi):\xi \geq C_* |\xi|^2 (\forall \xi \in \R_{sym}^{d\times d})
\end{cases}
\end{aligned}
\end{equation*}
Let's consider linear elasticity problem:
\begin{equation}\label{linearelcase}
(**)\begin{cases}
-div\ \sigma[u]=f(x),\ \text{in}\ \Omega\\
u=g(x)\ \text{on}\ \Gamma_D\\
\sigma[u] \nu = q(x)\ \text{on}\ \Gamma_N
\end{cases}
\end{equation}
\begin{equation*}
f \in L^2(\Omega : \R^d,\ g\in H^1(\Omega : \R^d),\ q\in L^2(\Gamma_N:\R^d))
\end{equation*}
\subsection{Strong Solution}
\begin{equation*}
u\in H^2 (\Omega : \R^d)\ \text{satisfies}\ (**)\ \text{then we call $u:$ a strong solution}
\end{equation*}
\subsection{Weak Solution}
\begin{equation*}
\begin{cases}
\int_\Omega \sigma[u] : e[v] dx = \int_\Omega f\cdot v dx + \int_{\Gamma_N} q\cdot v ds \big( \forall v \in V:= \{ v \in H^1\ (\Omega : \R^d)\ |v|_{\Gamma_D} = 0 \}\big)\\
u \in V+g
\end{cases}
\end{equation*}
\subsection{Properties}
\begin{equation*}
u:\ \text{strong solution}\ \Leftrightarrow \begin{cases}
u:\ \text{weak solution}\\
u\in H^2\ (\Omega : \R^d)
\end{cases}
\end{equation*}
\begin{equation*}
\begin{aligned}
&X:= H^1(\Omega : \R^d)\\
&a(u,v) := \int_\Omega \sigma[u] : e[v] dx\\
&l(v) := \int_\Omega f\cdot v dx + \int_{\Gamma_N} q \cdot v ds\\
\end{aligned}
\begin{aligned}
a(u,v) &= \int_\Omega (\Cmod e[u]) : e[v] dx\\
&= \int_\Omega e[v] : (\Cmod e[u]) dx\\
&= a(v,u)
\end{aligned}
\end{equation*}
For $v \in V$
\begin{equation*}
\begin{aligned}
a(v,v) &= \int_\Omega (\Cmod e[v]) : e[v] dx\\
&\geqq C_* \int_\Omega |e[v]|^2 dx\\
&\geqq C_* ||v||^2_x
\end{aligned}
\end{equation*}
\begin{prop}
	\begin{itemize}
		\item $a(\cdot,\cdot)$ is bounded symmetric, bilinear form on $X \times X$.
		\item $a(\cdot,\cdot)$ is coercive on $V \times V$.
		\item $l$ is bounded linear form on $X$.
	\end{itemize}
\end{prop}
\begin{theorem}
	For any $g\in H^1(\Omega : \R^d)$,
	\begin{equation*}
	\exists! u:a\ \text{weak solution of}\ (**),\ \text{and}\ \begin{cases}
	u=\argmin_{w\in V+g}E(w)
	\end{cases}
	\end{equation*}
\end{theorem}

\subsection{Abstract Theory (cont.)}
$X:$ a Hilbert Space, $V \subset X:$ a closed subspace, $a:X \times X \rightarrow \R$ a bilinear form.
\begin{equation*}
(*)\begin{cases}
\text{Continuous}\\
\text{Symmetric}\\
\text{Coercive}\ a(u,u) \geq\ \text{on}\ V \times V, \alpha ||u||^2_x (\forall u \in V)
\end{cases}
\end{equation*}

\subsection{Lax-Milgram}
If $a$ satisfies $(*),\ \forall l:x \rightarrow \R$ continuous, linear form,
\begin{equation*}
\forall g \in X,\ \exists! u \in V+g
\end{equation*}
s.t. $a(u,v) = l(v)(\forall v \in V)$\\
Moreover, 
\begin{equation*}
u= \argmin_{w\in V+g}
\end{equation*} 
where $E(w):= \frac{1}{2}a(w,w)-l(w)$
\subsubsection{Example 1}
\begin{equation*}
\begin{cases}
|\Gamma_D| > 0\\
X = H^1(\Omega)\\
V={v \in H^1(\Omega)|\ v|_{\Gamma_D}=0}\\
a(u,v) := \int_\Omega \triangledown u \cdot \triangledown v dx\\
l(v) := \int_\Omega fv dx + \int_{\Gamma_N} qv ds
\end{cases}
\end{equation*}
Using poincare inequality,
\begin{equation*}
\exists c>0\ \text{s.t.}\ \int_\Omega |\triangledown v|^2 dx \geqq c \int_\Omega |v|^2 dx (\forall v \in V)
\end{equation*}
\begin{equation*}
\begin{aligned}
\Rightarrow v \in V \Rightarrow a(v,v) &= \frac{1}{2} \int_\Omega |\triangledown v|^2 dx\\
&\geqq \frac{1}{2} \int_\Omega |\triangledown v|^2 dx + \frac{1}{2}c \int_\Omega |v|^2 dx\\
&\geqq \frac{1}{2} \min(1,c) \bigg( \int_\Omega |\triangledown v|^2 dx + \int_\Omega |v|^2 dx \bigg) \\
&= \alpha ||v||^2_{H^1(\Omega)}
\end{aligned}
\end{equation*}
$\therefore a$ is coercive on $V \times V$\\
L-M $\Rightarrow \exists! u:$ a weak solution for:
\begin{equation*}
\begin{cases}
-\triangle u = f\ \text{in}\ \Omega\\
u = g\ \text{on}\ \Gamma_D\\
\frac{\partial u}{\partial \nu} = q\ \text{on}\ \Gamma_N
\end{cases}
\end{equation*}
\begin{equation*}
u= \argmin_{w\in V+g} \bigg( \frac{1}{2} \int_\Omega |\triangledown w|^2 dx -\int_\Omega fw dx - \int_{\Gamma_N} qw ds \bigg)
\end{equation*}

\subsubsection{Example 2}
\begin{equation*}
\begin{cases}
X=H^1(\Omega : \R^d), d=2,3\\
V={v \in H^1(\Omega: \R^d): v|_{\Gamma_D}=0}\\
a(u,v) := \int_\Omega \sigma[u] : e[v] dx\\
l(v) := \int_\Omega f\cdot v dx + \int_{\Gamma_N} q\cdot v ds \begin{cases}
f \in L^2(\Omega:\R^d)\\
q \in L^2(\Gamma_N : \R^d), q \in X
\end{cases}
\end{cases}
\end{equation*}
\end{document}