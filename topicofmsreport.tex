\documentclass[a4paper,12pt]{article}
\usepackage[a4paper, hmargin={1.5cm,1.5cm}, vmargin={2cm,2cm}]{geometry}
\usepackage{amsmath}
\usepackage{amsthm}
\usepackage{amsfonts}
\usepackage{color}
\usepackage[final]{graphicx}
\usepackage{subcaption}
\usepackage{wrapfig}
\newtheorem{remark}{Remark}[]
\newtheorem{prop}{Proposition}
\newtheorem{theorem}{Theorem}
\usepackage{amssymb}

\newcommand{\R}{\mathbb{R}}
\newcommand{\C}{\mathbb{C}}
\newcommand{\N}{\mathbb{N}}
\newcommand{\Q}{\mathbb{Q}}
\newcommand{\W}{\mathbb{W}}
\newcommand{\Vspace}{\mathbb{V}}
\newcommand{\Hspace}{\mathbb{H}}
\newcommand{\Lspace}{\mathbb{L}}
\newcommand{\Lagr}{\mathcal{L}}


% Title Page
\title{Assignment 2\\ Topics of Mathematical Science}
\author{Alifian Mahardhika Maulana}


\begin{document}
\maketitle
\section*{Exercise: Q1}
\begin{enumerate}
	\item Show that:
	\begin{equation*}
	\begin{aligned}[center]
	\frac{d^2}{dt^2} &F(x(t),y(t))\\
	= &F_{xx}(x(t),y(t)) \bigg( \frac{dx}{dt}(t) \bigg)^2 + 2 F_{xy}(x(t),y(t)) \frac{dx}{dt}(t) \frac{dy}{dt}(t)\\
	+ &F_{yy}(x(t),y(t)) \bigg( \frac{dy}{dt}(t) \bigg)^2 + F_{x}(x(t),y(t)) \frac{d^2x}{dt^2}(t) + F_{y}(x(t),y(t)) \frac{d^2y}{dt^2}(t)
	\end{aligned}
	\end{equation*}
	Use fact that $F_{xy}(a,b) = F_{yx}(a,b)$ since $F$ is of class $C^2$ around a point $P(a,b).$\\
	\newline
	\textbf{Answer}:\\
	Take second derivative of $F(x(t),y(t))$ over time, and use chain rule, so that:
	\begin{equation*}
	\begin{aligned}[center]
	\frac{d^2}{dt^2} F(x(t),y(t)) = &\frac{d}{dt} \bigg( \frac{d}{dt} \big( F(x(t),y(t)) \big) \bigg)\\
	= & \frac{d}{dt} \bigg( F_{x}(x(t),y(t)) \frac{dx}{dt}(t) + F_{y}(x(t),y(t)) \frac{dy}{dt}(t) \bigg)\\
	= & F_{xx}(x(t),y(t)) \bigg( \frac{dx}{dt}(t) \bigg)^2 + F_{x}(x(t),y(t)) \frac{d^2x}{dt^2}(t) + F_{xy}(x(t),y(t)) \frac{dx}{dt}(t)\\
	&+ F_{yx}(x(t),y(t)) \frac{dy}{dt}(t) + F_{yy}(x(t),y(t)) \bigg( \frac{dy}{dt}(t) \bigg)^2 + F_{y}(x(t),y(t)) \frac{d^2y}{dt^2}(t)\\
	\text{Use fact that}\ F_{xy} = F_{yx}\\
	\therefore\frac{d^2}{dt^2} F(x(t),y(t)) = & F_{xx}(x(t),y(t)) \bigg( \frac{dx}{dt}(t) \bigg)^2 + F_{x}(x(t),y(t)) \frac{d^2x}{dt^2}(t) + 2 F_{xy}(x(t),y(t)) \frac{dx}{dt}(t) \frac{dy}{dt}(t)\\
	&+ F_{yy}(x(t),y(t)) \bigg( \frac{dy}{dt}(t) \bigg)^2 + F_{y}(x(t),y(t)) \frac{d^2y}{dt^2}(t)\\
	\end{aligned}
	\end{equation*}

	\newpage
	\item Show that:
	\begin{equation*}
	\varphi''(a) = -\frac{F_{xx}(a,b)F_{y}(a,b)^2 -2F_{xy}(a,b)F_{x}(a,b)F_{y}(a,b) + F_{yy}(a,b)F_{x}(a,b)^2}{F_{y}(a,b)^3}
	\end{equation*}
	\newline
	\textbf{Answer}:\\
	Take second derivative of $F(x,\varphi(x))$ over $x$, and use chain rule, so that:
	\begin{equation*}
	\begin{aligned}[center]
	\frac{d^2}{dx^2} F(x,\varphi(x)) = &\frac{d}{dx} \bigg( \frac{d}{dx} F(x,\varphi(x)) \bigg)\\
	= & \frac{d}{dx} \bigg( F_x(x,\varphi(x)) F_y(x,\varphi(x)) + F_y(x,\varphi(x)) \varphi'(x) \bigg)\\
	= & F_{xx}(x,\varphi(x)) F_{y}(x,\varphi(x))^2 + F_{xy}(x,\varphi(x))F_{y}(x,\varphi(x))\\
	&+ F_{yx}(x,\varphi(x))F_{y}(x,\varphi(x)) + F_{yy}(x,\varphi(x)) F_{x}(x,\varphi(x))^2 + \varphi''(x) F_{y}(x,\varphi(x))\\
	\text{Use fact that}\ F_{xy} = F_{yx}\\
	\end{aligned}
	\end{equation*}
	And bring all derivation of $F(x,\varphi(x))$ to the left side, hence:
	\begin{equation*}
	\varphi''(x) = -\frac{F_{xx}(x,\varphi(x)) F_{y}(x,\varphi(x))^2 + 2 F_{xy}(x,\varphi(x))F_{y}(x,\varphi(x)) F_{x}(x,\varphi(x)) + F_{yy}(x,\varphi(x)) F_{x}(x,\varphi(x))^2}{F_{y}(x,\varphi(x))^3}
	\end{equation*}
	take $x=a$,
	\begin{equation*}
	\therefore\varphi''(a) = -\frac{F_{xx}(a,b) F_{y}(a,b)^2 + 2 F_{xy}(a,b)F_{y}(a,b) F_{x}(a,b) + F_{yy}(a,b) F_{x}(a,b)^2}{F_{y}(a,b)^3}
	\end{equation*}
	\newpage
	\item Let
	\begin{equation*}
	\begin{aligned}
	D_1 = &{(x,y) \in \R^2 | 0 \leq x \leq 1,\ 0 \leq y \leq 2}\\
	D_2 = &{(x,y) \in \R^2 | 1 \leq x \leq 2,\ 0 \leq y \leq 1 + x^2}
	\end{aligned}
	\end{equation*}
	Calculate the following integral:
	\begin{enumerate}
		\item \begin{equation*}
		\iint_{D_1} xy^2 dx dy = \int_{0}^{2} \int_{0}^{1} xy^2 dx dy = \int_{0}^{2} \frac{1}{2} x^2 y^2 \Biggr|_0^1 dy = \int_{0}^{2} \frac{1}{2} y^2 dy = \frac{1}{6} y^3 \Biggr|_0^2 = \frac{4}{3} \approx 1.333
		\end{equation*}
		
		\item \begin{equation*}
		\begin{aligned}
		\iint_{D_1} (x+y)^2 dx dy =& \int_{0}^{2} \int_{0}^{1} (x+y)^2 dx dy\\
		=& \int_{0}^{2} \int_{0}^{1} x^2 + 2xy + y^2 dx dy\\
		=& \int_{0}^{2} \frac{1}{3}x^3 + x^2y + y^2 \Biggr|_0^1 dy\\
		=& \int_{0}^{2} \frac{1}{3} + y + y^2 dy\\
		=& \frac{1}{3}y + \frac{1}{2}y^2 + \frac{1}{3}y^3 \Biggr|_0^2 = \frac{16}{3} \approx 5.333
		\end{aligned}
		\end{equation*}
		
		\item \begin{equation*}
		\begin{aligned}
		\iint_{D_2} (x^2+y)^2 dx dy =& \int_{1}^{2} \int_{0}^{1+x^2} (x^2+y)^2 dy dx\\
		=& \int_{1}^{2} \int_{0}^{1+x^2} x^4 + 2x^2y + y^2 dy dx\\
		=& \int_{1}^{2} x^4y + x^2y^2 + \frac{1}{3}y^3 \Biggr|_0^{1+x^2} dx\\
		=& \int_{1}^{2} x^4(1+x^2) + x^2(1+x^2)^2 + \frac{1}{3}(1+x^2)^3 dx\\
		=& \int_{1}^{2} 4x^4 + \frac{7}{3} x^6 + 2x^2 + \frac{1}{3} dx\\
		=& \frac{4}{5}x^5 + \frac{1}{3}x^7 + \frac{2}{3}x^3 + \frac{1}{3}x \Biggr|_1^2 \approx 72.133
		\end{aligned}
		\end{equation*}
	\end{enumerate}

	\newpage
	\item Suppose $D$ is a bounded domain with smooth boundary. Using Green Theorem, show that the line integral:
	\begin{equation}\label{eq:4}
	\int_{\partial D} -ydx + xdy
	\end{equation}
	equal to the area of D.\\
	\newline
	\textbf{Answer:}
	\begin{theorem} Green Theorem\\
		Let $C$ be a positively oriented, piecewise smooth, simple closed curve in a plane, and let $D$ be the region bounded by $C$. If $L$ and $M$ are functions of $(x,y)$ defined on an open region containing $D$ and have continuous partial derivatives there, then:
		\begin{equation*}
		\oint_C (L dx + M dy) = \iint_{D} \bigg( \frac{\partial M}{\partial x} - \frac{\partial L}{\partial y} \bigg) dx dy
		\end{equation*}
	\end{theorem}
	We define area of $D$ as:
	\begin{equation*}
	\iint_{D} dx dy
	\end{equation*}
	using Green Theorem, we can rewrite \eqref{eq:4} becomes:
	\begin{equation*}
	\begin{aligned}
	\int_{\partial D} -ydx + xdy =& \iint_{D} \bigg( \frac{\partial (x)}{\partial x} - \frac{\partial (-y)}{\partial y} \bigg) dx dy\\
	=& \iint_{D} 1 + 1 dx dy\\
	=& 2 \iint_{D} dx dy = 2\times \text{Area of}\ D
	\end{aligned}
	\end{equation*}
\end{enumerate}
\end{document}