\documentclass[a4paper,12pt]{article}
\usepackage[a4paper, hmargin={1.5cm,1.5cm}, vmargin={1.5cm,1.5cm}]{geometry}
\usepackage{amsmath}
\usepackage{amsthm}
\usepackage{amsfonts}
\usepackage{color}
\usepackage[final]{graphicx}
\usepackage{subcaption}
\usepackage{wrapfig}
\newtheorem{remark}{Remark}[]
\newtheorem{prop}{Proposition}
\newtheorem{theorem}{Theorem}
\usepackage{amssymb}

\newcommand{\R}{\mathbb{R}}
\newcommand{\C}{\mathbb{C}}
\newcommand{\N}{\mathbb{N}}
\newcommand{\Q}{\mathbb{Q}}
\newcommand{\W}{\mathbb{W}}
\newcommand{\Vspace}{\mathbb{V}}
\newcommand{\Hspace}{\mathbb{H}}
\newcommand{\Lspace}{\mathbb{L}}
\newcommand{\Lagr}{\mathcal{L}}


% Title Page
\title{Assignment 6\\ Topics of Mathematical Science}
\author{Alifian Mahardhika Maulana}


\begin{document}
\maketitle
\begin{enumerate}
	\item Solve :
	\begin{equation}\label{eq:1}
	z_0 \in \C, \quad \int_{|\varrho-z_0|=R} \frac{d\varrho}{(\varrho-z_0)^n} = \begin{cases}
	2\pi i \quad (n=1)\\
	0 \quad (otherwise)
	\end{cases}
	\end{equation}
	\textbf{Answer:}\\
	\newline
	By Cauchy's integral theorem,\\
	Let $U={\varrho:|\varrho-z_0|<R}$ be a simply connected open subset of $C$ and $f$ a function which is holomorphic on $U$ and continuous on $\overline{U}$. Let $\gamma(t)=e^{it}, t \in [0,2\pi]$  be a loop in $\overline{U}$, then the path integral,
	\begin{equation}\label{eq:2}
	\oint_\gamma \frac{1}{\varrho} dz = \int_{0}^{2\pi} \frac{ie^{it}}{e^{it}} dt = \int_{0}^{2\pi} i dt = 2\pi i
	\end{equation}. We define,
	$$f(\varrho) = \frac{d\varrho}{(\varrho-z_0)^n}, \quad \varrho = z_0 + Re^{it},$$ then we can rewrite \eqref{eq:1} as follows:
	\begin{equation}\label{eq:3}
	\int_{0}^{2 \pi} \frac{1}{(Re^{it})^n} iRe^{it} dt = \frac{i}{R^{n-1}} \int_{0}^{2\pi}e^{i(1-n)t} dt
	\end{equation}
	which according to \eqref{eq:1}, the integral on the right handside equal to $2\pi i$ if $n=1$, therefore,
	\begin{equation*}\label{eq:4}
	\frac{i}{R^{n-1}} \int_{0}^{2\pi}e^{i(1-n)t} dt = 	\frac{i}{R^{1-1}} 2\pi i = -2\pi
	\end{equation*}
	\item Prove the Residue Theorem
	\newline
	Let $D$ be a domain with $f$ is holomorphic around $\partial D$ define by:
	$$f(z) = \frac{a_{(n-1)}}{(z-z_0)^{(n-1)}} + \cdots + \frac{a_1}{(z-z_0)} + H(z)$$
	then in it's interior, $f$ has a pole except at point $z_0$. Choosing $D_i$ small enough, and take the integral around $\partial D$ for $f$, we have:
	$$\int_{\partial D} f(z) dz = \int_{\partial D_i} \frac{a_i}{(z-z_0)^i} + \cdots + \frac{a_1}{(z-z_0)} + H(z) dz$$
	compute the integration term by term, we get,
	$$\int_{\partial D_i} \frac{a_1}{(z-z_0)} = 2\pi i a_1 = 2 \pi i\ res(f,z_0)$$
	if it's interior except at finite number in interior of C, then,
	$$\int_C f(z) dz = 2 \pi i (res(f,z_1) + \cdots + res(f,z_n))$$
\end{enumerate}
\end{document}