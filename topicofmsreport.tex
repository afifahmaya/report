\documentclass[a4paper,12pt]{article}
\usepackage[a4paper, hmargin={1.5cm,1.5cm}, vmargin={1.5cm,1.5cm}]{geometry}
\usepackage{amsmath}
\usepackage{amsthm}
\usepackage{amsfonts}
\usepackage{color}
\usepackage[final]{graphicx}
\usepackage{subcaption}
\usepackage{wrapfig}
\newtheorem{remark}{Remark}[]
\newtheorem{prop}{Proposition}
\newtheorem{theorem}{Theorem}
\usepackage{amssymb}

\newcommand{\R}{\mathbb{R}}
\newcommand{\C}{\mathbb{C}}
\newcommand{\D}{\mathbb{D}}
\newcommand{\N}{\mathbb{N}}
\newcommand{\Q}{\mathbb{Q}}
\newcommand{\W}{\mathbb{W}}
\newcommand{\Vspace}{\mathbb{V}}
\newcommand{\Hspace}{\mathbb{H}}
\newcommand{\Lspace}{\mathbb{L}}
\newcommand{\Lagr}{\mathcal{L}}


% Title Page
\title{Assignment 7\\ Topics of Mathematical Science}
\author{Alifian Mahardhika Maulana}


\begin{document}
\maketitle
		Formula for the residue of $f$ at $z_0$ pole of order $m$ is following,
		\begin{equation}\label{eq:1}
		res(f;z_0) = \lim_{z\to z_0} \frac{1}{(m-1)!} \frac{d^{m-1}}{dz^{m-1}} \left[ (z-z_0)^m f(z)\right]
		\end{equation}
\begin{enumerate}
	\item Find the poles and residues of the following:
	\begin{enumerate}
		\item $\frac{1}{z^2+5z+6}$, poles $z_0=\begin{cases}
		-3, m= 1\\
		-2, m= 1\\
		\end{cases}$, using formula \eqref{eq:1}, we can compute the residues as follow:
		\begin{equation*}
		\begin{aligned}
		res(f;-2) &= \lim_{z\to -2} \frac{1}{(0)!} \left[ (z+2) \frac{1}{(z+2)(z+3)}\right] = 1\\
		res(f;-3) &= \lim_{z\to -3} \frac{1}{(0)!} \left[ (z+3) \frac{1}{(z+2)(z+3)}\right] = -1
		\end{aligned}
		\end{equation*}
		
		\item $\frac{z}{(z^2-1)^2}$, it has poles $z_0=\begin{cases}
		1, m= 2\\
		-1, m= 2\\
		\end{cases}$, using formula \eqref{eq:1}, we can compute the residues as follow:
		\begin{equation*}
		\begin{aligned}
		res(f;1) &= \lim_{z\to 1} \frac{1}{(1)!} \frac{d}{dz} \left[ (z-1)^2 \frac{z}{(z+1)^2(z-1)^2}\right] = 0\\
		res(f;-1) &= \lim_{z\to -1} \frac{1}{(1)!} \frac{d}{dz} \left[ (z+1)^2 \frac{z}{(z+1)^2(z-1)^2}\right] = 0\\
		\end{aligned}
		\end{equation*}
		
		\item $\frac{1}{(z+1)^2 (z+2)}$, it has poles $z_0=\begin{cases}
		1, m= 2\\
		-2, m= 1\\
		\end{cases}$, using formula \eqref{eq:1}, we can compute the residues as follow:
		\begin{equation*}
		\begin{aligned}
		res(f;-1) &= \lim_{z\to -1} \frac{1}{(1)!} \frac{d}{dz} \left[ (z+1)^2 \frac{1}{(z+1)^2(z+2)}\right] = -1\\
		res(f;-2) &= \lim_{z\to -2} \frac{1}{(0)!} \left[ (z+2) \frac{1}{(z+1)^2(z+2)}\right] = 1\\
		\end{aligned}
		\end{equation*}
		
		\item $\frac{z^4+2z+1}{(z-1)^2}$, it has poles $z_0=1,m=2$, using formula \eqref{eq:1}, we can compute the residues as follow:
		\begin{equation*}
		\begin{aligned}
		res(f;1) &= \lim_{z\to 1} \frac{1}{(1)!} \frac{d}{dz} \left[ (z-1)^2 \frac{z^4+2z+1}{(z-1)^2}\right] = 6\\
		\end{aligned}
		\end{equation*}
		
		\item $\frac{z^3+z+1}{(z+1)^2(z+2)}$, it has poles $z_0=\begin{cases}
		-1, m= 2\\
		-2, m= 1\\
		\end{cases}$, using formula \eqref{eq:1}, we can compute the residues as follow:
		\begin{equation*}
		\begin{aligned}
		res(f;-1) &= \lim_{z\to -1} \frac{1}{(1)!} \frac{d}{dz} \left[ (z+1)^2 \frac{z^3+z+1}{(z+1)^2(z+2)}\right] = 5\\
		res(f;-2) &= \lim_{z\to -2} \frac{1}{(0)!} \left[ (z+2) \frac{z^3+z+1}{(z+1)^2(z+2)}\right] = -10\\
		\end{aligned}
		\end{equation*}
		
		\item $\frac{1}{z^m(1-z)^n}$, it has poles $z_0=\begin{cases}
		0, m= m\\
		1, m= n\\
		\end{cases}$, using formula \eqref{eq:1}, we can compute the residues as follow:
		\begin{equation*}
		\begin{aligned}
		res(f;0) &= \lim_{z\to 0} \frac{1}{(m-1)!} \frac{d^{m-1}}{dz^{m-1}} \left[ (z)^m \frac{1}{(z)^m(1-z)^n}\right] = \frac{(n+m-2)!}{(m-1)!(n-1)!}\\
		res(f;1) &= \lim_{z\to 1} \frac{1}{(n-1)!} \frac{d^{n-1}}{dz^{n-1}} \left[ -(1-z)^n \frac{1}{(z)^m(1-z)^n}\right] = - \frac{(m+n-2)!}{(n-1)!(m-1)!}\\
		\end{aligned}
		\end{equation*}
	\end{enumerate}

\textbf{Cauchy Residue Theorem}: the integral of $f(z)$ along $B(z_0,r) = \{z\in \C | |z-z_0|<r\}$ is equal to $2\pi i$ times the sum of the residues of the singularities in the interior of the contour,
\begin{equation}\label{eq:2}
\int_{\partial B(z_0,r)} f(z)dz = 2\pi i \sum\limits_{z_k\in \partial B} res(f;zk)
\end{equation}
\item Calculate the following line integrals with the Residue theorem \eqref{eq:2}
\begin{enumerate}
	\item $$\int_{\partial B(0,1)} \frac{1}{z^2+5z+6}dz$$ it has poles, $z=\{-3,-2\}$, with $B(0,1) = \{z\in \C | |z|<1\}$, because the poles outside of the ball, $$\int_{\partial B(0,1)} \frac{1}{z^2+5z+6}dz = 0$$
	
	\item $$\int_{\partial B(0,2)} \frac{z}{(z^2-1)^2}dz$$ it has poles $z=\begin{cases}
	1, m= 2\\
	-1, m= 2\\
	\end{cases}$, with $B(0,2) = \{z\in \C | |z|<2\}$, because the poles inside of the ball, hence, $$\int_{\partial B(0,2)} \frac{z}{(z^2-1)^2}dz = 2\pi i (res(f;1) + res(f;-1))$$
	with each residue, we computed as follow:
	\begin{equation*}
	\begin{aligned}
	res(f;1) &= \lim_{z\to 1} \frac{1}{(1)!} \frac{d}{dz} \left[ (z-1)^2 \frac{z}{(z-1)^2(z+1)^2}\right] = 0\\
	res(f;-1) &= \lim_{z\to -1} \frac{1}{(1)!} \frac{d}{dz} \left[ (z+1)^2 \frac{z}{(z-1)^2(z+1)^2}\right] = 0\\
	\end{aligned}
	\end{equation*}
	therefore,
	$$\int_{\partial B(0,2)} \frac{z}{(z^2-1)^2}dz = 2\pi i (res(f;1) + res(f;-1)) = 2\pi i (0 + 0) = 0$$
	
	\item $$\int_{\partial B(0,2)} \frac{1}{(z+1)^2(z+3)}dz$$ it has poles $z=\begin{cases}
	-1, m= 2\\
	-3, m= 1\\
	\end{cases}$, with $B(0,2) = \{z\in \C | |z|<2\}$, because the poles inside of the ball is just $z = -1$, hence, $$\int_{\partial B(0,2)} \frac{1}{(z+1)^2(z+3)}dz = 2\pi i (res(f;-1))$$
	with residue, we computed as follow:
	\begin{equation*}
	\begin{aligned}
	res(f;-1) &= \lim_{z\to -1} \frac{1}{(1)!} \frac{d}{dz} \left[ (z+1)^2 \frac{1}{(z+1)^2(z+3)}\right] = -\frac{1}{4}\\
	\end{aligned}
	\end{equation*}
	thus, $$\int_{\partial B(0,2)} \frac{1}{(z+1)^2(z+3)}dz = 2\pi i (res(f;-1)) = \frac{i\pi}{2}$$
\end{enumerate}

Method of residues, \textbf{Application to compute improper integral}:
Let $R(x) = P(x)/Q(x)$ be a rational function of a real variable satisfying the following two criteria: $Q(x) \neq 0$ and $deg (Q) − deg (P) \geq2$
\begin{equation}
\int_{-\infty}^{\infty} R(x) dx = 2 \pi i\sum_{Im(\alpha_k) > 0} res(R(x);\alpha_k)
\end{equation}
\item Evaluate the following integrals by the method of residues.
\begin{enumerate}
	\item $$\int_{0}^{2\pi} \frac{d\theta}{a+\sin\theta}, \quad (a>1)$$
	we change the form into,
	$$\int_{|z|=1} \frac{2}{z^2+i2az-1} dz$$, it has poles $\alpha=\begin{cases}
	i\left( -\frac{a}{2} -\frac{1}{2}\sqrt{a^2-1} \right)\\
	i\left( -\frac{a}{2} +\frac{1}{2}\sqrt{a^2-1} \right)\\
	\end{cases}$,thus the residue is,
	\begin{equation*}
	res(f;\alpha) = \lim_{z\to \alpha} (z-\alpha)f(z) = \lim_{z\to \alpha} \frac{2}{\alpha - \beta} = \frac{1}{i\sqrt{a^2-1}}
	\end{equation*}
	therefore
	\begin{equation*}
	\int_{0}^{2\pi} \frac{d\theta}{a+\sin\theta} = 2\pi i res(f;\alpha) = \frac{2\pi}{\sqrt{a^2-1}}
	\end{equation*}
	
	\item $$\int_{-\infty}^{\infty} \frac{x^2}{x^4+5x^2+6}dx$$, it has poles $\alpha=\begin{cases}
	i\sqrt{2}, m = 1\\
	i\sqrt{3}, m = 1\\
	-i\sqrt{2}, m = 1\\
	-i\sqrt{3}, m = 1\\
	\end{cases}$,thus the residue is,
	\begin{equation*}
	\begin{aligned}
	res(f;i\sqrt{2}) = \lim_{z\to i\sqrt{2}} \left( \frac{(z-i\sqrt{2})z^2}{(z^2+2)(z^2+3)} \right) = -\frac{\sqrt{2}}{i2}\\
	res(f;i\sqrt{3}) = \lim_{z\to i\sqrt{3}} \left( \frac{(z-i\sqrt{3})z^2}{(z^2+2)(z^2+3)} \right) = \frac{\sqrt{3}}{i2}
	\end{aligned}
	\end{equation*}
	therefore,
	\begin{equation*}
	\int_{-\infty}^{\infty} \frac{x^2}{x^4+5x^2+6}dx = 2\pi i (res(f;i\sqrt{2}) + res(f;i\sqrt{3})) = \pi \left( \sqrt{3} - \sqrt{2} \right)
	\end{equation*}
	
	\item $$\int_{-\infty}^{\infty} \frac{x \sin x}{x^2 + a^2} dx, \quad (a\neq 0)$$
	we can rewrite as,
	\begin{equation*}
	Im\left( \int_{-\infty}^{\infty} \frac{xe^{ix}}{x^2+a^2} dx \right)
	\end{equation*}, it has poles $\alpha=\{ia,-ia\}$,thus the residue is,
	\begin{equation*}
	res(f;ia) = \lim_{z\to ia} \left( (z-ia) \frac{ze^{iz}}{(z-ia)(z+ia)} \right) = \frac{e^{-\alpha}}{2}
	\end{equation*}
	Therefore,
	\begin{equation*}
	\int_{-\infty}^{\infty} \frac{x \sin x}{x^2 + a^2} dx = Im\left( \int_{-\infty}^{\infty} \frac{xe^{ix}}{x^2+a^2} dx \right) = Im(i\pi e^{-a}) = \pi e^{-a}
	\end{equation*}
\end{enumerate}

\item Let $\D = \{ z\in \C | |z|<1 \}$, compute
\begin{equation*}
\int_{\partial \D} \frac{|dz|}{|z-a|^2}, \quad (|a|<1)
\end{equation*}
we can rewrite as,
\begin{equation*}
\int_{0}^{2\pi} \frac{d\theta}{1-2a\cos \theta + a^2}
\end{equation*}
on $a>1, |\alpha|<1$, then $\alpha$ is the only pole of $R(z)$ in $\{ |z|<1 \}$, therefore, we get the integral
\begin{equation*}
\int_{0}^{2\pi} \frac{d\theta}{1-2a\cos \theta + a^2} = R(z) = \int_{|z|=1} \frac{1}{i}\frac{-1}{az^2 - (1+a^2)z+a} dz
\end{equation*}, it has poles $\alpha=\{\frac{1}{a}, a\}$, for $|a|<1$, the only pole is on $\{ |z| < 1\}$, thus the residue is,
\begin{equation*}
res(R(z);a) = \lim_{z\to a} (z-a)\frac{-1}{a(z-\frac{1}{a})(z-a)} = \frac{1}{i} \frac{-1}{a(a-\frac{1}{a})} = \frac{-1}{i(a^2-1)}
\end{equation*}
Therefore,
\begin{equation*}
\begin{aligned}
\int_{\partial \D} \frac{|dz|}{|z-a|^2} &= \int_{0}^{2\pi} \frac{d\theta}{1-2a\cos \theta + a^2}\\
&= 2\pi i res(R(z);\alpha) = 2\pi i \left( \frac{-1}{i(a^2-1)} \right)\\
&= \frac{-2\pi}{a^2-1} = \frac{2\pi}{1-a^2}
\end{aligned}
\end{equation*}
\end{enumerate}
\end{document}