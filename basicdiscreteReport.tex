\documentclass[a4paper,12pt]{article}
\usepackage[a4paper, hmargin={2cm,2cm}, vmargin={2cm,2cm}]{geometry}
\usepackage{amsmath}
\usepackage{amsthm}
\usepackage{amsfonts}
\usepackage{color}
\usepackage[final]{graphicx}
\usepackage{subcaption}
\usepackage{wrapfig}
\newtheorem{prob}{Problem}[]
\newtheorem{prop}{Proposition}[]
\theoremstyle{definition}
\newtheorem{definition}{Definition}[]
\usepackage{amssymb}
\usepackage{enumitem}
\usepackage[none]{hyphenat}
\usepackage{algorithm,algpseudocode}

\newcommand{\R}{\mathbb{R}}
\newcommand{\C}{\mathbb{C}}
\newcommand{\N}{\mathbb{N}}
\newcommand{\Q}{\mathbb{Q}}
\newcommand{\Z}{\mathbb{Z}}

\renewcommand{\algorithmicrequire}{\textbf{Input:}}
\renewcommand{\algorithmicensure}{\textbf{Output:}}
\renewcommand\qedsymbol{$\blacksquare$}

\let\oldproofname=\proofname
\renewcommand{\proofname}{\rm\bf{\oldproofname}}

\DeclareMathOperator*{\capmod}{\cap}
\DeclareMathOperator*{\rightarrowmod}{\longrightarrow}

% Title Page
\title{Final Report \\ Basic of Discrete Mathematics}
\author{Alifian Mahardhika Maulana}


\begin{document}
\maketitle
\begin{prob}
	Let $I$ be a $\Q[x,y]$-ideal generated by $f=(1-x-y)^2-4xy$ and $g=y+x^2-1$.\\ Find a Grobener basis of $I$.
\end{prob}
\textbf{Answer:}
To find the Grobener basis of an Ideal, we use Buchberger's Algorithm described in Algorithm \ref{alg:buchberger}. Before we proceed to the algorithm, let me introduce the notion of the least common multiple of a pair of monomials. Consider monomials $x^\alpha$ and $x^\beta$, where as usual this is short for $x^\alpha = x_1^{\alpha_1}x_2^{\alpha_2}\cdots x_n^{\alpha_n}$ and similar for the other monomial. Then, the least common multiple (LCM) of these can be found by taking the maximum of each index,
\begin{equation*}
LCM(x^\alpha,x^\beta) = x_1^{\max(\alpha_1,\beta_1)}x_2^{\max(\alpha_2,\beta_2)}\cdots x_n^{\max(\alpha_n,\beta_n)}
\end{equation*}
Now we define the S-polynomial of a pair of polynomials,
\begin{equation*}
S(f_1,f_2) = \frac{x^\gamma}{LT(f_1)}f_1 - \frac{x^\gamma}{LT(f_2)}f_2
\end{equation*}
where $x^\gamma = LCM(LM(f_1),LM(f_2))$ is the least common multiple of the leading monomials of the polynomials.
\begin{algorithm}[H] % enter the algorithm environment
	\caption{Buchberger's Algorithm} % give the algorithm a caption
	\label{alg:buchberger} % and a label for \ref{} commands later in the document
	\begin{algorithmic} % enter the algorithmic environment
		\Require $F = (f_1,f_2,\cdots,f_s)$, a list of generators for a non-zero ideal.
		\Ensure $G = (g_1,g_2,\cdots,g_t)$, a Grobener basis for the ideal with $F\subset G$
		\State Initialise $G:=F$,
		\Repeat
		\State $G':=G$
		\For{every pair $\{g_i,g_j\}$\quad $i\neq j$}
		\State $x^\gamma \leftarrow LCM(LM(g_i),LM(g_j))$
		\State $S(g_i,g_j) \leftarrow \frac{x^\gamma}{LT(g_i)}g_i - \frac{x^\gamma}{LT(g_j)}g_j$
		\State compute the remainder $S$ on dividing $S(g_i,g_i)$ by $G'$.
		\If{$S\neq 0$}
		\State $G:=G\cup \{S\}$,
		\EndIf
		\EndFor
		\Until{$G=G'$}
		\Return $G$
	\end{algorithmic}
\end{algorithm}
Using lex order $x>y$ and by Algorithm \ref{alg:buchberger}, Grobener basis of the ideal $I=\left< f,g \right>$ is following,
\begin{equation*}
G = \{x^2-2xy-2x+y^2-2y+1, x^2+y-1\}
\end{equation*}

\newpage
\begin{prob}
	Let $I=\left< x^2+y^2+z^2-3xyz, (3y+z-2)(2x-1) \right>$ be a $\Q[x,y,z]$-ideal.\\ Find a Grobener basis of $I$.
\end{prob}
\textbf{Answer:}
By using lex order $x>y>z$ and Algorithm \ref{alg:buchberger}, we can find Grobener basis of the ideal $I=\left< x^2+y^2+z^2-3xyz, (3y+z-2)(2x-1) \right>$ is following,
\begin{equation*}
G = \{x^2-3xyz+y^2+z^2, 6xy+2xz-4x-3y-z+2\}
\end{equation*}

\begin{prob}
	Solve the following system of algebraic equations in the complex domain:
	\begin{equation}\label{eq:prob3}
	x^4-4x^2+5y^2-11=0, \quad x^2y-2y+5=0.
	\end{equation}
\end{prob}
\textbf{Answer:}
By using application of Grobener basis, we want to solve \eqref{eq:prob3} as follows,\\
Let $I$ be an ideal defined by:
\begin{equation}\label{eq:id3}
I = \left< x^4-4x^2+5y^2-11, x^2y-2y+5 \right>
\end{equation}
By using lex order $x>y$ and Algorithm \ref{alg:buchberger}, we can find Grobener basis of \eqref{eq:id3} as follows,
\begin{equation}\label{eq:gb3}
G = \left< x^4-4x^2+5y^2-11, x^2y-2y+5, y^3-x^2-3y+2 \right>
\end{equation}
Thus, we can rewrite \eqref{eq:gb3} as follows:
\begin{equation}\label{eq:4}
\left[ \begin{array}{l}
x^4-4x^2+5y^2-11\\
x^2y-2y+5\\
y^3-x^2-3y+2
\end{array}\right] = \left[ \begin{array}{c}
0\\
0\\
0
\end{array}\right]
\end{equation}
Then, by using Gauss-Elimination method, we solve \eqref{eq:4} in terms of $y$, we get,
\begin{equation}\label{eq:5}
\begin{aligned}
y^4-3y^2+12y+5 = 0\\
(y^2-3y+5)(y^2+3y+1)=0
\end{aligned}
\end{equation}
Then we solve \eqref{eq:5}, we get,
\begin{equation}\label{eq:6}
\begin{array}{cc}
y_1 = 1.5 + 1.658i, & y_2 = 1.5 - 1.658i\\
y_3 = -1.5 + 1.118i, & y_4 = -1.5 - 1.658i
\end{array}
\end{equation}
Substitute each $y:=\{y_1,y_2,y_3,y_4\}$ on \eqref{eq:6} to \eqref{eq:4}. To get,\\
(*)For $y = y_1 = 1.5 + 1.658i$, \eqref{eq:4} becomes:
\begin{equation}\label{eq:7}
\left[ \begin{array}{l}
x^4-4x^2-13.5+24.875i\\
(1.5 + 1.658i)x^2+2-3.317i\\
-x^2 -11.5 + 1.658i
\end{array}\right] = \left[ \begin{array}{c}
0\\
0\\
0
\end{array}\right]
\end{equation}
Then, by using Gauss-Elimination method, we solve \eqref{eq:7} in terms of $x$, we get,
\begin{equation}\label{eq:8}
x^4 -4x^2-31.5+5i = 0
\end{equation}
By solving \eqref{eq:8}, we get one of the solution of \eqref{eq:prob3} as follows,
\begin{equation}
\begin{array}{cc}
x_1 = \pm (2.825 - 0.074i), & x_2 = \pm (0.105 + 1.996i)
\end{array}
\end{equation}
Using the same step as (*) for $y_2,y_3,y_4$ to find $x$ for the solution of \eqref{eq:prob3}, therefore the solution of \eqref{eq:prob3} is as follows,
\begin{equation}
\begin{array}{ccc}
y = 1.5 + 1.658i & x = \pm (2.825 - 0.074i), & x = \pm (0.105 + 1.996i)\\
y = 1.5 - 1.658i & x = \pm (2.957 + 0.567i), & x = \pm (0.751 - 2.233i)\\
y = -1.5 + 1.118i & x = \pm (2.57 + 0.289i), & x = \pm (0.449 - 1.65i)\\
y = -1.5 - 1.658i & x = \pm (2.825 - 0.074i), & x = \pm (0.105 + 1.996i)
\end{array}
\end{equation}

\begin{prob}
	On a polynomial ring $\Q[x,y]$, let $I=\left< x,y^2 \right>$ and $J=\left< x^3,y \right>$. Find the intersection $I\cap J$.
\end{prob}
\textbf{Answer:}
If $I$ and $J$ are two ideals generated respectively by $\{f_1, \cdots, f_m\}$ and $\{g_1, \cdots, g_k\}$, then a grobener basis of $I\cap J$ consists in the polynomials that do not contain t, in the Grobener basis of the ideal,
\begin{equation*}
K = \left< tf_1,\cdots,tf_m, (1-t)g_1,\cdots,(1-t)g_k \right>.
\end{equation*}
In other words, $I\cap J$ is obtained by eliminating $t$ in $K$. Therefore,
\begin{equation}\label{eq:intersect}
I\cap J := G \cap \Q[x,y]
\end{equation}
Hence, for this problem, $K$ is define by,
\begin{equation*}
K = \left< tx,ty^2, (1-t)x^3,(1-t)y \right>
\end{equation*}
By using lex order $x>y>t$ and Algorithm \ref{alg:buchberger}, we can find the grobener basis of $K$-ideal as following,
\begin{equation*}
G = \{ tx^3, txy, tx^3y^2, ty^2 \}
\end{equation*}
Thus, by Equation \eqref{eq:intersect}, we get:
\begin{equation*}
I\cap J = \{x^3, xy, y^2\}
\end{equation*}

\begin{prob}
	Consider the monomials $M_2 = \{ x_1^{a_1} x_2^{a_2} | a_1,a_2 \in \N_0 \}$ of two variables. Prove that, on $M_2$, the graded lexicographic order (grlex) coincides with the graded reverse lexicographic order (grevlex).
\end{prob}
\textbf{Answer:}
\begin{definition}\label{def:grlex}
	\textbf{(Graded Lex Order).} Let $\alpha,\beta \in \Z^n_{\geq 0}$. We say $\alpha >_{grlex} \beta$ if
	\begin{equation*}
	\left|\alpha\right| = \sum_{i=1}^{n} \alpha_i > \left|\beta\right| = \sum_{i=1}^{n} \beta_i, \quad \text{or} \quad \left|\alpha\right| = \left|\beta\right|\ \text{and}\ \alpha >_{lex} \beta
	\end{equation*}
\end{definition}
\begin{definition}\label{def:grevlex}
	\textbf{(Graded Reverse Lex Order).} Let $\alpha,\beta \in \Z^n_{\geq 0}$. We say $\alpha >_{grevlex} \beta$ if
	\begin{equation*}
	\left|\alpha\right| = \sum_{i=1}^{n} \alpha_i > \left|\beta\right| = \sum_{i=1}^{n} \beta_i, \quad \text{or} \quad \left|\alpha\right| = \left|\beta\right|
	\end{equation*}
\end{definition}
By definition \ref{def:grlex} and \ref{def:grevlex}, both $>_{grlex}$ and $>_{grevlex}$ use total degree in the same way. To break a tie, $>_{grlex}$ uses lex order, so that it looks at the leftmost (or largest) variable and favors the larger power. In contrast, when $>_{grevlex}$ finds the same total degree, it looks at the rightmost (or smallest) variable and favors the smaller power.\\
Therefore, in $M_2$ it is clear that $>_{grlex}$ coincides with $>_{grevlex}$ because, if we have a monomial of two variable and arrange by $>_{grlex}$ which looks at the leftmost (or largest) variable and favors larger power $\Rightarrow$ the rightmost (or smaller) variable is favors the smaller power.
\begin{proof}
	Let $P$ be a polynomial defined by:
	\begin{equation*}
	P = x^3y + x^2y^2
	\end{equation*}
	By graded lexicographic order,
	\begin{equation*}
	x^3y >_{grlex} x^2y^2
	\end{equation*}
	since both monomials have total degree 4 and $x^3y >_{lex} x^2y^2$. In this case, we also have
	\begin{equation*}
	x^3y >_{grevlex} x^2y^2
	\end{equation*}
	but for a different reason: $x^3y$ is larger because the smaller variable y appears to a smaller power.
\end{proof}

\begin{prob}
	Let $\leq$ be a monomial order and let $G$ be a finite set of polynomials. Prove that $G$ is a Grobener basis of $I=\left< G \right>$ with respect to $\leq$ if and only if
	\begin{equation}\label{eq:remzero}
	S(f,g) \rightarrowmod_G^* 0
	\end{equation}
	for any $f,g \in G$
\end{prob}
\textbf{Answer:}
\begin{prop}\label{prop:gbasis}
	Let G = $\{g_1, \cdots, g_t\}$ be a Groebner basis for an ideal $I \subset k[x_1,\cdots, x_n]$ and let $f \in k[x_1,\cdots, x_n]$. Then there is a unique $r \in k[x_1,\cdots, x_n]$ with
	the following two properties:
	\begin{enumerate}[label=(\roman*)]
		\item No term of $r$ is divisible by any of $LT(g_1),\cdots, LT(g_t)$.
		\item There is $g \in I$ such that $f = g + r$.
	\end{enumerate}
	In particular, $r$ is the remainder on division of $f$ by $G$ no matter how the elements of $G$
	are listed when using the division algorithm.
\end{prop}
By using Proposition \ref{prop:gbasis} we want to proof \eqref{eq:remzero},
\begin{proof}
	If the remainder is zero, then we have already observed that $f \in I$. Conversely,
	given $f \in I$ then $f = f + 0$ satisfies the two conditions of Proposition \ref{prop:gbasis}. It follows
	that 0 is the remainder of $f$ on division by $G$.
\end{proof}
\end{document}