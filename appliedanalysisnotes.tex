\documentclass[a4paper,12pt]{article}
\usepackage[a4paper, hmargin={3cm,2cm}, vmargin={2cm,2cm}]{geometry}
\usepackage{amsmath}
\usepackage{amsthm}
\usepackage{amsfonts}
\usepackage{color}
\usepackage[final]{graphicx}
\usepackage{subcaption}
\usepackage{wrapfig}
\newtheorem{remark}{Remark}[]
\newtheorem{prop}{Proposition}
\usepackage{amssymb}

\newcommand{\R}{\mathbb{R}}
\newcommand{\C}{\mathbb{C}}
\newcommand{\N}{\mathbb{N}}
\newcommand{\Q}{\mathbb{Q}}
\newcommand{\W}{\mathbb{W}}
\newcommand{\Vspace}{\mathbb{V}}
\newcommand{\Hspace}{\mathbb{H}}
\newcommand{\Lspace}{\mathbb{L}}
\newcommand{\Lagr}{\mathcal{L}}


% Title Page
\title{Applied Analysis Notes}
\author{Alifian Mahardhika Maulana}


\begin{document}
\maketitle
\section{Finite Element Method}
\subsection{Strong Form}
Find $u:\Omega\rightarrow\R$ s.t.
\begin{equation}
\begin{cases}
-\triangle u = f\ in\ \Omega\\
u = g\ on\ \partial\Omega
\end{cases}
\end{equation}
where $f:\Omega \rightarrow \R,g:\partial\Omega \rightarrow \R$ are given functions
\subsection{Weak Form}
Find $u\in V(g)$ s.t.
\begin{equation}\nonumber
a(u,v) = (f,v), \forall v \in V,
\end{equation}
where
\begin{equation}\nonumber
\begin{aligned}
(u,v) &\equiv \int_\Omega u(x)v(x)dx,\\
a(u,v) &\equiv (\triangledown u, \triangledown v) = \int_\Omega \triangledown u \cdot \triangledown v dx,\\
V(g) &\equiv {v \in H^1(\Omega); v = g\ on\ \partial\Omega},\\
V &\equiv V(0)
\end{aligned}
\end{equation}
\subsection{FEM}
\begin{equation}\nonumber
\begin{aligned}
Find u_h \in V_h(g_h) s.t.\\
a(u_h,v_h) = (f,v_h), \forall v_h \in V_h\\
where\\
V_h \subset V, dim V_h < +\infty\\
g_h ~ g, approximation of g\\
u_h ~u, approximation solution of u\\
h : mesh size
\end{aligned}
\end{equation}
\subsection{1-Dimension Case}
Let us define basis functions ${\varphi_i}_i^4,\varphi_i:(0,1)\rightarrow \R$
\begin{equation}\nonumber
\begin{aligned}
\varphi_i : piecewise linear\\
\varphi_i(x_j) = \delta_{ij}=\begin{cases}
1 (i=j)\\
0 (i!=j)
\end{cases}\\
X_h = {<\varphi_0,\cdots,\varphi_4> = {\sum_{i=0}^{4}}c_i\varphi_i(x);c_i \in \R, i=0,\cdots, 4}\\
V_h(g) &= {v_h\in X_h;v_h(x_0) = g_0, v_h(x_1)=g_1}\\
&= {g_0 \varphi_0(x) + \sum_{i=1}^{3}c_i\varphi_i(x) + g_1\varphi_4(x);c_i\in\R, i = 1,2,3}\\
dim X_h = 5, dim V_h(g) = 3\\
V_h = V_h(0)={\sum_{i=1}^{3}c_i\varphi_i(x)c_i\in\R, i = 1,2,3}
\end{aligned}
\end{equation}
Rewrite FEM
\begin{equation}\nonumber
\begin{aligned}
u_h(x) = g_0 \varphi_0(x) + \sum_{i=1}^{3}c_i\varphi_i(x) + g_1\varphi_4(x)
\end{aligned}
\begin{cases}
Find {c_i}_{i=1}^3 \subset \R s.t.\\
a(u_h,v_h), \forall v_h\in V_h
\end{cases}
\begin{cases}
Find {c_i}_{i=1}^3\\
a(u_h,\varphi_i) = (f,\varphi_i), i = 1,2,3,
\end{cases}
proof the equality
\end{equation}
\end{document}