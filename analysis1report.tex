\documentclass[a4paper,12pt]{article}
\usepackage[a4paper, hmargin={2cm,2cm}, vmargin={2cm,2cm}]{geometry}
\usepackage{amsmath}
\usepackage{amsthm}
\usepackage{amsfonts}
\usepackage{color}
\usepackage[final]{graphicx}
\usepackage{subcaption}
\usepackage{wrapfig}
\newtheorem{prob}{Problem}[]
\newtheorem{prop}{Proposition}
\usepackage{amssymb}
\usepackage{enumitem}

\newcommand{\R}{\mathbb{R}}
\newcommand{\C}{\mathbb{C}}
\newcommand{\N}{\mathbb{N}}

\DeclareMathOperator*{\capmod}{\cap}

% Title Page
\title{Analysis Ia Report}
\author{Alifian Mahardhika Maulana}


\begin{document}
\maketitle
\begin{prob}
	Let $M=(S,d)$ be a metric space.\\
	Let $G \subseteq S$. Then, $G$ is open in $M \iff$ it is a union of open balls.
	\begin{proof}
		($\Rightarrow$)
		Let $G$ be open set in $M$.
		Let $x \in G$.\\
		by definition of open set:
		\begin{equation*}
		\exists \delta_x \in \R^n : B_{\delta_x}(x,d) \subseteq G
		\end{equation*}
		where $B_{\delta_x}(x,d)$ is the open $\delta_x$-ball of $x$ in $M$.
		\begin{equation*}
		\therefore G = \capmod_{x\in G} B_{\delta_x}(x,d)
		\end{equation*}
		($\Leftarrow$) Let $G$ be an union of open balls in $M$.\\
		Let the outer of the open balls be the elements of an indexing set $I$.\\
		Then $G$ can be written:
		\begin{equation*}
		G = \capmod_{x\in I} B_{\delta_x}(x,d)
		\end{equation*}
		where $\delta_x \in \R^n$ is the radius of open ball-of $x$.\\
		Let $y\in G$. By definition of union:
		\begin{equation*}
		\exists x \in I : y \in B_{\delta_x}(x,d)
		\end{equation*}
		because an open ball is neighborhood of all points inside, we can say that $B_{\delta_x}(x,d)$ is neighborhood of $y$, by set: $B_{\delta_x}(x,d) \subseteq G$,from theory of superset of neighborhood in Metric Space, it follows that $G$ is a neighborhood of $y$.\\
		Since $y$ is arbitrary, it follows that $G$ is a neighborhood of its point. Hence, by definition:
		\begin{equation*}
		\therefore G\ \text{is open in}\ M
		\end{equation*}
	\end{proof}
\end{prob}
\newpage
\begin{prob}
	Let $C([0,1])$ be the set of all continuous functions $f:[0,1]\rightarrow \R$, for $f,g \in C([0,1])$. Show that $(C([0,1]),d_1)$ is not complete.\\
	\begin{proof}
		Suppose that:
		\begin{equation*}
		d_1 (f,g) := \int_{0}^{1} |f(x)-g(x)| dx,\ f,g \in C[0,1]
		\end{equation*}
		Let's consider a sequence $\{f_n\}_{n\geq 3}:$
		\begin{equation*}
		f_n(x) = \begin{cases}
		\begin{aligned}
		&0,\ &0 &\leq x < \frac{1}{2}-\frac{1}{n},\\
		&n\Bigg(x+\frac{1}{n}-\frac{1}{2}\Bigg), \ &\frac{1}{2}-\frac{1}{n} &\leq x < \frac{1}{2},\\
		&1,\ &\frac{1}{2} &\leq x \leq 1
		\end{aligned}
		\end{cases}
		\end{equation*}
		It shows that the sequence $(f_n)$ converges to discontinuous function $f(x):= 0$ for $0 \leq x < \frac{1}{2}$ and $f(x):=1$ for $\frac{1}{2}\leq x \leq 1$. Hence, $f \notin C[0,1];$
		\begin{equation*}
		\therefore\ \text{there is no}\ g \in C[0,1]\ \text{s.t.}\ d_1(f_n,g) \rightarrow 0
		\end{equation*}
	\end{proof}
\end{prob}
\end{document}