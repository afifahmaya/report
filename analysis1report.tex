\documentclass[a4paper,12pt]{article}
\usepackage[a4paper, hmargin={2cm,2cm}, vmargin={2cm,2cm}]{geometry}
\usepackage{amsmath}
\usepackage{amsthm}
\usepackage{amsfonts}
\usepackage{color}
\usepackage[final]{graphicx}
\usepackage{subcaption}
\usepackage{wrapfig}
\newtheorem{prob}{Problem}[]
\newtheorem{prop}{Proposition}
\usepackage{amssymb}
\usepackage{enumitem}

\newcommand{\R}{\mathbb{R}}
\newcommand{\C}{\mathbb{C}}
\newcommand{\N}{\mathbb{N}}

% Title Page
\title{Analysis Ia Report}
\author{Alifian Mahardhika Maulana}


\begin{document}
\maketitle
\begin{prob}
	Suppose:
	\begin{equation}\label{eq:metric1}
	d_2(f,g) := \bigg( \int_{a}^{b}\{f(x)-g(x)\}^2dx \bigg)^{\frac{1}{2}},f:[a,b]\rightarrow \R
	\end{equation}
	we will show $(C([a,b]),d_2)$ is a \textbf{metric space}.
	\begin{enumerate}[label=(\alph*)]
		\item \textbf{Positivity}\\
		Take: $h=\int_{a}^{b}\{f(x)-g(x)\}^2dx$, then equation \eqref{eq:metric1} becomes:
		\begin{equation*}
		d_2(f,g) = \sqrt{h}
		\end{equation*}
		Since the value of $(f(x)-g(x))^2\geq 0$ then $\sqrt{h}\geq 0,\ \forall h\in C[a,b]$, then:
		\begin{equation*}
		d_2(f,g) \geq 0
		\end{equation*}
		
		\item \textbf{Definiteness}\\
		$(\Leftarrow)$ put $f(x) = g(x)$, then equation \eqref{eq:metric1} becomes:
		\begin{equation*}
		\begin{aligned}
		d_2(f,g) &= \sqrt{\int_{a}^{b}\{f(x)-g(x)\}^2dx}\\
		&= \sqrt{\int_{a}^{b}\{g(x)-g(x)\}^2dx}\\
		d_2(f,g) &= 0
		\end{aligned}
		\end{equation*}
		
		$(\Rightarrow)$ put $d_2(f,g) = 0$, then equation \eqref{eq:metric1} becomes:
		\begin{equation*}
		\begin{aligned}
		0 &= \sqrt{\int_{a}^{b}\{f(x)-g(x)\}^2dx}\\
		0^2 &=\Bigg( \sqrt{\int_{a}^{b}\{f(x)-g(x)\}^2dx} \Bigg)^2\\
		0 &= \int_{a}^{b}\{f(x)-g(x)\}^2dx
		\end{aligned}
		\end{equation*}
		to satisfies $(\Rightarrow),\ \forall a,b \in \R$
		\begin{equation*}
		\begin{aligned}
		f(x) - g(x) &= 0\\
		\therefore f(x) &= g(x)
		\end{aligned}
		\end{equation*}
		
		\item \textbf{Symmetry}
		\begin{equation*}
		\begin{aligned}
		d_2(f,g) &= \sqrt{\int_{a}^{b}\{f(x)-g(x)\}^2dx}\\
		d_2(f,g) &= \sqrt{\int_{a}^{b}\bigg( f(x)^2 -2f(x)g(x) + g(x)^2 \bigg)dx}\\
		d_2(f,g) &= \sqrt{\int_{a}^{b}\bigg( g(x)^2 -2g(x)f(x) + f(x)^2 \bigg)dx}\\
		d_2(f,g) &= \sqrt{\int_{a}^{b}\{g(x)-f(x)\}^2dx}\\
		d_2(f,g) &= d_2(g,f),\ \forall f,g \in C
		\end{aligned}
		\end{equation*}
		
		\item \textbf{Triangle Inequality}\\
		Put $g(x)\leq h(x)\leq f(x)$
		\begin{equation*}
		\begin{aligned}
		d_2(f,g) &= \sqrt{\int_{a}^{b}\{f(x)-h(x)+h(x)-g(x)\}^2dx}\\
		\end{aligned}
		\end{equation*}
		then we take $r(x)=f(x)-h(x)$,and $s(x)=h(x)-g(x)$\\
		\begin{equation*}
		\begin{aligned}
		d_2(f,g) &= \sqrt{\int_{a}^{b}r(x)^2 +2r(x)s(x) + s(x)^2dx}\\
		&= \sqrt{\int_{a}^{b}r(x)^2 dx +2\int_{a}^{b}r(x)s(x) dx + \int_{a}^{b}s(x)^2dx}\\
		\end{aligned}
		\end{equation*}
		according to Cauchy-Schwartz inequality:
		\begin{equation*}
		\int_{a}^{b}r(x)s(x) dx \leq \sqrt{\int_{a}^{b}r(x)^2 dx} \sqrt{\int_{a}^{b}s(x)^2 dx}
		\end{equation*}
		thus:
		\begin{equation*}
		\begin{aligned}
		d_2(f,g) &\leq \sqrt{\int_{a}^{b}r(x)^2 dx +2\sqrt{\int_{a}^{b}r(x)^2 dx} \sqrt{\int_{a}^{b}s(x)^2 dx} + \int_{a}^{b}s(x)^2dx}\\
		&\leq \sqrt{\Bigg( \sqrt{\int_{a}^{b}r(x)^2 dx} + \sqrt{\int_{a}^{b}s(x)^2dx} \Bigg)^2}\\
		&\leq \sqrt{\int_{a}^{b}r(x)^2 dx} + \sqrt{\int_{a}^{b}s(x)^2dx}\\
		d_2(f,g) &\leq \sqrt{\int_{a}^{b}\{ f(x)-h(x) \}^2 dx} + \sqrt{\int_{a}^{b}\{ h(x)-g(x) \}^2dx}\\
		\therefore d_2(f,g) &\leq d_2(f,h) + d_2(h,g)
		\end{aligned}
		\end{equation*}
	\end{enumerate}
\end{prob}
\newpage
\begin{prob}
	Let $(X,d)$ be a metric space, suppose:
	\begin{equation}\label{eq:metric2}
	\tilde{d}(x,y) := \frac{d(x,y)}{1+d(x,y)}\ (x,y\in X)
	\end{equation}
	we will show that $\tilde{d}(x,y)$ is also \textbf{metric} on $X$
	
	\begin{enumerate}[label=(\alph*)]
		\item \textbf{Positivity}\\
		Since $d(x,y)$ is a metric, it satisfies $d(x,y) \geq 0$, thus:
		\begin{equation*}
		\tilde{d}(x,y) = \frac{d(x,y)}{1+d(x,y)} \geq 0
		\end{equation*}
		\item \textbf{Definiteness}\\
		$(\Leftarrow)$ put $x=y$\\
		Since $d(x,y)$ is a metric, it satisfies $x=y \Rightarrow d(x,y)=0$, thus:
		\begin{equation*}
		\tilde{d}(x,y) = \frac{d(x,y)}{1+d(x,y)} = \frac{0}{1+0} = 0
		\end{equation*}
		\begin{equation*}
		\therefore \tilde{d}(x,y) = 0 \Leftarrow x = y
		\end{equation*}
		$(\Rightarrow)$ put $\tilde{d}(x,y) = 0$\\
		\begin{equation*}
		\begin{aligned}
		\tilde{d}(x,y) &= \frac{d(x,y)}{1+d(x,y)}\\
		0 &= \frac{d(x,y)}{1+d(x,y)}\\
		\end{aligned}
		\end{equation*}
		to satisfies the equation, $d(x,y) = 0$\\
		and since $d(x,y)$ is a metric, it satisfies $d(x,y) = 0 \Rightarrow x=y$
		then $x$ should be equal to $y$\\
		\begin{equation*}
		\therefore \tilde{d}(x,y) = 0 \Rightarrow x=y
		\end{equation*}
		\item \textbf{Symmetry}\\
		Since $d(x,y)$ is a metric, it satisfies $d(x,y) = d(y,x)$,\\
		then we can rewrite equation \eqref{eq:metric2} as:
		\begin{equation*}
		\begin{aligned}[center]
		\tilde{d}(x,y) &= \frac{d(x,y)}{1+d(x,y)}\\
		&= \frac{d(y,x)}{1+d(y,x)} = \tilde{d}(y,x)\\
		\therefore \tilde{d}(x,y) &= \tilde{d}(y,x)
		\end{aligned}
		\end{equation*}
		
		\item \textbf{Triangle Inequality}\\
		Since $d(x,y)$ is a metric, it satisfies $d(x,y) \leq d(x,z) + d(z,y),\ \forall x,y,z \in X$,\\
		then we can rewrite equation \eqref{eq:metric2} as:
		\begin{equation*}
		\begin{aligned}[center]
		\tilde{d}(x,y) = \frac{d(x,y)}{1+d(x,y)} &\leq \frac{d(x,z) + d(z,y)}{1+d(x,z) + d(z,y)}\\
		&\leq \frac{d(x,z)}{1+d(x,z) + d(z,y)} + \frac{d(z,y)}{1+d(x,z) + d(z,y)}\\
		&\leq \tilde{d}(x,z) + \tilde{d}(z,y)\\
		\therefore \tilde{d}(x,y) &\leq \tilde{d}(x,z) + \tilde{d}(z,y)
		\end{aligned}
		\end{equation*}
	\end{enumerate}
\end{prob}
\end{document}