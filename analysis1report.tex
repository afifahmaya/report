\documentclass[a4paper,12pt]{article}
\usepackage[a4paper, hmargin={3cm,2cm}, vmargin={2cm,2cm}]{geometry}
\usepackage{amsmath}
\usepackage{amsthm}
\usepackage{amsfonts}
\usepackage{color}
\usepackage[final]{graphicx}
\usepackage{subcaption}
\usepackage{wrapfig}
\newtheorem{prob}{Problem}[]
\newtheorem{prop}{Proposition}
\usepackage{amssymb}

\newcommand{\R}{\mathbb{R}}
\newcommand{\C}{\mathbb{C}}
\newcommand{\N}{\mathbb{N}}

% Title Page
\title{Analysis Ia Report}
\author{Alifian Mahardhika Maulana}


\begin{document}
\maketitle
\begin{prob}
	Suppose that for every $n\in \N$ we have:
	\begin{equation*}
	b_n \leq a_n \leq c_n
	\end{equation*}
	Let \begin{equation*}
	\lim_{n\to\infty}b_n = l =\lim_{n\to\infty}c_n
	\end{equation*}
	given $\epsilon > 0$, then it follows from the convergence of $b_n\ \text{and}\ c_n\ \text{to}\ l$ that there exists a natural number $N$ such that if $n\geq N$ then:
	\begin{equation*}
	\begin{aligned}
	|b_n-l|<\epsilon\\
	-\epsilon < b_n-l < \epsilon
	\end{aligned}
	\ \text{and}\
	\begin{aligned}
	|c_n-l|<\epsilon\\
	-\epsilon < c_n-l < \epsilon
	\end{aligned}	
	\end{equation*}
	Since the hypothesis implies that
	\begin{equation*}
	\begin{aligned}
	b_n-l \leq &a_n-l \leq c_n-l\\
	-\epsilon<b_n-l \leq &a_n-l \leq c_n-l < \epsilon
	\end{aligned}
	\end{equation*}
	it follows that
	\begin{equation*}
	-\epsilon < a_n-l < \epsilon
	\end{equation*}
	for all $n\geq K.$ Since $\epsilon > 0$ is arbitrary, this implies that
	\begin{equation*}
	\lim_{n\to\infty} a_n =l
	\end{equation*}
\end{prob}
\begin{prob}
	Prove if a sequence of real numbers converges, then it is bounded and it is a Cauchy sequence.
	\begin{enumerate}
		\item If a sequence of real numbers converges, then it is bounded.
		\begin{proof}
			Suppose $x_n$ be a sequence converges to $x$ and let $\epsilon:=1$. Then there exist a natural number $K=K(1)$ such that $|x_n-x|<1\ \text{for all}\ n\geq K$. Then if we apply Triangle Inequality with $n\geq K$ we obtain
			\begin{equation*}
			|x_n|=|x_n-x+x|\leq|x_n-x|+|x|<1+|x|
			\end{equation*}
			put
			\begin{equation*}
			M:=\sup\{|x_1|,|x_2|,\cdots,|x_{K-1}|,1+|x|\},
			\end{equation*}
			then it follows that $|x_n|\leq M$ for all $n\in \N$.
		\end{proof}
	\newpage
		\item If a sequence of real numbers converges, then it is a Cauchy sequence.
		\begin{proof}
			Suppose $x_n$ be a sequence converges to $x$ and let $\epsilon:=\frac{\epsilon}{2}$, then there exist a natural number $K=K(1)$ such that $|x_n-x|<\frac{\epsilon}{2}\ \text{for all}\ n\geq K$.\\
			Let $x_m$ be a sequence converges to $x$ and let $\epsilon:=\frac{\epsilon}{2}$, then there exist a natural number $K=K(1)$ such that $|x_m-x|<\frac{\epsilon}{2}\ \text{for all}\ m\geq K$.\\
			Applying Triangular Inequality to substraction of $x_n\ \text{and}\ x_m$, we obtain
			\begin{equation*}
			\begin{aligned}
			|x_n - x_m|&=|x_n-x+x-x_m|\leq|x_n-x|+|x_m-x| < \frac{\epsilon}{2}+\frac{\epsilon}{2}\\
			&=|x_n-x+x-x_m|\leq|x_n-x|+|x_m-x| < \epsilon
			\end{aligned}
			\end{equation*}
			then it follows that $|x_n - x_m|<\epsilon$.
		\end{proof}
	\end{enumerate}
\end{prob}
\end{document}