\documentclass[a4paper,12pt]{article}
\usepackage[a4paper, hmargin={2cm,2cm}, vmargin={2cm,2cm}]{geometry}
\usepackage{amsmath}
\usepackage{amsthm}
\usepackage{amsfonts}
\usepackage{color}
\usepackage[final]{graphicx}
\usepackage{subcaption}
\usepackage{wrapfig}
\newtheorem{prob}{Problem}[]
\newtheorem{prop}{Proposition}
\theoremstyle{definition}
\newtheorem{definition}{Definition}[]
\usepackage{amssymb}
\usepackage{enumitem}
\usepackage[none]{hyphenat}

\newcommand{\R}{\mathbb{R}}
\newcommand{\C}{\mathbb{C}}
\newcommand{\N}{\mathbb{N}}

\DeclareMathOperator*{\capmod}{\cap}

% Title Page
\title{Assignment 5 \\ Analysis I Report}
\author{Alifian Mahardhika Maulana}


\begin{document}
\maketitle
\begin{prob}
	Prove that every sequentially compact subset of a metric space is bounded and closed.
\end{prob}
	\textbf{Answer:}
	Let $X$ be a metric space and $K \subset X$ be compact. If $x_n$ is a convergent sequence in $K$ with limit $x \in X$, then every subsequence of $x_n$ converges to $x$. Since $K$ is compact, some subsequence of $x_n$ converges to a limit in $K$, so $x \in K$ and $K$ is closed.\\
	\newline
	We use contradiction to prove bounded. Assume $K$ is closed and bounded. Take sequence $X_n \subset \R^n$ where $X_n = (x_1^n, \cdots, x_N^n)$. Since $X_n$ is bounded, each of the sequences $(x_j^n),i\leq j \leq N$, is bounded. Since every bounded sequence in $\R$ has a converging subsequence in $K$, then seting $X(x_1,\cdots,x_N)$, we have that $x_{n_k} \to x \in \R^N$. Since $K$ is closed, $X \in K$ and $K$ is compact.
	
\begin{prob}
	Let $X$ be a Banach space and let $f,g$ be linear operators on $X$, $f:X \to X$ is compact and $g:X \to X$ is continuous. Prove that the composition maps $g \circ f$ and $f \circ g$ are compact.
\end{prob}
	\textbf{Answer:}
	\begin{prop}\label{prop:1}
		Let $(X,d_x),(Y,d_y): \text{metric space}\quad f:X \to Y$ continuous, then $M\subset X:\ \text{compact} \Rightarrow f(M)$ is compact in $Y$
	\end{prop}
	\begin{definition}\label{def:1}
		$X,Y:$ Banach space, $f:M\subset X \to Y$. $f$ is called a compact mapping if $f$ is continuous and $\forall B \subset M$ bounded, $f(B)$ is relatively compact
	\end{definition}
	\begin{enumerate}
		\item \begin{proof}[$f\circ g$ is compact]
			Since $f$ is compact, by definition \eqref{def:1}, $f(x)$ is continuous and by proposition \eqref{prop:1}, a continuous function map compact sets into compact sets, therefore $f \circ g$ is compact.
		\end{proof}
		\item  \begin{proof}[$g \circ f$ is compact]
			Since $g$ is continuous, by proposition \eqref{prop:1}, a continuous function map compact sets into compact sets, moreover $f$ is compact, hence subset of $f$ is also compact, therefore $g \circ f$ is compact.
		\end{proof}
	\end{enumerate}
\newpage
\begin{prob}
	Let $X = C[0,1]$ and $||u||:= \max_{0\leq x \leq 1} |u(x)|$. For given $\alpha \in \R$ and $f \in X$, consider the nonlinear integral equation:
	$$u(x) = \alpha \int_{0}^{1} \sin u(x) dx + f(x) \quad (*)$$
	\begin{enumerate}
		\item Show that if $|\alpha| < 1$, $(*)$ has a unique solution $u \in X$. (Hint: Contraction mapping principle)
		\item (extra) Consider the case $|\alpha| \geq 1$. Does $(*)$ has a solution $u\in X$?
	\end{enumerate}
\end{prob}
	\textbf{Answer:}
	\begin{enumerate}
		\item 	Let $g(u(x)) = \alpha \int_{0}^{1} \sin u(x) dx + f(x),\ \text{and}\ g(v(x)) = \alpha \int_{0}^{1} \sin v(x) dx + f(x)$, taking distance of $g(u(x))$ and $g(v(x))$, we get
		\begin{equation}\label{eq:1}
		\begin{aligned}
		d(g(u(x)) - g(v(x))) &= \left| \alpha \int_{0}^{1} \sin u(x) dx + f(x) - \alpha \int_{0}^{1} \sin v(x) dx - f(x) \right|\\
		&= \left| \alpha \int_{0}^{1} (\sin u(x) - \sin v(x)) dx \right|\\
		\end{aligned}
		\end{equation}
		by triangle inequality,
		\begin{equation}\label{eq:2}
		d(g(u(x)) - g(v(x))) \leq \left| \alpha \right| \left| \int_{0}^{1} (\sin u(x) - \sin v(x)) dx \right|
		\end{equation}
		Since $X=C[0,1]$ and $\left|| u \right|| := \max_{0\leq x \leq 1} \left| u(x) \right|$, we can rewrite \eqref{eq:2} as follows:
		\begin{equation}\label{eq:3}
		\begin{aligned}
		d(g(u(x)) - g(v(x))) &\leq \left| \alpha \right| \max_{0\leq x \leq 1}\left| \int_{0}^{1} (\sin u(x) - \sin v(x)) dx \right|\\
		&\leq \left| \alpha \right| \left|| u(x) - v(x) \right||
		\end{aligned}
		\end{equation}
		Therefore, (*) satisfied contraction mapping principle, so that $u(x)$ has unique solution $u \in X$ if $\left| \alpha \right| < 1$.
		
		\item Assume $u(x)$ and $v(x)$ are fixed point, then it should satisfies
		\begin{equation}\label{eq:4}
		0 \leq d(u(x) - v(x)) = d(g(u(x)) - g(v(x))) < \alpha d(g(u(x)) - g(v(x))), \quad \alpha \in [0,1]
		\end{equation}
		if we choose $\left| \alpha \right| \geq 1$, then $u(x)$ doesn't satisfied \eqref{eq:4}, therefore $u(x)$ doesn't have an unique solution $u\in X$.
	\end{enumerate}
\end{document}