\documentclass[a4paper,12pt]{article}
\usepackage[a4paper, hmargin={2cm,2cm}, vmargin={2cm,2cm}]{geometry}
\usepackage{amsmath}
\usepackage{amsthm}
\usepackage{amsfonts}
\usepackage{color}
\usepackage[final]{graphicx}
\usepackage{subcaption}
\usepackage{wrapfig}
\newtheorem{prob}{Problem}[]
\newtheorem{prop}{Proposition}
\usepackage{amssymb}
\usepackage{enumitem}

\newcommand{\R}{\mathbb{R}}
\newcommand{\C}{\mathbb{C}}
\newcommand{\N}{\mathbb{N}}

\DeclareMathOperator*{\capmod}{\cap}

% Title Page
\title{Assignment 3 \\ Analysis Ia Report}
\author{Alifian Mahardhika Maulana}


\begin{document}
\maketitle
\begin{enumerate}
	\item Let $(X,d)$ be a complete metric space and let $f:X \to X$ be a map. Suppose the iterated map
	\begin{equation}\label{eq:1}
	f^k = f \circ \cdots \circ f \quad \text{(k times)}
	\end{equation}
	is a contraction for some $k\geq 2$. Prove that $f$ has a unique fixed point $x \in X$.\\
	\newline
	\textbf{Answer:}\\
	By the contraction mapping theorem, $f^k$ has a unique fixed point, let's call it $x$, so that
	\begin{equation}\label{eq:2}
	f^k(x) = x
	\end{equation}
	by \eqref{eq:2} and \eqref{eq:1} we note that $$f^k(f(x)) = f(f^k(x)) = f(x)$$
	Therefore, $f(x)$ and $x$ are both fixed points of $f^k$. Since $f^k$ has a unique fixed point, $f(x) = x$. Now, we show that for any $x_0 \in X$ the points $f^k(x_0)$ converges to $x$ as $k\to \infty$. Let's consider $f^k(x_0)$ as $k$ runs through some iteration until $N$. i.e. pick $0 \leq i \leq N-1$ look at the points $f^{kN+i}(x_0)$ as $k\to\infty$. Since
	$$f^{kN+i}(x_0) = f^{kN}(f^i(x_0)) = (f^k)^N(f^i(x_0))$$
	and $f^k$ is a contraction, it must be tend to $x$ by the contraction mapping theorem. So all k sequences $\{f^{kN+i}(x_0)\}_{k\geq 1}$ tend to $x$.\\
	$\therefore f$ has a unique fixed point $x \in X$.
	\newpage
	\item Complete the proof of Picard-Lindelof Theorem by showing that $A:M \to M$ is a contraction if $h$ is small\\
	\newline
	\textbf{Answer:}\\
	We define:
	\begin{equation}\label{eq:fa}
	\text{for}\ x\in M,\quad A(x,t) = x_0 + \int_{t_0}^{t} f(s,x(t)) ds
	\end{equation}
	and also consider Lipschitz-continuous
	\begin{equation}\label{eq:lipschitz}
	d(f(x,t) - f(y,t)) \leq Ld(x-y)\bigr|\forall(x,t),(y,t)\in S \quad \text{for some}\ L > 0
	\end{equation}
	Let's consider two function $x,y \in M$, we want to show
	\begin{equation}\label{eq:contr}
	d(A(x), A(y)) \leq K d(x,y)\quad \text{for some}\ K \in (0,1).
	\end{equation}
	So let $t$ be such that
	\begin{equation*}
	d(A(x), A(y)) = d(A(x,t), A(y,t))
	\end{equation*}
	then using definition of $A$
	\begin{equation}\label{eq:proof}
	\begin{aligned}
		d(A(x,t), A(y,t)) &= d\bigg(\int_{t_0}^{t}(f(s,x(t)) - f(s,y(t))) ds \bigg)\\
		&= \int_{t_0}^{t} d(f(s,x(t)) - f(s,y(t))) ds\\
		&\leq L \int_{t_0}^{t} d(x-y) ds,\quad f\ \text{is Lipschitz-continuous}\\
		&\leq Lhd(x-y)\\
	\end{aligned}
	\end{equation}
	based on \eqref{eq:lipschitz} we know that $L>0$, therefore \eqref{eq:proof} is a contraction if $h<\frac{1}{L}$($h$ is small enough).
\end{enumerate}
\end{document}