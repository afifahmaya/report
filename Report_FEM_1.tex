\documentclass[]{report}
\usepackage{amsmath}
\usepackage{amsfonts}
\usepackage{amssymb}

\newcommand{\R}{\mathbb{R}}
\newtheorem{remark}{Remark}[section]
% Title Page
\title{Basic of Finite Element Method}
\author{Afifah Maya Iknaningrum}

\begin{document}
%\maketitle

\section{Problem}
Consider Poisson Equation problem as shown below. We want to find $ u $ such that
\begin{equation}\label{Poisson}
\begin{cases}
-\Delta u = f(x) & \text{ in } \Omega \\
u = g(x) & \text{ on } \Gamma = \partial \Omega.
\end{cases}
\end{equation}

\section{Continuous (Partial Differential Equation)}
We need to know some notation beforehand.
\begin{eqnarray}\label{notation} \nonumber
X &:=& H^{1}(\Omega) \\ \nonumber
V &:=& H^{1}_{0}(\Omega) \subset X \\ \nonumber
H^{1}(\Omega) &\equiv& \{ v \in L^{2}(\Omega) ; \dfrac{\partial v}{\partial x} \in L^{2}(\Omega) \} \\ \nonumber
L^{2}(\Omega) &\equiv& \{ v: \Omega \rightarrow \mathbb{R} ;  \int_{\Omega} v^{2}(x) dx < \infty \} \\ \nonumber
V(g) &:=& \{ v \in X ; v=g \text{ on } \Gamma \text{ or } v-g \in V \} \\ \nonumber
V &=& V(0).
\end{eqnarray}

From the strong form in equation (\ref{Poisson}), we can obtain the weak form
$ \forall $ test function $ v(x) $, where $ v|_{\Gamma_0} = 0 $, then,
\begin{align*}
	&\int_\Omega (-\Delta u)(x) v(x) dx\\
	&= \int_\Omega \Big(-\dfrac{\partial^2u}{\partial{x_1}^2} (x) v(x) -\dfrac{\partial^2u}{\partial{x_2}^2} (x) v(x) \Big) \ dx \\
	&= - \int_{\Omega} \dfrac{\partial^{2}u}{\partial x_{1}^{2}}  (x) v(x) \ dx - \int_{\Omega} \dfrac{\partial^{2} u}{\partial x_{2}^{2}} (x) v(x) \ dx\\
	&= - \Big( \int_{\partial \Omega} \dfrac{\partial u}{\partial x_{1}} (x) v(x) n_{i} \ ds - \int_{\Omega} \dfrac{\partial u}{\partial x_{1}} (x) \dfrac{\partial v}{\partial x_{1}} (x) \ dx \Big) - \Big( \int_{\partial \Omega} \dfrac{\partial u}{\partial x_{2}} (x) v(x) n_{i} \ ds - \int_{\Omega} \dfrac{\partial u}{\partial x_{2}} (x) \dfrac{\partial v}{\partial x_{2}} (x) \ dx \Big)\\
	&= \Big( \int_{\Gamma_0} \dfrac{\partial u}{\partial x_{1}} (x) v(x) n_{i} + \dfrac{\partial u}{\partial x_{2}} (x) v(x) n_{i} \ ds + \int_{\Gamma_1} \dfrac{\partial u}{\partial x_{1}} (x) v(x) n_{i} + \dfrac{\partial u}{\partial x_{2}} (x) v(x) n_{i} \ ds \Big) \\
	&+ \int_{\Omega} \dfrac{\partial u}{\partial x_{1}} (x) \dfrac{\partial v}{\partial x_{1}} (x) + \dfrac{\partial u}{\partial x_{2}} (x) \dfrac{\partial v}{\partial x_{2}} (x) \ dx \\
	&= \int_{\Omega} \nabla u(x) \cdot \nabla v(x) \ dx - \int_{\Gamma_1} (\nabla u(x) \cdot n(x)) v(x) \ ds \\
	&= \int_{\Omega} \nabla u(x) \cdot \nabla v(x) \ dx - \int_{\Gamma_1} \dfrac{\partial u}{\partial n} (x) v(x) \ ds \\
	&= \int_{\Omega} \nabla u(x) \cdot \nabla v(x) \ dx - \int_{\Gamma_1} g(x) v(x) \ ds\\
	&= \int_{\Omega} f(x) v(x) \ dx
\end{align*}
such that $ \int_{\Omega} \nabla u \cdot \nabla v \ dx = \int_{\Omega} f v \ dx + \int_{\Gamma_1} g v \ ds $. For simplicity, we assume $ \Omega = (0.1) $

\begin{equation}\label{weakform}
\begin{cases}
a(u,v) = l(v), \forall v \in V \\
u \in V(g).
\end{cases}
\end{equation}
Where $ a(u,v) := \int_{\Omega} \nabla u \cdot \nabla v \ dx$ which is bilinear form and $ l(v) := \int_{\Omega} f \ v \ dx $ which is linear form.
\begin{remark}
	$ \exists ! u = \underset{v \in V(g)}{argmin} \ (\dfrac{1}{2} a(v,v)-l(v)) = \underset{w \in V(g)}{argmin} \ J(w)$
\end{remark}
\textbf{Proposition}\\
For $ J(v) := \dfrac{1}{2} a(v,v) -l (v) $,\\
$ u = \underset{v \in V(g)}{argmin} \ J(v) \iff (\ref{weakform}) $

\textbf{Proof:}\\
$ (\Rightarrow) $ if $ u = argmin \ J $ then $ u+tv \in V(g) , \ \forall t \in \mathbb{R}, \forall v \in V = H_{0}^{1} (\Omega) $. Since it is on $ \Gamma $ or boundary, then $ g=u=u+tv $.
\begin{eqnarray}\nonumber
J(u) &\leq J(w) &, \forall w \in V(g), \ w=u+tv \in V(g)\\ \nonumber
J(u) &\leq J(u+tv) &, \forall t \in \mathbb{R}, \ \forall v \in V
\end{eqnarray}
Then
\begin{eqnarray}\nonumber
J(u+tv) &=& \dfrac{1}{2} a(u+tv,u+tv)-l(u+tv)\\ \nonumber
&=& \dfrac{1}{2} a(u,u) + t a(u,v) + \dfrac{t^2}{2} a (v,v) - l(u) - t l(v)\\ \nonumber
&=& \dfrac{t^2}{2} a (v,v) + t (a(u,v) - l(v)) + J(u)\\ \nonumber
&=:& \varphi(t)
\end{eqnarray}
Because $ \varphi(t) $ is in quadratic form, then its minimum obtained at $ t=0 $. So that $ \varphi =0 $ such that $ a(u,v) - l(v) =0 $.\\
$ (\Leftarrow) $ $ \forall t \in \mathbb{R}, \forall v \in V $ we have
\begin{equation}\nonumber
J(u,tv) = J(u) + \dfrac{t^2}{2} a(v,v) \geq J(u).
\end{equation}
$ \forall w \in V(g) $, we set $ v := w -u \in V , \ t:=1 , \ w=u+tv $
\begin{equation}\nonumber
J(w) = J(u+tv) \geq J(u)
\end{equation}

\section{Discrete (Finite Element Method)}
Notation
\begin{eqnarray}\nonumber
X_{h} &\subset & X \text{ (usually dim } X_{h} < \infty \text{)} \\ \nonumber
V_{h} & = & X_{h} \cap V\\ \nonumber
g_{h} & \in & X_{h} \text{(approximation of } g \text{)} \\ \nonumber
V_{h}(g_{h}) & = & \{ v_{h} \in X_{h} ; v_{h}-g_{h} \in V_{h} \}.
\end{eqnarray}
Then the weak form is approximated with
\begin{equation} \nonumber
\begin{cases}
a(u_{h}, v_{h}) &= l(v_{h}), \ \forall v_{h} \in V_{h} \\
u_{h} &\in V_{h}(g_{h})
\end{cases}
\iff u_{h} = \underset{w_{h} \in V_{h}(g_{h})}{argmin}J(w_{h})
\end{equation}

Finite element method
\begin{eqnarray}\nonumber
X_{h} &=& \{ v_{h} \in C^{0}(\overline{\Omega}) ; {v_{h}|}_{K} \text{ is linear} \} \\ \nonumber
V_{h} &=& X_{h} \cap H_{0}^{1}(\Omega)
\end{eqnarray}
\begin{align*}
X_h &= \langle \varphi_1, \cdots , \varphi_{N_p} \rangle, \{\varphi_i\}^{N_p}_{i=1} \text{ become a basis of the vector space } X_h\\
&=\{\sum_{i=1}^{N_p} c_i\varphi_i ; c_i \in \R \}\\
x &= (x_1,x_2) \in \R^2\\
\forall v_h \in X_h\\
v_h (\cdot) &= \sum_{i=1}^{N_p} v_h(P_i) \varphi_i(\cdot) \in X_h\\
w_h &:= \sum_{i=1}^{N_p} v_h(P_i) \varphi_i \in X_h\\
w_h(P_j) &= \sum_{i=1}^{N_p} v_h (P_i) \varphi_i (P_j)\\
&= \sum_{i=1}^{N_p} v_h (P_i) \delta_{ij}\\
&= v_h (P_j)\\
\end{align*}
\textbf{a basis of} $ V_h $
\begin{align*}
\Omega \cap \Gamma &= \Phi\\
\{\varphi_i ; P_i \in \Omega \} &\subset \{P_i \}_{i=1}^{N_p}\\
\text{for simplicity, we assume}\\
\{\varphi_i ; P_i \in \Omega \} &= \{P_i\}_{i=1}^N (N < N_p)
\end{align*}
$ \text{\textbf{s.t.}} \{P_i\}_{i=1}^N \subset \Omega \text{ and } \{P_i\}_{i=N+1}^{N_p} \subset \Gamma $
\begin{align*}
V_h &= \langle \varphi_1, \cdots , \varphi_N \rangle\\
(\ast \ast) a(u_h , v_h) &= l(v_h) (\forall v_h \in V_h)\\
&\Updownarrow \\
a(u_h , \varphi_i) &= l(\varphi_i) (i=1, \cdots , N)\\
&\Downarrow \text{choose } v_h = \varphi_i \in V\\
\forall v_h \in V_h, c_i &= v_h (P_i), v_h = \sum_{i=1}^{N} c_i \varphi_i \\
a(u_h,v_h) &= a(u_h,\sum_{i=1}^{N} c_i \varphi_i)\\
&=\sum_{i=1}^{N} c_i a (u_h, \varphi_i) = \sum_{i=1}^{N} c_i l(\varphi_i)\\
&= l(\sum_{i=1}^{N} c_i \varphi_i) = l(v_h)\\
\end{align*}
$ \text{\textbf{we set }} u_j := u_h(P_j) \ (j=1, \cdots , N_p) $
\begin{align*}
\text{boundary} \rightarrow u_j = g_j &= g_h(P_j) (j=N+1, \cdots , N_p)\\
u_h & \in v_h(g_h)\\
P_j  & \in \Gamma\\
\text{unknown} & : u_1, \cdots , u_N
\end{align*}
\begin{align*}
(\ast \ast) \Leftrightarrow \begin{cases}
a(u_h, \varphi_i) = l(\varphi_i) (i=1, \cdots , N)\\
u_h = \sum_{j=1}^{N} u_j \varphi_j + \sum_{j=N+1}^{N_p} g_j \varphi_j (u_h \in V_h (g_h))
\end{cases}
\end{align*}
\textbf{we set} $ a_{ij} := a(u_i, u_j) = a(u_j, u_i)\\
\text{\textbf{s.t.}} \sum_{j=1}^{N} a_{ij} u_{j} + \sum_{j=N+1}^{N_p} a_{ij} g_j = l(\varphi_i) (i=1, \cdots , N) $
\begin{align*}
\text{\textbf{we set }} A &:= (a_{ij}) \in \R_{\text{sym}}^{N \times N}\\
\text{\textbf{u}} &:= \begin{pmatrix}
u_i \\
\vdots \\
u_N
\end{pmatrix}\\
\text{\textbf{b}} &:= (l(u_i) - \sum_{j=N+1}^{N_p} a_{ij} g_j)_{i=1, \cdots , N}
(\ast \ast) \Leftrightarrow \text{A} \textbf{\text{u}} = \textbf{b}
\end{align*}
\end{document}