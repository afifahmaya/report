\documentclass[a4paper,12pt]{article}
\usepackage[a4paper, hmargin={1.5cm,1.5cm}, vmargin={2cm,2cm}]{geometry}
\usepackage{amsmath}
\usepackage{amsthm}
\usepackage{amsfonts}
\usepackage{color}
\usepackage[final]{graphicx}
\usepackage{subcaption}
\usepackage{wrapfig}
\newtheorem{prob}{Problem}[]
\newtheorem{prop}{Proposition}
\usepackage{amssymb}
\usepackage{enumitem}
\usepackage{listings}

\newcommand{\R}{\mathbb{R}}
\newcommand{\C}{\mathbb{C}}
\newcommand{\N}{\mathbb{N}}

\DeclareMathOperator*{\capmod}{\cap}

% Title Page
\title{Basics of Applied Analysis A Report}
\author{Alifian Mahardhika Maulana}


\begin{document}
\maketitle
\begin{prob}
	Discuss the stability of the difference schemes for the transport equation
	\begin{equation}\label{eq:1}
	u_x + bu_x = 0
	\end{equation}
	using von Neumann stability analysis:
	\begin{enumerate}
		\item "naive" explicit scheme
		\begin{equation}\label{eq:2}
		\frac{v_k^{n+1}-v_k^n}{\tau} + b \frac{v_{k+1}^{n}-v_{k-1}^n}{2h} = 0
		\end{equation}
		
		\item implicit scheme
		\begin{equation}\label{eq:3}
		\frac{v_k^{n+1}-v_k^n}{\tau} + b \frac{v_{k+1}^{n+1}-v_{k-1}^{n+1}}{2h} = 0
		\end{equation}
		
		\item Discuss the dissipation and dispersion properties of the implicit scheme in (b). Is it a satisfactory scheme for \eqref{eq:1}
	\end{enumerate}
\end{prob}
\textbf{Answer:}
\begin{enumerate}
	\item von Neumann stability analysis for "naive" explicit scheme:\\
	we rewrite \eqref{eq:2} become,
	\begin{equation}\label{eq:4}
	v_k^{n+1} = v_k^n - \frac{R}{2}(v_{k+1}^n - v_{k-1}^n), \quad R:=\frac{b\tau}{h}
	\end{equation}
	then substitute $v_{k+q} = e^{iq\xi}\hat{v}^n$ to \eqref{eq:4}, we get
	\begin{equation}\label{eq:5}
	\hat{v}^{n+1} = \hat{v}^n \left( 1- \frac{R}{2} (e^{i\xi}-e^{-i\xi}) \right)
	\end{equation}
	then we define, $g(\xi) = \left( 1- \frac{R}{2} (e^{i\xi}-e^{-i\xi}) \right) = 1-iR \sin (\xi)$, taking norm of $g(\xi)$ we get
	\begin{equation}\label{eq:6}
	|g(\xi)| = |1-iR\sin (\xi)| = \sqrt{1+R^2\sin^2(\xi)}, \quad \xi = (-\pi,\pi)
	\end{equation}
	by \eqref{eq:6} we get $|g(\xi)| > 1$, according to von Neumann stability, the explicit scheme \eqref{eq:2} is unstable.
	
	\item von Neumann stability analysis for implicit scheme:\\
	we rewrite \eqref{eq:3} become,
	\begin{equation}\label{eq:7}
	\frac{R}{2} \left(v_{k+1}^{n+1} - v_{k-1}^{n+1}\right) + v_k^{n+1} = v_k^n, \quad R:=\frac{b\tau}{h}
	\end{equation}
	then substitute $v_{k+q} = e^{iq\xi}\hat{v}^n$ to \eqref{eq:7}, we get
	\begin{equation}\label{eq:8}
	\begin{aligned}
	\left(1+\frac{R}{2} \left(e^{i\xi}-e^{-i\xi}\right) \right) \hat{v}^{n+1} &= \hat{v}^{n}\\
	\hat{v}^{n+1} &= \frac{1}{\left(1+\frac{R}{2} \left(e^{i\xi}-e^{-i\xi}\right) \right)} \hat{v}^{n}
	\end{aligned}
	\end{equation}
	then we define, $g(\xi) = \frac{1}{\left(1+\frac{R}{2} \left(e^{i\xi}-e^{-i\xi}\right) \right)} = \frac{1}{\left(1+iR\sin(\xi) \right)}$, taking norm of $g(\xi)$ we get
	\begin{equation}\label{eq:9}
	|g(\xi)| = \Bigr|\frac{1}{\left(1+iR\sin(\xi) \right)}\Bigr| = \frac{1}{\sqrt{1+R^2 \sin^2 (\xi)}}, \quad \xi = (-\pi,\pi)
	\end{equation}
	by \eqref{eq:9} we get $|g(\xi)| < 1$, according to von Neumann stability, the implicit scheme \eqref{eq:3} is stable.
	
	\item To analyze the dissipation and dispersion of \eqref{eq:3}, we substitute $v_{k+p}^{n+q} = e^{i(q\omega\tau + p\beta h)}$ to \eqref{eq:7} we get,
	\begin{equation}\label{eq:10}
	\begin{aligned}
	\frac{R}{2} \left(e^{i(\omega\tau + \beta h)} - e^{i(\omega\tau - \beta h)}\right) + e^{i\omega\tau} &= 1, \quad R:=\frac{b\tau}{h}\\
	\left(\frac{R}{2} \left(e^{i\beta h} - e^{-i\beta h}\right) + 1 \right) e^{i\omega \tau} &= 1\\
	\left(iR\sin(\beta h) + 1\right) e^{i\omega \tau} &= 1\\
	e^{i\omega \tau} &= \frac{1}{\left(iR\sin(\beta h) + 1\right)}
	\end{aligned}
	\end{equation}
	taking norm of $e^{i\omega \tau}$, we get
		\begin{equation}\label{eq:11}
		|e^{i\omega \tau}| = e^{-\omega_2\tau} = \frac{1}{R^2\sin^2(\beta h) + 1}
		\end{equation}
	by \eqref{eq:11}, $e^{-\omega_2\tau} < 1$, therefore according to von Neumann stability analysis, the implicit scheme on \eqref{eq:3} is \textbf{dissipative}.\\
	Then, to analyze the dispersion, we take $\arg(e^{i\omega\tau})$,
	\begin{equation}\label{eq:12}
	\begin{aligned}
	\arg(e^{i\omega\tau}) &= \arg(e^{i\omega_1\tau}) + \arg(e^{-\omega_2\tau})\\
	&= \omega_1\tau + 0
	\end{aligned}
	\end{equation}
	which, $\omega_1\tau = \arctan\left(\frac{Im(e^{i\omega \tau})}{Re(e^{i\omega \tau})}\right)$, we can get the real and imajiner part of $e^{i\omega \tau}$ by first multiplying it with it's rational factor,
	\begin{equation}\label{eq:13}
	e^{i\omega \tau} = \frac{1}{\left(iR\sin(\beta h) + 1\right)} \frac{\left(iR\sin(\beta h) - 1\right)}{\left(iR\sin(\beta h) - 1\right)} = -\frac{iR\sin(\beta h)+1}{R^2\sin^2+1}
	\end{equation}
	then, $\omega_1\tau = \arctan\left(\frac{Im(e^{i\omega \tau})}{Re(e^{i\omega \tau})}\right) = \frac{R\sin(\beta h)}{1}=R\sin(\beta h)$. Since $\omega_1\tau$ is not a constant, therefore, according to von Neumann stability analysis, the implicit scheme on \eqref{eq:3} is \textbf{dispersive}.
\end{enumerate}
\newpage
\begin{prob}
	Show that the following implicit difference schemes for approximating the solution to
	\begin{equation}
	u_t + bu_x = au_{xx}
	\end{equation}
	are unconditionally stable using the von Neumann stability analysis. Here $R = b\frac{\tau}{h},r=a\frac{\tau}{h^2}$.
	\begin{enumerate}
		\item \begin{equation}\label{eq:15}
		v_{k}^{n+1} + \frac{R}{2}(v_{k+1}^{n+1}-v_{k-1}^{n+1}) - r(v_{k+1}^{n+1}-2v_{k}^{n+1} + v_{k-1}^{n+1}) = v_{k}^{n}
		\end{equation}
		\item \begin{equation}\label{eq:16}
		v_{k}^{n+1} + \frac{R}{4}(v_{k+1}^{n+1}-v_{k-1}^{n+1}) - \frac{r}{2}(v_{k+1}^{n+1}-2v_{k}^{n+1} + v_{k-1}^{n+1}) = v_{k}^{n} - \frac{R}{4}(v_{k+1}^{n}-v_{k-1}^{n}) + \frac{r}{2}(v_{k+1}^{n}-2v_{k}^{n} + v_{k-1}^{n})
		\end{equation}
	\end{enumerate}
\end{prob}
\textbf{Answer:}
\begin{enumerate}
	\item Substitute $v_{k+q}=e^{iq\xi}\hat{v}^n$ to \eqref{eq:15} we get,
	\begin{equation}\label{eq:17}
	\begin{aligned}
	\hat{v}^{n+1} + \frac{R}{2}(e^{i\xi} - e^{-i\xi}) \hat{v}^{n+1} - r(e^{i\xi}-2+e^{-i\xi})\hat{v}^{n+1} = \hat{v}^n\\
	\hat{v}^{n+1} \bigg( 1 + \frac{R}{2}(e^{i\xi} - e^{-i\xi}) - r(e^{i\xi}-2+e^{-i\xi}) \bigg) = \hat{v}^n\\
	\hat{v}^{n+1} = \frac{1}{\bigg( 1 + \frac{R}{2}(e^{i\xi} - e^{-i\xi}) - r(e^{i\xi}-2+e^{-i\xi}) \bigg)} \hat{v}^n
	\end{aligned}
	\end{equation}
	then, we define:
	\begin{equation}\label{eq:18}
	\begin{aligned}
	g(\xi) &= \frac{1}{\bigg( 1 + \frac{R}{2}(e^{i\xi} - e^{-i\xi}) - r(e^{i\xi}-2+e^{-i\xi}) \bigg)} = \frac{1}{\bigg(1+iR\sin(\xi) +r(2\cos(\xi)-2)\bigg)}\\ 
	&=\frac{1}{\bigg(1+iR\sin(\xi) +r(-4\sin^2(\frac{\xi}{2}))\bigg)}
	\end{aligned}
	\end{equation}
	by \eqref{eq:18}, $g(\xi)<1$, according to von Neumann stability analysis, if $g(\xi)<1$ the scheme will be unconditionally stable.
	
	\item Substitute $v_{k+q}=e^{iq\xi}\hat{v}^n$ to \eqref{eq:16} we get,
	\begin{equation}\label{eq:19}
	\begin{aligned}
	\hat{v}^{n+1} + \frac{R}{4}(e^{i\xi} - e^{-i\xi}) \hat{v}^{n+1} - \frac{r}{2}(e^{i\xi}-2+e^{-i\xi})\hat{v}^{n+1} &= \hat{v}^n -\frac{R}{4}(e^{i\xi} - e^{-i\xi}) \hat{v}^{n} +\frac{r}{2}(e^{i\xi}-2+e^{-i\xi})\hat{v}^{n}\\
	\hat{v}^{n+1} \bigg( 1 + \frac{R}{4}(e^{i\xi} - e^{-i\xi})- \frac{r}{2}(e^{i\xi}-2+e^{-i\xi}) \bigg) &= \hat{v}^n
	\bigg( 1 -\frac{R}{4}(e^{i\xi} - e^{-i\xi}) +\frac{r}{2}(e^{i\xi}-2+e^{-i\xi}) \bigg)\\
	\hat{v}^{n+1} &= \frac{\bigg( 1 -\frac{R}{4}(e^{i\xi} - e^{-i\xi}) +\frac{r}{2}(e^{i\xi}-2+e^{-i\xi}) \bigg)}{\bigg( 1 + \frac{R}{4}(e^{i\xi} - e^{-i\xi})- \frac{r}{2}(e^{i\xi}-2+e^{-i\xi}) \bigg)}\hat{v}^n
	\end{aligned}
	\end{equation}
	then, we define:
	\begin{equation}\label{eq:20}
	\begin{aligned}
	g(\xi) &= \frac{\bigg( 1 -\frac{R}{4}(e^{i\xi} - e^{-i\xi}) +\frac{r}{2}(e^{i\xi}-2+e^{-i\xi}) \bigg)}{\bigg( 1 + \frac{R}{4}(e^{i\xi} - e^{-i\xi})- \frac{r}{2}(e^{i\xi}-2+e^{-i\xi}) \bigg)} = \frac{\bigg( 1 -\frac{R}{4}(2i\sin(\xi)) +\frac{r}{2}(2\cos(\xi)-2) \bigg)}{\bigg( 1 + \frac{R}{4}(2i\sin(\xi))- \frac{r}{2}(2\cos(\xi)-2) \bigg)}\\ 
	&=\frac{\bigg( 1 -\frac{R}{4}(2i\sin(\xi)) +\frac{r}{2}(-4\sin^2(\frac{\xi}{2})) \bigg)}{\bigg( 1 + \frac{R}{4}(2i\sin(\xi))- \frac{r}{2}(-4\sin^2(\frac{\xi}{2})) \bigg)}
	\end{aligned}
	\end{equation}
	by \eqref{eq:20}, $g(\xi)<1$, according to von Neumann stability analysis, if $g(\xi)<1$ the scheme will be unconditionally stable.
\end{enumerate}

\begin{prob}
	Discuss the dissipation and dispersion of the following implicit numerical schemes for the wave equation
	\begin{enumerate}
		\item \begin{equation}\label{eq:21}
		\frac{v_k^{n+1} -2v_k^n+v_k^{n-1}}{\tau^2} = \frac{v_{k+1}^{n+1}-2v_k^{n+1}+v_{k-1}^{n+1}}{h^2}
		\end{equation}
		\item \begin{equation}\label{eq:22}
		\frac{v_k^{n+1} -2v_k^n+v_k^{n-1}}{\tau^2} = \frac{v_{k+1}^{n+1}-2v_k^{n+1}+v_{k-1}^{n+1}}{2h^2} + \frac{v_{k+1}^{n-1}-2v_k^{n-1}+v_{k-1}^{n-1}}{2h^2}
		\end{equation}
	\end{enumerate}
\end{prob}
\textbf{Answer:}
\begin{enumerate}
	\item Substitute $v_{k+p}^{n+q} = e^{i(q\omega\tau + p\beta h)}\hat{v}^n$ with $R=\frac{\tau}{h}$ to \eqref{eq:21} we get,
	\begin{equation}\label{eq:23}
	\begin{aligned}
	\frac{(e^{i\omega\tau} -2 + e^{-i\omega\tau})}{\tau^2}\hat{v}^n &= \frac{(e^{i(\omega\tau+\beta h)}-2e^{i\omega\tau}+e^{i(\omega\tau-\beta h)})}{h^2}\hat{v}^n\\
	e^{i\omega\tau} -2 + e^{-i\omega\tau} &= R^2(2\cos(\beta h)-2)e^{i\omega\tau}\\
	\end{aligned}
	\end{equation}
	take $g=e^{i\omega\tau}$, \eqref{eq:23} become
	\begin{equation}
	\begin{aligned}
	-2 + g^{-1} + g\left(1 - R^2(2\cos(\beta h)-2)\right) &= 0\\
	-2 + g^{-1} + g\left(1 - R^2(-4\sin^2(\frac{\beta h}{2}))\right) &= 0\\
	\end{aligned}
	\end{equation}
	multiple by $g$ we get
	\begin{equation}\label{eq:25}
	\begin{aligned}
	g^2(1-R^2(-4\sin^2(\frac{\beta h}{2}))) -2g + 1 = 0
	\end{aligned}
	\end{equation}
	solving \eqref{eq:25} we get,
	\begin{equation}\label{eq:26}
	g = e^{i\omega\tau} = \frac{1\pm i2R\sin(\frac{\beta h}{2})}{(1-R^2(-4\sin^2(\frac{\beta h}{2})))}
	\end{equation}
	taking norm of \eqref{eq:26}, we get
	\begin{equation}\label{eq:27}
	|e^{i\omega\tau}| = e^{i\omega_2\tau} = \max_{-\pi \leq \frac{\beta h}{2} \leq \pi} \left| \frac{1\pm i2R\sin(\frac{\beta h}{2})}{(1-R^2(-4\sin^2(\frac{\beta h}{2})))} \right| = \frac{1\pm 2R}{1+4R^2}
	\end{equation}
	from \eqref{eq:27}, $e^{i\omega_2\tau} < 1$, therefore according to von Neumann stability analysis, the implicit numerical scheme on \eqref{eq:21} is \textbf{dissipative}.
	
	Then, to analyze the dispersion, we take $\arg(e^{i\omega\tau})$,
	\begin{equation}\label{eq:28}
	\begin{aligned}
	\arg(e^{i\omega\tau}) &= \arg(e^{i\omega_1\tau}) + \arg(e^{-\omega_2\tau})\\
	&= \omega_1\tau + 0
	\end{aligned}
	\end{equation}
	which, $\omega_1\tau = \arctan\left(\frac{Im(e^{i\omega \tau})}{Re(e^{i\omega \tau})}\right)$,
	then, $\omega_1\tau = \arctan\left(\frac{Im(e^{i\omega \tau})}{Re(e^{i\omega \tau})}\right) = \frac{2R\sin(\beta h)}{1}=2R\sin(\beta h)$. Since $\omega_1\tau$ is not a constant, therefore, according to von Neumann stability analysis, the implicit scheme on \eqref{eq:21} is \textbf{dispersive}.
	
	\item Substitute $v_{k+p}^{n+q} = e^{i(q\omega\tau + p\beta h)}\hat{v}^n$ with $R=\frac{\tau^2}{2h^2}$ to \eqref{eq:22} we get,
	\begin{equation}\label{eq:29}
	\begin{aligned}
	&\frac{(e^{i\omega\tau} -2 + e^{-i\omega\tau})}{\tau^2}\hat{v}^n = \frac{(e^{i(\omega\tau+\beta h)}-2e^{i\omega\tau}+e^{i(\omega\tau-\beta h)})}{2h^2}\hat{v}^n + \frac{(e^{i(-\omega\tau+\beta h)}-2e^{-i\omega\tau}+e^{-i(\omega\tau+\beta h)})}{2h^2}\hat{v}^n\\
	&(e^{i\omega\tau}-2+e^{-i\omega\tau}) = R \left( (2\cos(\beta h)-2)e^{i\omega\tau} + (2\cos(\beta h)-2)e^{-i\omega\tau} \right)\\
	&(e^{i\omega\tau}-2+e^{-i\omega\tau}) - R \left( (2\cos(\beta h)-2)e^{i\omega\tau} + (2\cos(\beta h)-2)e^{-i\omega\tau} \right) = 0\\
	&e^{i\omega\tau} \left( 1- R (2\cos(\beta h)-2)\right) + e^{-i\omega\tau} \left( 1- R (2\cos(\beta h)-2)\right) -2 = 0
	\end{aligned}
	\end{equation}
	take $g=e^{i\omega\tau}$, \eqref{eq:29} become
	\begin{equation}
	\begin{aligned}
	g \left( 1- R (2\cos(\beta h)-2)\right) + g^{-1} \left( 1- R (2\cos(\beta h)-2)\right) -2 = 0\\
	g \left( 1- R (-4\sin^2(\frac{\beta h}{2})\right) + g^{-1} \left( 1- R (-4\sin^2(\frac{\beta h}{2})\right) -2 = 0\\
	\end{aligned}
	\end{equation}
	multiple by $g$, we get
	\begin{equation}\label{eq:31}
	g^2 \left( 1- R (-4\sin^2(\frac{\beta h}{2})\right) + \left( 1- R (-4\sin^2(\frac{\beta h}{2})\right) -2g = 0
	\end{equation}
	solving \eqref{eq:31} we get,
	\begin{equation}\label{eq:32}
	g = e^{i\omega\tau} = \frac{2\pm \sqrt{4-4(1+4R\sin^2(\frac{\beta h}{2}))^2}}{2(1+4R\sin^2(\frac{\beta h}{2}))}
	\end{equation}
	taking norm of \eqref{eq:32}, we get
	\begin{equation}\label{eq:33}
	|e^{i\omega\tau}| = e^{i\omega_2\tau} = \max_{-\pi \leq \frac{\beta h}{2} \leq \pi} \left| \frac{2\pm \sqrt{4-4(1+4R\sin^2(\frac{\beta h}{2}))^2}}{2(1+4R\sin^2(\frac{\beta h}{2}))} \right| = \frac{2\pm (1+4R)}{2(1+4R)}
	\end{equation}
	from \eqref{eq:33}, $e^{i\omega_2\tau} = 1$, therefore according to von Neumann stability analysis, the implicit numerical scheme on \eqref{eq:22} is \textbf{non-dissipative}.
	
	Then, to analyze the dispersion, we take $\arg(e^{i\omega\tau})$,
	\begin{equation}\label{eq:28}
	\begin{aligned}
	\arg(e^{i\omega\tau}) &= \arg(e^{i\omega_1\tau}) + \arg(e^{-\omega_2\tau})\\
	&= \omega_1\tau + 0
	\end{aligned}
	\end{equation}
	which, $\omega_1\tau = \arctan\left(\frac{Im(e^{i\omega \tau})}{Re(e^{i\omega \tau})}\right)$,
	then, $\omega_1\tau = \arctan\left(\frac{Im(e^{i\omega \tau})}{Re(e^{i\omega \tau})}\right) = 0$. Since $\omega_1\tau$ is 0 because there is no imaginer part, therefore, according to von Neumann stability analysis, the implicit scheme on \eqref{eq:22} is \textbf{non-dispersive}.
\end{enumerate}
\end{document}