\documentclass[a4paper,12pt]{article}
\usepackage[a4paper, hmargin={2cm,2cm}, vmargin={2cm,2cm}]{geometry}
\usepackage{amsmath}
\usepackage{amsthm}
\usepackage{amsfonts}
\usepackage{color}
\usepackage[final]{graphicx}
\usepackage{subcaption}
\usepackage{wrapfig}
\newtheorem{prob}{Problem}[]
\newtheorem{prop}{Proposition}
\usepackage{amssymb}
\usepackage{enumitem}
\usepackage{listings}

\newcommand{\R}{\mathbb{R}}
\newcommand{\C}{\mathbb{C}}
\newcommand{\N}{\mathbb{N}}

\DeclareMathOperator*{\capmod}{\cap}

% Title Page
\title{Basics of Applied Analysis A\\Report}
\author{Alifian Mahardhika Maulana}


\begin{document}
\maketitle
\begin{prob}\label{problem1}
	Let $N \in \N, N \geq 2$, and $\alpha > 0$ be given.\\
	We define $A=(a_{ij})$ as finite difference discretization of :
	\begin{equation}\label{eq:1}
	\begin{cases}
	\alpha u - u'' = f\ \text{in}\ (0,1)\\
	u(0) = u(1) = 0
	\end{cases}
	\end{equation}
	with entries
	\begin{equation}\label{eq:2}
	a_{ij} = \begin{cases}
	\alpha + \frac{2}{h^2},\ &i = j\\
	-\frac{1}{h^2},\ &|i-j| = 1\\
	0\ &\text{otherwise}
	\end{cases}
	\end{equation}
	where $h=\frac{1}{N}$
	\begin{enumerate}[label=(\alph*)]
		\item To find eigenvalues $\lambda$ s.t. $Av = \lambda v$, We define $v_i := \sin\big(\frac{m\pi i}{N}\big)$ for some $m \in \N$, hence the discrete form of eigen problem becomes:
		\begin{equation*}
		\begin{aligned}
		&\sum_{i=1}^{N} a_{ij} v_j = \lambda v_i\\
		&\bigg(\alpha+\frac{2}{h^2}\bigg) v_i - \frac{1}{h^2} v_{i+1} - \frac{1}{h^2} v_{i-1} = \lambda v_i\\
		&\bigg(\alpha+\frac{2}{h^2} - \lambda\bigg) v_i - \frac{1}{h^2} (v_{i+1} + v_{i-1}) = 0\\
		&\text{substitute $v_i$ to the discrete eigen problem, it becomes}\\
		&\bigg(\alpha+\frac{2}{h^2} - \lambda\bigg) \sin\bigg(\frac{m\pi i}{N}\bigg) - \frac{1}{h^2} \bigg(\sin\bigg(\frac{m\pi (i+1)}{N}\bigg) + \sin\bigg(\frac{m\pi (i-1)}{N}\bigg)\bigg) = 0\\
		&\bigg(\alpha+\frac{2}{h^2} - \lambda\bigg) \sin\bigg(\frac{m\pi i}{N}\bigg) - \frac{1}{h^2} \bigg(2\sin\bigg(\frac{m\pi i}{N}\bigg) \cos\bigg(\frac{m\pi}{N}\bigg)\bigg) = 0\\
		&\sin\frac{m\pi i}{N} \bigg(\alpha+\frac{2}{h^2} - \lambda - \frac{1}{h^2} 2 \cos\bigg(\frac{m\pi}{N}\bigg)\bigg) = 0
		\end{aligned}
		\end{equation*}
		to find $\lambda$, we set $\sin\frac{m\pi i}{N}$ equal to some constant $C$, so that:
		\begin{equation*}
		\bigg(\alpha+\frac{2}{h^2} - \lambda - \frac{2}{h^2} \cos\bigg(\frac{m\pi}{N}\bigg)\bigg) = 0
		\end{equation*}
		\begin{equation}\label{eq:3}
		\therefore \lambda = \alpha+\frac{2}{h^2} \bigg(1 - \cos\bigg(\frac{m\pi}{N}\bigg) \bigg)
		\end{equation}
		With Eigenvectors:
		\begin{equation*}
		\therefore v = \sin\bigg(\frac{m\pi i}{N}\bigg)
		\end{equation*}
		
		\item We define spectral radius $\sigma(A) := \max\{|\lambda|\}$, using $\lambda$ computed in \eqref{eq:3}, we get:
		\begin{equation*}
		\begin{aligned}
		\sigma(A) &= \max\{|\lambda|\}\\
		&= \alpha+\frac{2}{h^2} \bigg(1 - \cos\bigg(\frac{m\pi}{N}\bigg) \bigg)
		\end{aligned}
		\end{equation*}
		then, thanks to the symmetricity of $\lambda$, we can use taylor expansion to approximate the value of $$\cos\bigg(\frac{m\pi}{N}\bigg) \approx 1 - \frac{1}{2}\bigg(\frac{\pi}{N}\bigg)^2$$, hence the spectral radius $\sigma(A)$ can be calculated by:
		\begin{equation*}
		\begin{aligned}
		\sigma(A) &\approx \alpha+\frac{2}{h^2} \bigg(1 - \bigg( 1 - \frac{m^2\pi^2}{2N^2}\bigg) \bigg)\\
		&\approx \alpha + \frac{2}{h^2} \frac{m^2\pi^2}{2N^2}\\
		&\therefore \sigma(A) \approx \alpha + (m\pi)^2
		\end{aligned}
		\end{equation*}
		
		\item Let:
		\begin{equation*}
		R := -D^{-1}(L+U)
		\end{equation*}
		be the Jacobi iteration matrix, we want to find the eigenvalues and eigenvectors s.t. $Rv = \lambda v$.
		\begin{equation*}
		\begin{aligned}
		Rv =& \lambda v\\
		-D^{-1}(L+U) v =& \lambda v\\
		-(L+U) v =& \lambda D v
		\end{aligned}
		\end{equation*}
		the discrete form of the eigen problem is:
		\begin{equation*}
		\begin{aligned}
		\frac{1}{h^2}(v_{i-1} + v_{i+1}) =& \lambda \bigg( \alpha + \frac{2}{h^2} \bigg) v_i
		\end{aligned}
		\end{equation*}
		and then we set, $v_i := \sin\big(\frac{m\pi i}{N}\big)$ and substitute to the discrete form of the eigen problem, thus:
		\begin{equation*}
		\begin{aligned}
		\frac{1}{h^2} \bigg(\sin\bigg(\frac{m\pi (i+1)}{N}\bigg) + \sin\bigg(\frac{m\pi (i-1)}{N}\bigg)\bigg) &= \lambda \bigg( \alpha + \frac{2}{h^2} \bigg) \sin\bigg(\frac{m\pi i}{N}\bigg)\\
		\frac{1}{h^2} \bigg(2\sin\bigg(\frac{m\pi i}{N}\bigg) \cos\bigg(\frac{m\pi}{N}\bigg)\bigg) &= \lambda \bigg( \alpha + \frac{2}{h^2} \bigg) \sin\bigg(\frac{m\pi i}{N}\bigg)\\
		\sin\frac{m\pi i}{N} \bigg( \frac{2}{h^2} \cos\bigg(\frac{m\pi}{N}\bigg) - \lambda \bigg( \alpha + \frac{2}{h^2} \bigg) \bigg) &= 0
		\end{aligned}
		\end{equation*}
		to find $\lambda$, we set $\sin\frac{m\pi i}{N}$ equal to some constant $D$, so that:
		\begin{equation}\label{eq:4}
		\begin{aligned}
		&\bigg( \frac{2}{h^2} \cos\bigg(\frac{m\pi}{N}\bigg) - \lambda \bigg( \alpha + \frac{2}{h^2} \bigg) \bigg) = 0\\
		&\lambda \bigg( \alpha + \frac{2}{h^2} \bigg) = \frac{2}{h^2} \cos\bigg(\frac{m\pi}{N}\bigg)\\
		\text{with $h=\frac{1}{N}$}\\
		&\lambda = \frac{\frac{2}{h^2} \cos\big(\frac{m\pi}{N}\big)}{\alpha + \frac{2}{h^2}}
		\end{aligned}
		\end{equation}
		with $h=\frac{1}{N}$, $$\therefore \lambda = \frac{2N^2 \cos\big(\frac{m\pi}{N}\big)}{\alpha + 2N^2}$$
		
		\item We define spectral radius $\sigma(R) := \max\{|\lambda|\}$, using $\lambda$ computed in \eqref{eq:4}, we get:
		\begin{equation*}
		\begin{aligned}
		\sigma(R) &= \max\{|\lambda|\}\\
		&= \frac{2N^2 \cos\big(\frac{m\pi}{N}\big)}{\alpha + 2N^2}
		\end{aligned}
		\end{equation*}
		again, thanks to the symmetricity of $\lambda$, we can use taylor expansion to approximate the value of $$\cos\bigg(\frac{m\pi}{N}\bigg) \approx 1 - \frac{1}{2}\bigg(\frac{m\pi}{N}\bigg)^2$$, hence the spectral radius $\sigma(R)$ can be calculated by:
		\begin{equation*}
		\begin{aligned}
		\therefore\sigma(R) &\approx \frac{2N^2 \bigg( 1 - \frac{m^2\pi^2}{2N^2}\bigg)}{\alpha + 2N^2}\\
		\end{aligned}
		\end{equation*}
	\end{enumerate}
\end{prob}
\newpage
\begin{prob}
	Let $N,h=\frac{1}{N}, \alpha$ and $A$ be as in Problem \ref{problem1}. We define:
	\begin{equation}\label{eq:5}
	\begin{cases}
	\alpha u - u'' = \sin(\pi x)\ \text{in}\ (0,1)\\
	u(0) = u(1) = 0
	\end{cases}
	\end{equation}
	
	\begin{enumerate}[label=(\alph*)]
		\item To find the exact solution of $u$, we use general solution for Ordinary Differential Equation (ODE):
		\begin{equation}\label{eq:6}
		u(x) = A \sin(\pi x) + B \cos(\pi x)
		\end{equation}
		and take the second derivative of $u$
		\begin{equation}\label{eq:7}
		u''(x) = -A\pi^2 \sin(\pi x) + B\pi^2 \cos(\pi x)
		\end{equation}
		then substitute \eqref{eq:6} and \eqref{eq:7} to \eqref{eq:5}, we get:
		\begin{equation}\label{eq:8}
		\begin{aligned}[center]
		&\alpha(A \sin (\pi x) + B \cos (\pi x)) + A\pi^2 \sin(\pi x) + B\pi^2 \cos(\pi x) &= \sin (\pi x)\\
		&A(\alpha + \pi^2) \sin (\pi x) + B(\alpha + \pi^2) \cos (\pi x) &= \sin (\pi x)
		\end{aligned}
		\end{equation}
		from \eqref{eq:8} we know that $$A(\alpha + \pi^2) \sin (\pi x) = \sin (\pi x)$$ and $$B(\alpha + \pi^2) \cos (\pi x) = 0$$
		\begin{eqnarray}\label{eq:9}
		\therefore A = \frac{1}{(\alpha + \pi^2)} & \therefore B = 0
		\end{eqnarray}
		then we substitute \eqref{eq:9} to \eqref{eq:6}, we get the exact solution as:
		\begin{equation}
		u(x) = \frac{\sin(\pi x)}{(\alpha + \pi^2)}
		\end{equation}
		
		\item To find exact solution s.t. $Av = b$, with $b_i = \sin(\pi h i)$, we set $$v_i = C \sin (\pi h i)$$ thus,
		\begin{equation*}
		\begin{aligned}
		A v &= b\\
		\sum_{i=1}^{N-1}a_{ij} v_i &= b_i\\
		\bigg( \bigg(\alpha + \frac{2}{h^2}\bigg)v_i - \frac{1}{h^2} \big( v_{i+1} + v_{i-1}\big) \bigg) &= \sin(\pi h i)\\
		\bigg( \bigg(\alpha + \frac{2}{h^2}\bigg)C \sin (\pi h i) - \frac{1}{h^2} \big( C \sin (\pi h (i+1)) + C \sin (\pi h (i-1))\big) \bigg) &= \sin(\pi h i)\\
		\bigg( \bigg(\alpha + \frac{2}{h^2}\bigg)C \sin (\pi h i) - C \frac{1}{h^2} \big( 2\sin (\pi h i) \cos (\pi h)\big) \bigg) &= \sin(\pi h i)\\
		\end{aligned}
		\end{equation*}
		divide by $\sin (\pi h i)$ on the both side we get:
		\begin{equation*}
		C \bigg( \bigg(\alpha + \frac{2}{h^2}\bigg) - \frac{2}{h^2} \cos (\pi h) \bigg) = 1\\
		\end{equation*}
		$$\therefore C = \frac{1}{\bigg( \bigg(\alpha + \frac{2}{h^2}\bigg) - \frac{2}{h^2} \cos (\pi h) \bigg)}$$
		so the exact solution in terms of $v_i$ is: $$v_i = \frac{\sin (\pi h i)}{\bigg( \bigg(\alpha + \frac{2}{h^2}\bigg) - \frac{2}{h^2} \cos (\pi h) \bigg)}$$
		
		\item We define $$\epsilon(h) := \max_{1\leq i \leq N-1} |u(hi) - v_i|$$
		then we use $u(x)$ and $v_i$ that we derived before to find the explicit formula of $\epsilon(h)$:
		\begin{equation*}
		\begin{aligned}
		\epsilon(h) &:= \max_{1\leq i \leq N-1} \Biggr|\frac{\sin(\pi hi)}{(\alpha + \pi^2)} - \frac{\sin (\pi h i)}{\bigg( \bigg(\alpha + \frac{2}{h^2}\bigg) - \frac{2}{h^2} \cos (\pi h) \bigg)}\Biggr|\\
		\epsilon(h) &:= \max_{1\leq i \leq N-1} \Biggr|\sin(\pi hi) \Bigg( \frac{1}{(\alpha + \pi^2)} - \frac{1}{\bigg( \bigg(\alpha + \frac{2}{h^2}\bigg) - \frac{2}{h^2} \cos (\pi h) \bigg)} \Bigg) \Biggr|
		\end{aligned}
		\end{equation*}
		for simplicity, we rewrite:
		$$D :=\sin(\pi hi)$$
		$$E := \Bigg( \frac{1}{(\alpha + \pi^2)} - \frac{1}{\bigg( \bigg(\alpha + \frac{2}{h^2}\bigg) - \frac{2}{h^2} \cos (\pi h) \bigg)} \Bigg)$$
		\begin{equation*}
		\begin{aligned}
		\max_{1\leq i \leq N-1} \bigr|DE\bigr| &= \max_{1\leq i \leq N-1} \bigr|D\bigr| \max_{1\leq i \leq N-1} \bigr|E\bigr|\\
		& = 1 \Bigg( \frac{1}{(\alpha + \pi^2)} - \frac{1}{\bigg( \bigg(\alpha + \frac{2}{h^2}\bigg) - \frac{2}{h^2} \cos (\pi h) \bigg)} \Bigg)
		\end{aligned}
		\end{equation*}
		then, the explicit formula for $\epsilon (h)$ is:
		\begin{equation}\label{eq:11}
		\epsilon(h) = \Bigg( \frac{1}{(\alpha + \pi^2)} - \frac{1}{\bigg( \bigg(\alpha + \frac{2}{h^2}\bigg) - \frac{2}{h^2} \cos (\pi h) \bigg)} \Bigg)
		\end{equation}
		around $h=0$, leading order of taylor expansion of $\epsilon (h)$ can be obtained by expand $\cos(\pi h)$, so that
		$$\cos(\pi h) \approx \cos(a) + \frac{\cos'(a)(\pi h - a)}{1!} + \frac{\cos''(a)(\pi h - a)^2}{2!} + \cdots$$
		then we take $a=0$, we get
		$$\cos(\pi h) \approx 1-\frac{1}{2}(\pi h)^2$$
		substitute it to \eqref{eq:11} we get:
		\begin{equation*}
		\begin{aligned}
		\epsilon(h) &= \Bigg( \frac{1}{(\alpha + \pi^2)} - \frac{1}{\bigg( \bigg(\alpha + \frac{2}{h^2}\bigg) - \frac{2}{h^2} (1-\frac{1}{2}(\pi h)^2) \bigg)} \Bigg)\\
		\epsilon(h) &= \Bigg( \frac{1}{(\alpha + \pi^2)} - \frac{1}{(\alpha + \pi^2)} \Bigg)
		\end{aligned}
		\end{equation*}
		which is just a constant, so the leading order is depend on what order of taylor expansion we choose to approximate the value of $\cos(\pi h)$ around $h=0$. In this case if we choose taylor expansion orde 2, we get the leading term of $\epsilon$ is $\mathbb{O}(F)$ with $F$ is a constant.
	\end{enumerate}
\end{prob}
\newpage
\begin{prob}
	We consider system of linear equations:
	\begin{equation}\label{eq:12}
	\begin{cases}
	-v_{i-1,j}-v_{i+1,j}-v_{i,j-1}-v_{i,j+1}+4v_{i,j} = b_{i,j} & i,j = 1,\cdots , N-1\\
	v_{0,j} = v_{0,N} = v_{i,0} = v_{i,N} = 0 & i,j = 1,\cdots , N-1
	\end{cases}
	\end{equation}
	for unknowns $v_{i,j}, i,j = 1,\cdots , N-1$.
	\begin{enumerate}[label=(\alph*)]
		\item We want to find the eigenvalues and eigenvectors of the matrix $A$ for the system \eqref{eq:12}. Using the idea of discrete separation variables, we set $w_{i,j} = v_i \tilde{v_j}$ s.t.
		$$Aw_{i,j} = \lambda w_{i,j}$$
		\begin{equation*}
		\begin{aligned}
		4w_{i,j}-w_{i-1,j}-w_{i+1,j}-w_{i,j-1}-w_{i,j+1} &= \lambda w_{i,j}\\
		4(v_i \tilde{v_j})-(v_{i-1} \tilde{v_j})-(v_{i+1} \tilde{v_j})-(v_i \tilde{v}_{j-1})-(v_i \tilde{v}_{j+1}) &= \lambda (v_i \tilde{v_j})\\
		4(v_i \tilde{v_j}) -\tilde{v_j} (v_{i-1} + v_{i+1}) -v_i( \tilde{v}_{j-1} + \tilde{v}_{j+1}) &= \lambda (v_i \tilde{v_j})\\
		(4-\lambda)(v_i \tilde{v_j}) -\tilde{v_j} (v_{i-1} + v_{i+1}) -v_i( \tilde{v}_{j-1} + \tilde{v}_{j+1}) &= 0\\
		\end{aligned}
		\end{equation*}
		then, divide by $v_i \tilde{v_j}$ on both side, we get:
		\begin{equation*}
		\begin{aligned}
		(4-\lambda) -\frac{(v_{i-1} + v_{i+1})}{v_i} -\frac{( \tilde{v}_{j-1} +\tilde{v}_{j+1})}{\tilde{v}_j} = 0\\
		(4-\lambda) -\frac{(v_{i-1} + v_{i+1})}{v_i} = \frac{( \tilde{v}_{j-1} +\tilde{v}_{j+1})}{\tilde{v}_j}\\
		\end{aligned}
		\end{equation*}
		now, to satisfies equality, both side should be equal to some constant, let say $G$, hence:
		\begin{equation*}
		\begin{aligned}
		(4-\lambda) -\frac{(v_{i-1} + v_{i+1})}{v_i} = G = \frac{( \tilde{v}_{j-1} +\tilde{v}_{j+1})}{\tilde{v}_j}\\
		\end{aligned}
		\end{equation*}
		
		First, we solve for:
		\begin{equation*}
		\begin{aligned}
		\frac{( \tilde{v}_{j-1} +\tilde{v}_{j+1})}{\tilde{v}_j} = G\\
		( \tilde{v}_{j-1} +\tilde{v}_{j+1}) = \tilde{v}_j G\\
		\end{aligned}
		\end{equation*}
		set: $\tilde{v}_j = \varphi^j$
		\begin{equation*}
		\begin{aligned}
		\varphi^{j-1} + \varphi^{j+1} = \varphi^j G\\
		\varphi^{j-1} + \varphi^{j+1} - \varphi^j G = 0\\
		\end{aligned}
		\end{equation*}
		divide both side by $\varphi^{j-1}$
		\begin{equation}\label{eq:13}
		\begin{aligned}
		1 + \varphi^2 - \varphi G = 0\\
		\therefore \varphi_\pm = \frac{G \pm \sqrt{(G^2 - 4)}}{2}
		\end{aligned}
		\end{equation}
		here we recall that eigenvector of $\tilde{v}_j$ should be a linear combination of $\varphi$ s.t. $$\tilde{v}_j = c_1\varphi_+^j + c_2\varphi_-^j$$
		and we know the boundary condition on \eqref{eq:12}, when $j=N$ then $\tilde{v}_N=0$, to satisfies this condition, then $c_1 = -c_2$, then $$\tilde{v}_j = c_1(\varphi_+^j - \varphi_-^j)$$ $\therefore$ $\varphi_+^j$ and $\varphi_-^j$ should be distinguished.\\
		Hence, we choose $G<2$ for \eqref{eq:13} we then rewrite \eqref{eq:13} as:
		\begin{equation*}
		\begin{aligned}
		\varphi_\pm = \frac{G \pm \sqrt{-(4-G^2)}}{2}\\
		\varphi_\pm = \frac{G \pm i\sqrt{(4-G^2)}}{2}
		\end{aligned}
		\end{equation*}
		then, the possible solution is $$\varphi_\pm^j = \cos(\theta j) \pm i \sin(\theta j)$$
		\begin{equation*}
		\begin{aligned}
		\therefore \tilde{v}_j &= c_1(\cos(\theta j) + i \sin(\theta j) - (\cos(\theta j) - i \sin(\theta j)))\\
		\tilde{v}_j &= c_1(2 i \sin(\theta j))
		\end{aligned}
		\end{equation*}
		we choose $c_1 = -\frac{i}{2}$ so that $$\tilde{v}_j = \sin(\theta j)$$
		inserting boundary condition, when $j=N$ then $\tilde{v}_N=0$, we get:
		\begin{equation*}
		\begin{aligned}
		0 &= \sin(\theta N)\\
		\theta &= m\pi,\ \forall m:2,4,6\cdots N,\ \text{even}\\
		\theta &= \frac{m\pi}{N}
		\end{aligned}
		\end{equation*}
		then, we know that from imaginer triangle $$\cos \theta = \frac{G}{2}$$
		$\therefore G = 2 \cos(\theta) = 2 \cos(\frac{m\pi}{N})$
		
		Second, we solve for:
		$$(4-\lambda) -\frac{(v_{i-1} + v_{i+1})}{v_i} = G$$
		multiplied both side by $v_i$, we get:
		\begin{equation*}
		\begin{aligned}
		v_i(4-\lambda) -(v_{i-1} + v_{i+1}) = v_i G\\
		v_i(4-\lambda) -(v_{i-1} + v_{i+1}) - v_i G = 0\\
		v_i(4-\lambda - G) -(v_{i-1} + v_{i+1}) = 0\\
		\end{aligned}
		\end{equation*}
		set $v_i = \xi^i$
		\begin{equation*}
		\begin{aligned}
		\xi^i(4-\lambda - G) -(\xi^{i-1} + \xi^{i+1}) = 0\\
		\end{aligned}
		\end{equation*}
		divide both side by $\xi^{i-1}$
		\begin{equation*}
		\begin{aligned}
		\xi(4-\lambda - G) -(1 + \xi^2) = 0\\
		\xi^2 - \xi(4-\lambda - G) + 1 = 0\\
		\therefore \xi_\pm = \frac{(4-\lambda - G) \pm \sqrt{(4-\lambda - G)^2 - 4}}{2}
		\end{aligned}
		\end{equation*}
		take $2H = (4-\lambda - G)$
		\begin{equation}\label{eq:14}
		\begin{aligned}
		\xi_\pm = \frac{2H \pm \sqrt{(2H)^2 - 4}}{2}\\
		\xi_\pm = H \pm \sqrt{H^2 - 1}
		\end{aligned}
		\end{equation}
		here we recall that eigenvector of $v_i$ should be a linear combination of $\xi$ s.t. $$v_i = d_1\xi_+^i + d_2\xi_-^i$$
		and we know the boundary condition on \eqref{eq:12}, when $i=N$ then $v_N=0$, to satisfies this condition, then $d_1 = -d_2$, then $$v_i = d_1(\xi+^j - \xi_-^j)$$ $\therefore$ $\xi_+^j$ and $\xi_-^j$ should be distinguished.\\
		Hence, we choose $H<1$ for \eqref{eq:14} we then rewrite \eqref{eq:14} as:
		\begin{equation*}
		\begin{aligned}
		\xi_\pm = H \pm \sqrt{-(1-H^2)}\\
		\xi_\pm = H \pm i\sqrt{(1-H^2)}
		\end{aligned}
		\end{equation*}
		then, the possible solution is $$\xi_\pm^i = \cos(\theta i) \pm I \sin(\theta i)$$
		here i use "I" as imaginer number, so it is not confusing between "i" index and "I" imaginer number.
		\begin{equation*}
		\begin{aligned}
		\therefore v_i &= d_1(\cos(\theta i) + I \sin(\theta i) - (\cos(\theta i) - I \sin(\theta i)))\\
		v_i &= d_1(2 I \sin(\theta i))
		\end{aligned}
		\end{equation*}
		we choose $d_1 = -\frac{I}{2}$ so that $$v_i = \sin(\theta i)$$
		inserting boundary condition, when $i=N$ then $v_N=0$, we get:
		\begin{equation*}
		\begin{aligned}
		0 &= \sin(\theta N)\\
		\theta &= m\pi,\ \forall m:2,4,6\cdots N,\ \text{even}\\
		\theta &= \frac{m\pi}{N}
		\end{aligned}
		\end{equation*}
		then, we know that from imaginer triangle $$\cos \theta = H$$
		$\therefore H = \cos(\theta) = \cos(\frac{m\pi}{N})$\\
		substitute $H$ and $G$ to $2H=(4-\lambda-G)$, we get
		\begin{equation*}
		\begin{aligned}
		2 \cos\bigg(\frac{m\pi}{N}\bigg) = 4-\lambda -2 \cos\bigg(\frac{m\pi}{N}\bigg)\\
		\lambda = 4 -2 \cos\bigg(\frac{m\pi}{N}\bigg) -2 \cos\bigg(\frac{m\pi}{N}\bigg)\\
		\therefore \lambda = 4 -2 \bigg(\cos\bigg(\frac{m\pi}{N}\bigg) + \cos\bigg(\frac{m\pi}{N}\bigg) \bigg)\\
		\end{aligned}
		\end{equation*}
		for the eigenvectors, we already know that 
		$$w_{i,j} = v_i \tilde{v_j}$$
		$$v_i = \sin(\theta i)$$
		$$\tilde{v}_j = \sin(\theta j)$$
		$$\therefore w_{i,j} = \sin\bigg(\frac{m\pi i}{N}\bigg) \sin\bigg(\frac{m\pi j}{N}\bigg)$$
		
		\item I attached the Python Code that solves \eqref{eq:12} in \ref{code}
		
		\item Minimal Number of Iterations K such the:
		\begin{equation*}
		\max_{1\leq i,j\leq N-1} \Bigg|v_{i,j}^{(K+1)} - v_{i,j}^{(K)}\Bigg| \leq 10^{-4}
		\end{equation*}
		\begin{enumerate}
			\item N = 10
			\begin{enumerate}
				\item Jacobi, K = 53
				\item Gauss-Seidel, K = 33
				\item SOR (omega=1.5), K = 12
			\end{enumerate}
			\item N = 20
			\begin{enumerate}
				\item Jacobi, K = 137
				\item Gauss-Seidel, K = 94
				\item SOR (omega=1.5), K = 45
			\end{enumerate}
			\item N = 50
			\begin{enumerate}
				\item Jacobi, K = 1
				\item Gauss-Seidel, K = 173
				\item SOR (omega=1.5), K = 148
			\end{enumerate}
		\end{enumerate}
		
		\item Optimal $\omega_b$ for SOR:
		\begin{enumerate}
			\item N = 10, $\omega_b$ = 1.16527511683289
			\item N = 20, $\omega_b$ = 1.2933223938735654
			\item N = 50, $\omega_b$ = 1.3746381523099196
		\end{enumerate}
	\end{enumerate}
\end{prob}
\newpage
\section{Attachment}\label{code}
\lstinputlisting[basicstyle=\ttfamily\tiny,language=Python]{linsolv.py}
\end{document}