\documentclass[a4paper,9pt]{article}
\usepackage[a4paper, hmargin={1.5cm,1.5cm}, vmargin={1.5cm,1.5cm}]{geometry}
\usepackage{amsmath}
\usepackage{amsthm}
\usepackage{amsfonts}
\usepackage{color}
\usepackage[final]{graphicx}
\usepackage{subcaption}
\usepackage{wrapfig}
\newtheorem{remark}{Remark}[]
\newtheorem{prop}{Proposition}
\newtheorem{prob}{Problem}
\usepackage{amssymb}
\usepackage{verbatim}

\newcommand{\R}{\mathbb{R}}
\newcommand{\C}{\mathbb{C}}
\newcommand{\N}{\mathbb{N}}
\newcommand{\Q}{\mathbb{Q}}
\newcommand{\W}{\mathbb{W}}
\newcommand{\Vspace}{\mathbb{V}}
\newcommand{\Hspace}{\mathbb{H}}
\newcommand{\Lspace}{\mathbb{L}}
\newcommand{\Lagr}{\mathcal{L}}


% Title Page
\title{Topics in Computational Science Report \\ Derivation of Position Function on Chaos Process}
\author{Alifian Mahardhika Maulana}


\begin{document}
\maketitle
\begin{prob}
	For $\alpha \in \R^+$ and $x(t) : \R^+ \rightarrow \R$, Let:
	\begin{equation}
	\frac{dx}{dt} = \alpha(1-x)x
	\end{equation}
	Proof the following:
	\begin{equation*}
	x(t) = \frac{1}{1+Ae^{-\alpha t}},\ \text{with}\ A=\frac{1-x(0)}{x(0)}
	\end{equation*}
\end{prob}
\begin{proof}
	\begin{equation*}
	\begin{aligned}[center]
		\frac{dx}{dt} &= \alpha(1-x)x\\
		\frac{dx}{(1-x)x} &= \alpha dt\\
		\bigg(\frac{1}{x} + \frac{1}{(1-x)}\bigg) dx &= \alpha dt\\
	\end{aligned}
	\end{equation*}
	taking integral on the both side, we obtain:
	\begin{equation*}
	\begin{aligned}[center]
	\int \frac{1}{x} dx + \int \frac{1}{(1-x)} dx &= \int \alpha dt\\
	\ln(x) - \ln(1-x) &= \alpha t + C\\
	\ln\bigg(\frac{x}{(1-x)}\bigg) &= \alpha t + C
	\end{aligned}
	\end{equation*}
	multiplied by $e(\exp)$ on the both side, we obtain:
	\begin{equation*}
	\begin{aligned}[center]
	\exp\bigg(\ln\bigg(\frac{x}{(1-x)}\bigg)\bigg) &= \exp(\alpha t + C)\\
	\frac{x}{(1-x)} &= e^{\alpha t + C}\\
	\end{aligned}
	\end{equation*}
	let $e^{\alpha t + C} = e^{\alpha t}e^{C}=Ce^{\alpha t}$, thus:
	\begin{equation}\label{eq:2}
	\begin{aligned}[center]
	\frac{x}{(1-x)} &= Ce^{\alpha t}\\
	\end{aligned}
	\end{equation}
	for $t=0$, we get:
	\begin{equation*}
	\begin{aligned}[center]
	\frac{x(0)}{(1-x(0))} &= C\\
	\end{aligned}
	\end{equation*}
	insert $C$ to equation \eqref{eq:2}, hence the equation becomes:
	\begin{equation*}\label{eq:3}
	\begin{aligned}[center]
	\frac{x}{(1-x)} &= \frac{x(0)}{(1-x(0))}e^{\alpha t}\\
	\frac{(1-x)}{x} &= \frac{(1-x(0))}{x(0)}e^{-\alpha t}\\
	\frac{1}{x} - 1 &= \frac{(1-x(0))}{x(0)}e^{-\alpha t}\\
	\end{aligned}
	\end{equation*}
	let $A=\frac{(1-x(0))}{x(0)}$ thus the equation becomes:
	\begin{equation*}
	\begin{aligned}[center]
	\frac{1}{x} - 1 &= Ae^{-\alpha t}\\
	\frac{1}{x} &= 1 + Ae^{-\alpha t}\\
	\therefore x &= \frac{1}{1 + Ae^{-\alpha t}}
	\end{aligned}
	\end{equation*}
\end{proof}
\end{document}