\documentclass[a4paper,9pt]{article}
\usepackage[a4paper, hmargin={1.5cm,1.5cm}, vmargin={1.5cm,1.5cm}]{geometry}
\usepackage{amsmath}
\usepackage{amsthm}
\usepackage{amsfonts}
\usepackage{color}
\usepackage[final]{graphicx}
\usepackage{subcaption}
\usepackage{wrapfig}
\newtheorem{remark}{Remark}[]
\newtheorem{prop}{Proposition}
\newtheorem{prob}{Problem}
\usepackage{amssymb}
\usepackage{verbatim}
\usepackage{listings}
\usepackage{multicol}

\newcommand{\R}{\mathbb{R}}
\newcommand{\C}{\mathbb{C}}
\newcommand{\N}{\mathbb{N}}
\newcommand{\Q}{\mathbb{Q}}
\newcommand{\W}{\mathbb{W}}
\newcommand{\Vspace}{\mathbb{V}}
\newcommand{\Hspace}{\mathbb{H}}
\newcommand{\Lspace}{\mathbb{L}}
\newcommand{\Lagr}{\mathcal{L}}


% Title Page
\title{Topics in Computational Science Report \\ Norbert Pozar Class}
\author{Alifian Mahardhika Maulana}


\begin{document}
\maketitle
\begin{enumerate}
	\item Suppose that $u$ is a twice continuously differentiable positive solution of the porous medium equation:
	\begin{equation}\label{eq:1}
	u_t - \triangle (u^m) = 0, \quad x \in \R^n, t>0
	\end{equation}
	In the pressure form:
	\begin{equation}\label{eq:2}
	p_t - (m-1)p \triangle p - |\triangledown p|^2 = 0
	\end{equation}
	Then, we define:
	\begin{equation}\label{eq:3}
	p(x,t) := \frac{m}{m-1} u^{m-1}(x,t)
	\end{equation}
	Show that \eqref{eq:3} is a solution of \eqref{eq:1} in \eqref{eq:2} form.\\
	\newline
	\textbf{Answer:}\\
	First we compute:
	\begin{eqnarray} \label{eq:4}
	p_t = \frac{d}{dt}p(x,t), & \triangle p = \frac{d^2}{dx^2} p(x,t) = \frac{d}{dx} \triangledown p\\ \label{eq:5}
	\triangledown p = \frac{d}{dx} p(x,t), & \triangle (u^m) = \frac{d^2}{dx^2} u^m
	\end{eqnarray}
	Applied Chain Rule to \eqref{eq:4},\eqref{eq:5}, we get:
	\begin{multicols}{2}
		\begin{equation}\label{eq:6}
		\begin{aligned}
		p_t &= \frac{d}{dt}p(x,t)\\
		&= \frac{dp}{du} \frac{du}{dt}\\
		&= \frac{m}{(m-1)} (m-1) u^{(m-2)} \frac{d}{dt}u\\
		&= m u^{(m-2)} u_t
		\end{aligned}
		\end{equation}
		\newline
		\begin{equation}\label{eq:7}
		\begin{aligned}
		\triangledown p &= \frac{d}{dx} p(x,t)\\
		&= \frac{dp}{du} \frac{du}{dx}\\
		&= \frac{m}{(m-1)} (m-1) u^{(m-2)} \frac{d}{dx}u\\
		&= m u^{(m-2)} \triangledown u
		\end{aligned}
		\end{equation}
		\columnbreak
		\begin{equation}\label{eq:8}
		\begin{aligned}
		\triangle p &= \frac{d}{dx} \triangledown p\\
		&= \frac{d}{dx} \big(mu^{(m-2)} \triangledown u\big)\\
		&=\big(m(m-2)u^{(m-3)}\triangledown u\big)\triangledown u + mu^{(m-2)} \triangle u\\
		&= m(m-2)u^{(m-3)} |\triangledown u|^2 + mu^{(m-2)} \triangle u
		\end{aligned}
		\end{equation}
		\newline
		\begin{equation}\label{eq:9}
		\begin{aligned}
		\triangle (u^m) &= \frac{d^2}{dx^2} u^m\\
		&= \frac{d}{dx} \bigg(\frac{d}{dx}u^m\bigg)\\
		&= \frac{d}{dx} \bigg(mu^{(m-1)}\triangledown u\bigg)\\
		&= \big(m(m-1)u^{(m-2)}\triangledown u\big)\triangledown u + (mu^{(m-1)}) \triangle u\\
		&= m(m-1)u^{(m-2)}|\triangledown u|^2 + mu^{(m-1)}
		\end{aligned}
		\end{equation}
	\end{multicols}
	Substitute \eqref{eq:6}, \eqref{eq:7} $\rightarrow$ \eqref{eq:2}, we get:
	\begin{equation}\label{eq:10}
	m u^{(m-2)} u_t - mu^{(m-1)} \triangle p - m^2u^{2(m-2)} |\triangledown u|^2 = 0
	\end{equation}
	Divide \eqref{eq:10} with $mu^{(m-2)}$, we get:
	\begin{equation}\label{eq:11}
	u_t - u \triangle p - m u^{(m-2)} |\triangledown u|^2 = 0
	\end{equation}
	Substitute \eqref{eq:8} $\rightarrow$ \eqref{eq:11}, we get:
	\begin{equation}\label{eq:12}
	\begin{aligned}
	u_t - u \bigg( m(m-2)u^{(m-3)} |\triangledown u|^2 + mu^{(m-2)} \triangle u \bigg) - m u^{(m-2)} |\triangledown u|^2 = 0\\
	u_t - \bigg( m(m-2)u^{(m-2)} |\triangledown u|^2 + mu^{(m-1)} \triangle u \bigg) - m u^{(m-2)} |\triangledown u|^2 = 0\\
	u_t - m u^{(m-2)} |\triangledown u|^2\bigg( (m-2) + 1 \bigg) - mu^{(m-1)} \triangle u = 0\\
	u_t - m(m-1) u^{(m-2)} |\triangledown u|^2 - mu^{(m-1)} \triangle u = 0\\
	\end{aligned}
	\end{equation}
	Substitute \eqref{eq:9} $\rightarrow$ \eqref{eq:12},
	\begin{equation}
	\therefore u_t - \triangle (u^m) = 0
	\end{equation}
\end{enumerate}
\end{document}