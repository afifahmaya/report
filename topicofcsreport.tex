\documentclass[a4paper,9pt]{article}
\usepackage[a4paper, hmargin={2cm,2cm}, vmargin={2cm,2cm}]{geometry}
\usepackage{amsmath}
\usepackage{amsthm}
\usepackage{amsfonts}
\usepackage{color}
\usepackage[final]{graphicx}
\usepackage{subcaption}
\usepackage{wrapfig}
\newtheorem{remark}{Remark}[]
\newtheorem{prop}{Proposition}
\newtheorem{prob}{Problem}
\newtheorem{defn}{Definition}[section]
\usepackage{amssymb}
\usepackage{verbatim}
\usepackage{listings}
\usepackage{multicol}
\usepackage[none]{hyphenat}

\newcommand{\R}{\mathbb{R}}
\newcommand{\C}{\mathbb{C}}
\newcommand{\N}{\mathbb{N}}
\newcommand{\Q}{\mathbb{Q}}
\newcommand{\W}{\mathbb{W}}
\newcommand{\Vspace}{\mathbb{V}}
\newcommand{\Hspace}{\mathbb{H}}
\newcommand{\Lspace}{\mathbb{L}}
\newcommand{\Lagr}{\mathcal{L}}


% Title Page
\title{Topics in Computational Science Report \\ Gibbs Energy in Isothermal-Isobaric Ensemble}
\author{Alifian Mahardhika Maulana}


\begin{document}
\maketitle
\begin{enumerate}
	\item Derive the following equation,
		\begin{equation}\label{eq:1}
		\begin{aligned}
		\triangle G(r) &\equiv G(r) - G(r_0)\\
		& = -\int_{r_0}^{r} \langle F(r) \rangle_{r=r'} dr'
		\end{aligned}
		\end{equation}
		\textbf{Answer:}\\
		\newline
		Suppose that,
		\begin{center}
			\begin{minipage}{0.29\linewidth}
				\begin{equation}\label{eq:2}
				G = -k_B T \ln Y_N(P,T)
				\end{equation}
			\end{minipage}
		\quad
			\begin{minipage}{0.67\linewidth}
				\begin{equation}\label{eq:3}
				Y_N(P,T) = \frac{1}{h^{3N}N!} \iiint\exp\Bigg(-\frac{H(\textbf{r}^N,\textbf{p}^N) + PV}{k_B T}\Bigg) d\textbf{r}^N d\textbf{p}^N dV
				\end{equation}
			\end{minipage}
		\end{center}
	We call \eqref{eq:2} Gibbs free energy and \eqref{eq:3} \textit{configurational integral}.
		Because \eqref{eq:3} is an indefinite integral, we can rewrite it as follow,
		\begin{equation}\label{eq:4}
		\begin{aligned}
		Y_N(P,T) &= \frac{1}{h^{3N}N!} \iiint\exp\bigg(-\frac{H(\textbf{r}^N,\textbf{p}^N) + PV}{k_B T}\bigg) dV d\textbf{r}^N d\textbf{p}^N\\
		&= -\frac{1}{h^{3N}N!}\frac{P}{k_B T} \iint\exp\bigg(-\frac{H(\textbf{r}^N,\textbf{p}^N) + PV}{k_B T}\bigg)d\textbf{r}^N d\textbf{p}^N
		\end{aligned}
		\end{equation}
		By using \textbf{Thermodynamic Integration},
		if the Gibbs free energy, $G$ is a continuous function of $r$, then we can write,
		\begin{equation}\label{eq:5}
		\triangle G(r) = \int_{r_0}^{r} \frac{d G(r)}{dr} dr
		\end{equation}
		Substitute \eqref{eq:2} to \eqref{eq:5} we get,
		\begin{equation}\label{eq:6}
		\triangle G(r) = \int_{r_0}^{r} -k_B T \frac{\partial \ln Y_N(P,T) }{\partial Y_N} \frac{\partial Y_N}{\partial r} = \int_{r_0}^{r} -k_B T \frac{1}{Y_N} \frac{\partial Y_N}{\partial r}
		\end{equation}
		From definition of $Y_N$ in \eqref{eq:4}, we can write the following for $\partial Y_N / \partial r$,
		\begin{equation}\label{eq:7}
		\frac{\partial Y_N}{\partial r} = -\frac{1}{h^{3N}N!}\frac{P}{k_B T} \iint \frac{\partial}{\partial r} \exp\bigg(-\frac{H(\textbf{r}^N,\textbf{p}^N) + PV}{k_B T}\bigg)d\textbf{r}^N d\textbf{p}^N
		\end{equation}
		Applying chain rule to \eqref{eq:7}, thus
		\begin{equation}\label{eq:8}
		\frac{\partial Y_N}{\partial r} = \frac{1}{h^{3N}N!}\frac{P}{k_B T}\frac{1}{k_B T} \iint \frac{\partial H(\textbf{r}^N,\textbf{p}^N)}{\partial r} \exp\bigg(-\frac{H(\textbf{r}^N,\textbf{p}^N) + PV}{k_B T}\bigg)d\textbf{r}^N d\textbf{p}^N
		\end{equation}
		Substitute \eqref{eq:8}, \eqref{eq:4} into \eqref{eq:6} gives:
		\begin{equation}\label{eq:9}
		\begin{aligned}
		\triangle G(r) &= \int_{r_0}^{r} k_B T \frac{k_B T h^{3N}N!}{P \iint\exp\bigg(-\frac{H(\textbf{r}^N,\textbf{p}^N) + PV}{k_B T}\bigg)d\textbf{r}^N d\textbf{p}^N} \frac{P \iint \frac{\partial H(\textbf{r}^N,\textbf{p}^N)}{\partial r} \exp\bigg(-\frac{H(\textbf{r}^N,\textbf{p}^N) + PV}{k_B T}\bigg)d\textbf{r}^N d\textbf{p}^N}{k_B^2 T^2 h^{3N}N!} dr\\
		&= \int_{r_0}^{r} \frac{\iint \frac{\partial H(\textbf{r}^N,\textbf{p}^N)}{\partial r} \exp\bigg(-\frac{H(\textbf{r}^N,\textbf{p}^N) + PV}{k_B T}\bigg)d\textbf{r}^N d\textbf{p}^N}{ \iint\exp\bigg(-\frac{H(\textbf{r}^N,\textbf{p}^N) + PV}{k_B T}\bigg)d\textbf{r}^N d\textbf{p}^N} dr
		\end{aligned}
		\end{equation}
		for simplicity, define
		\begin{equation}\label{eq:10}
		Z := \iint\exp\bigg(-\frac{H(\textbf{r}^N,\textbf{p}^N) + PV}{k_B T}\bigg)d\textbf{r}^N d\textbf{p}^N
		\end{equation}
		Then we can rewrite \eqref{eq:9} as follow,
		\begin{equation}\label{eq:11}
		\begin{aligned}
		\triangle G(r) = \int_{r_0}^{r} \bigg( \frac{1}{Z} \iint \frac{\partial H(\textbf{r}^N,\textbf{p}^N)}{\partial r} \exp\bigg(-\frac{H(\textbf{r}^N,\textbf{p}^N) + PV}{k_B T}\bigg)d\textbf{r}^N d\textbf{p}^N \bigg) dr
		\end{aligned}
		\end{equation}
		By \textbf{Ergodic Hypothesis} for expectation value of $X$,
		\begin{equation}\label{eq:12}
		\langle X \rangle = \frac{1}{Z} \iint X \exp\bigg(-\frac{H(\textbf{r}^N,\textbf{p}^N) + PV}{k_B T}\bigg)d\textbf{r}^N d\textbf{p}^N
		\end{equation}
		Thus \eqref{eq:11} become,
		\begin{equation}\label{eq:13}
		\begin{aligned}
		\triangle G(r) = \int_{r_0}^{r} \bigg\langle \frac{\partial H(\textbf{r}^N,\textbf{p}^N)}{\partial r} \bigg\rangle_{r=r'} dr'
		\end{aligned}
		\end{equation}
		Recalling \textbf{Hamiltonian} for Free Energy,
		\begin{equation}\label{eq:14}
		H(\textbf{r}^N,\textbf{p}^N) := K(\textbf{p}^N) + V(\textbf{r}^N)
		\end{equation}
		We consider our system as \textbf{Isothermal-Isobaric} system, the term with \textit{pressure} $\textbf{p}^N$ are constant, therefore we can rewrite \eqref{eq:13} as follow,
		\begin{equation}\label{eq:15}
		\triangle G(r) = \int_{r_0}^{r} \bigg\langle \frac{\partial V(r)}{\partial r} \bigg\rangle_{r=r'} dr'
		\end{equation}
		\textbf{Driving Force} in Molecular Dynamics is defined by,
		\begin{equation}\label{eq:16}
		F(r) := -\frac{\partial V(r)}{\partial r}
		\end{equation}
		Therefore,
		$$\triangle G(r) = -\int_{r_0}^{r} \langle F(r) \rangle_{r=r'} dr'$$
\end{enumerate}
\end{document}