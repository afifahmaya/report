\documentclass[a4paper,12pt]{article}
\usepackage[a4paper, hmargin={2cm,2cm}, vmargin={2cm,2cm}]{geometry}
\usepackage{amsmath}
\usepackage{amsthm}
\usepackage{amsfonts}
\usepackage{color}
\usepackage[final]{graphicx}
\usepackage{subcaption}
\usepackage{wrapfig}
\newtheorem{remark}{Remark}[]
\newtheorem{prop}{Proposition}
\newtheorem{prob}{Problem}
\newtheorem{defn}{Definition}[section]
\usepackage{amssymb}
\usepackage{verbatim}
\usepackage{listings}
\usepackage{multicol}
\usepackage[none]{hyphenat}

\newcommand{\R}{\mathbb{R}}
\newcommand{\C}{\mathbb{C}}
\newcommand{\N}{\mathbb{N}}
\newcommand{\Q}{\mathbb{Q}}
\newcommand{\W}{\mathbb{W}}
\newcommand{\Vspace}{\mathbb{V}}
\newcommand{\Hspace}{\mathbb{H}}
\newcommand{\Lspace}{\mathbb{L}}
\newcommand{\Lagr}{\mathcal{L}}


% Title Page
\title{Topics in Computational Science Report \\ Diffusion Equation}
\author{Alifian Mahardhika Maulana}


\begin{document}
\maketitle
\begin{enumerate}
	\item Suppose a Diffusion Equation in 1D defined by:
	\begin{equation}\label{eq:1}
	\frac{\partial \rho(x,t)}{\partial t} = D \frac{\partial^2 \rho(x,t)}{\partial x^2}
	\end{equation}
	with initial position (assumption), $x=0$, the solution for \eqref{eq:1} defined by:
	\begin{equation}\label{eq:2}
	\rho(x,t) = \frac{1}{\sqrt{4\pi D t}} e^{-\frac{x^2}{4Dt}}
	\end{equation}
	Show that \eqref{eq:2} is the solution for \eqref{eq:1}!\\
	\textbf{Answer:}\\
	First, taking first order partial time derivative of \eqref{eq:2} we get,
	\begin{equation}\label{eq:3}
	\begin{aligned}
	\rho_t(x,t) = \frac{\partial \rho(x,t)}{\partial t} &= -\frac{1}{2} 4\pi D (4\pi D t)^{-\frac{3}{2}} e^{-\frac{x^2}{4Dt}} + (4\pi D t)^{-\frac{1}{2}} \frac{x^2}{(4Dt)^2} 4D e^{-\frac{x^2}{4Dt}}\\
	&= \frac{1}{\sqrt{4\pi D t}} e^{-\frac{x^2}{4Dt}} \bigg( -\frac{1}{2} 4\pi D \frac{1}{4\pi Dt} + \frac{x^2}{(4Dt)^2}4D \bigg)\\
	&= \frac{1}{\sqrt{4\pi D t}} e^{-\frac{x^2}{4Dt}} \bigg( -\frac{1}{2t} + \frac{x^2}{4Dt^2} \bigg)
	\end{aligned}
	\end{equation}
	Then, taking first order partial space derivative of \eqref{eq:2} we get,
	\begin{equation}\label{eq:4}
	\begin{aligned}
	\rho_x(x,t) = \frac{\partial \rho(x,t)}{\partial x} &= -\frac{1}{\sqrt{4\pi D t}}\frac{2x}{4Dt} e^{-\frac{x^2}{4Dt}}\\
	&= -\frac{1}{\sqrt{4\pi D t}}\frac{x}{2Dt} e^{-\frac{x^2}{4Dt}}
	\end{aligned}
	\end{equation}
	After that, we take partial space derivative of \eqref{eq:4} we get,
	\begin{equation}\label{eq:5}
	\begin{aligned}
	\frac{\partial \rho_x(x,t)}{\partial x} = \frac{\partial^2 \rho(x,t)}{\partial x^2} &= -\frac{1}{\sqrt{4\pi D t}}e^{-\frac{x^2}{4Dt}} \frac{1}{2Dt} + \frac{1}{\sqrt{4\pi D t}} e^{-\frac{x^2}{4Dt}}\frac{x^2}{(2Dt)^2}\\
	&= \frac{1}{\sqrt{4\pi D t}} e^{-\frac{x^2}{4Dt}}\frac{1}{2Dt} \bigg( -1+\frac{x^2}{2Dt} \bigg)
	\end{aligned}
	\end{equation}
	Substitute \eqref{eq:5} and \eqref{eq:3} into \eqref{eq:1} we have,
	\begin{equation}\label{eq:6}
	\begin{aligned}
	\frac{\partial \rho(x,t)}{\partial t} &= D \frac{\partial^2 \rho(x,t)}{\partial x^2}\\
	\frac{1}{\sqrt{4\pi D t}} e^{-\frac{x^2}{4Dt}} \bigg( -\frac{1}{2t} + \frac{x^2}{4Dt^2} \bigg) &= D\frac{1}{\sqrt{4\pi D t}} e^{-\frac{x^2}{4Dt}}\frac{1}{2Dt} \bigg( -1+\frac{x^2}{2Dt} \bigg)\\
	\frac{1}{\sqrt{4\pi D t}} e^{-\frac{x^2}{4Dt}} \bigg( -\frac{1}{2t} + \frac{x^2}{4Dt^2} \bigg) &= \frac{1}{\sqrt{4\pi D t}} e^{-\frac{x^2}{4Dt}}\frac{1}{2t} \bigg( -1+\frac{x^2}{2Dt} \bigg)\\
	\frac{1}{\sqrt{4\pi D t}} e^{-\frac{x^2}{4Dt}} \bigg( -\frac{1}{2t} + \frac{x^2}{4Dt^2} \bigg) &= \frac{1}{\sqrt{4\pi D t}} e^{-\frac{x^2}{4Dt}} \bigg( -\frac{1}{2t}+\frac{x^2}{4Dt^2} \bigg)
	\end{aligned}
	\end{equation}
	By \eqref{eq:6} we have shown that \eqref{eq:2} satisfies equation \eqref{eq:1}, therefore it is proved that \eqref{eq:2} is the solution of \eqref{eq:1}.
	
	\item What did you study in the Topics in Computational Science Class (Professor Nagao Course)?\\
	\textbf{Answer:}\\
	In this class, we study about the behaviour of particle as a single and many particle system. As a single particle system, the particle moves to the other side with probability $P$ and the moves is a random walk with average of $\left< x(t) \right> = 0$. Movement of a particle system also can be explained by Langevin Equation defined by:
	\begin{equation}
	m \frac{d^2x}{dt^2} = F - \varsigma \frac{dx}{dt} + \xi (t)
	\end{equation}
	with:
	\begin{enumerate}
		\item $m \frac{d^2x}{dt^2}$: Newton's Equation of Motion
		\item $\varsigma \frac{dx}{dt}$: Frictional force proportional to velocity
		\item $\xi (t)$: Random force (on Brownian dynamics)
	\end{enumerate}
and average of the Random force, $\left< \xi(t) \right> = 0$ and $\varsigma$ is a constant number. While many particle system can be explained by Fick's Law with flux defined by:
\begin{equation}
J(x,t) = \rho(x,t) \left< \frac{dx}{dt} \right> = \rho(x,t) \frac{F(x,t)}{\varsigma}
\end{equation}
and Gradient of distribution of particle (density) is
\begin{equation}
J(x,t) = -D \frac{\partial \rho(x,t)}{\partial x} + \frac{F(x,t)}{\varsigma}, \quad F=0\rightarrow \text{Fick's law}
\end{equation}
We also learn about Smoluchowski's equation which is an extension from the diffusion equation.  At the last lecture, we also learn about the non-equilibrium system for example: biological system and cell universe. We can classify the non-equilibrium system by the relaxation process and the open system.\\
By relaxation process, we can classify non-equilibrium system into:
\begin{enumerate}
	\item Linear relaxation process $\Rightarrow$ fluctuation-dissipation theorem holds.
	\item Non-linear relaxation process : Internal degree of freedom
\end{enumerate}
By the open system, non-equilibrium classified into:
\begin{enumerate}
	\item Non-equilibrium steady state : steady state presence of gradient of temperature, or density, or pressure.
	\item Non-equilibrium unsteady state : Distribution change with time.
\end{enumerate}
\end{enumerate}
\end{document}