\documentclass[a4paper,12pt]{article}
\usepackage[a4paper, hmargin={3cm,2cm}, vmargin={2cm,2cm}]{geometry}
\usepackage{amsmath}
\usepackage{amsthm}
\usepackage{amsfonts}
\usepackage{color}
\usepackage[final]{graphicx}
\usepackage{subcaption}
\usepackage{wrapfig}
\newtheorem{remark}{Remark}[]
\newtheorem{prop}{Proposition}
\usepackage{amssymb}

\newcommand{\R}{\mathbb{R}}
\newcommand{\C}{\mathbb{C}}
\newcommand{\N}{\mathbb{N}}
\newcommand{\Q}{\mathbb{Q}}
\newcommand{\W}{\mathbb{W}}
\newcommand{\Vspace}{\mathbb{V}}
\newcommand{\Hspace}{\mathbb{H}}
\newcommand{\Lspace}{\mathbb{L}}
\newcommand{\Lagr}{\mathcal{L}}


% Title Page
\title{Assignment 1 Applied Computational Science}
\author{Alifian Mahardhika Maulana}


\begin{document}
\maketitle
\begin{enumerate}
	\item Suppose,
	\begin{equation}\label{rdef}
	r=\sqrt{x^2+y^2+z^2}
	\end{equation}
	then we define:
	\begin{equation}\label{fdef}
	f_x(r)=-\frac{\partial U(r)}{\partial x}\ \text{: x-component of the force.}
	\end{equation}
	with
	\begin{equation}\label{udef}
	U(r) = 4 \bigg(\frac{1}{r^{12}}-\frac{1}{r^6}\bigg)\ \text{: Lennard-Jones Potential System}
	\end{equation}
\end{enumerate}
Derive the $f_x,f_y,\ \text{and}\ f_z$ component of Lennard-Jones Potential System.\\
\newline
From Equation \eqref{fdef}, we know that:
\begin{equation}\label{derf}
\begin{aligned}
f_x(r) &=-\frac{\partial U(r)}{\partial x}\\
&= -\frac{\partial r}{\partial x}\frac{\partial U(r)}{\partial r}
\end{aligned}
\end{equation}
then we substitute $r$ and $U(r)$ from equation $\eqref{rdef}\ \text{and}\ \eqref{udef} \rightarrow\eqref{derf}$, therefore:
\begin{equation}\label{fxcomp}
\begin{aligned}
f_x(r) &= - \bigg( \frac{\partial}{\partial x} \sqrt{x^2+y^2+z^2}\ \frac{\partial}{\partial r} 4 \bigg(\frac{1}{r^{12}}-\frac{1}{r^6}\bigg) \bigg) \\
&= - \bigg( \frac{1}{2} \frac{1}{\sqrt{x^2+y^2+z^2}}2x\ \bigg( \frac{-48}{r^{13}}+\frac{24}{r^7} \bigg) \bigg)\\
&= - \bigg( \frac{x}{\sqrt{x^2+y^2+z^2}} \frac{48}{r} \bigg( \frac{-1}{r^{12}}+\frac{1}{2} \cdot \frac{1}{r^6} \bigg) \bigg)\\
&=  \frac{x}{r} \frac{48}{r} \bigg( \frac{1}{r^{12}}-\frac{1}{2} \cdot \frac{1}{r^6} \bigg)\\
&=  \frac{48x}{r^2} \bigg( \frac{1}{r^{12}}-\frac{1}{2} \cdot \frac{1}{r^6} \bigg)\\
\end{aligned}
\end{equation}
For the $f_y$ component, we define:
\begin{equation}\label{fydef}
\begin{aligned}
f_y(r) &=-\frac{\partial U(r)}{\partial y}\\
&= -\frac{\partial r}{\partial y}\frac{\partial U(r)}{\partial r}
\end{aligned}
\end{equation}
Again we substitute $r$ and $U(r)$ from equation $\eqref{rdef}\ \text{and}\ \eqref{udef} \rightarrow\eqref{fydef}$, thus:
\begin{equation}\label{fycomp}
\begin{aligned}
f_y(r) &= - \bigg( \frac{\partial}{\partial y} \sqrt{x^2+y^2+z^2}\ \frac{\partial}{\partial r} 4 \bigg(\frac{1}{r^{12}}-\frac{1}{r^6}\bigg) \bigg) \\
&= - \bigg( \frac{1}{2} \frac{1}{\sqrt{x^2+y^2+z^2}}2y\ \bigg( \frac{-48}{r^{13}}+\frac{24}{r^7} \bigg) \bigg)\\
&= - \bigg( \frac{x}{\sqrt{x^2+y^2+z^2}} \frac{48}{r} \bigg( \frac{-1}{r^{12}}+\frac{1}{2} \cdot \frac{1}{r^6} \bigg) \bigg)\\
&=  \frac{y}{r} \frac{48}{r} \bigg( \frac{1}{r^{12}}-\frac{1}{2} \cdot \frac{1}{r^6} \bigg)\\
&=  \frac{48y}{r^2} \bigg( \frac{1}{r^{12}}-\frac{1}{2} \cdot \frac{1}{r^6} \bigg)\\
\end{aligned}
\end{equation}
For the $f_z$ component, we define:
\begin{equation}\label{fzdef}
\begin{aligned}
f_z(r) &=-\frac{\partial U(r)}{\partial z}\\
&= -\frac{\partial r}{\partial z}\frac{\partial U(r)}{\partial r}
\end{aligned}
\end{equation}
Again we substitute $r$ and $U(r)$ from equation $\eqref{rdef}\ \text{and}\ \eqref{udef} \rightarrow\eqref{fzdef}$, thus:
\begin{equation}\label{fzcomp}
\begin{aligned}
f_z(r) &= - \bigg( \frac{\partial}{\partial z} \sqrt{x^2+y^2+z^2}\ \frac{\partial}{\partial r} 4 \bigg(\frac{1}{r^{12}}-\frac{1}{r^6}\bigg) \bigg) \\
&= - \bigg( \frac{1}{2} \frac{1}{\sqrt{x^2+y^2+z^2}}2z\ \bigg( \frac{-48}{r^{13}}+\frac{24}{r^7} \bigg) \bigg)\\
&= - \bigg( \frac{x}{\sqrt{x^2+y^2+z^2}} \frac{48}{r} \bigg( \frac{-1}{r^{12}}+\frac{1}{2} \cdot \frac{1}{r^6} \bigg) \bigg)\\
&=  \frac{z}{r} \frac{48}{r} \bigg( \frac{1}{r^{12}}-\frac{1}{2} \cdot \frac{1}{r^6} \bigg)\\
&=  \frac{48z}{r^2} \bigg( \frac{1}{r^{12}}-\frac{1}{2} \cdot \frac{1}{r^6} \bigg)\\
\end{aligned}
\end{equation}
\end{document}